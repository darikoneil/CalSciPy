%% Generated by Sphinx.
\def\sphinxdocclass{report}
\documentclass[letterpaper,10pt,english]{sphinxmanual}
\ifdefined\pdfpxdimen
   \let\sphinxpxdimen\pdfpxdimen\else\newdimen\sphinxpxdimen
\fi \sphinxpxdimen=.75bp\relax
\ifdefined\pdfimageresolution
    \pdfimageresolution= \numexpr \dimexpr1in\relax/\sphinxpxdimen\relax
\fi
%% let collapsible pdf bookmarks panel have high depth per default
\PassOptionsToPackage{bookmarksdepth=5}{hyperref}

\PassOptionsToPackage{booktabs}{sphinx}
\PassOptionsToPackage{colorrows}{sphinx}

\PassOptionsToPackage{warn}{textcomp}
\usepackage[utf8]{inputenc}
\ifdefined\DeclareUnicodeCharacter
% support both utf8 and utf8x syntaxes
  \ifdefined\DeclareUnicodeCharacterAsOptional
    \def\sphinxDUC#1{\DeclareUnicodeCharacter{"#1}}
  \else
    \let\sphinxDUC\DeclareUnicodeCharacter
  \fi
  \sphinxDUC{00A0}{\nobreakspace}
  \sphinxDUC{2500}{\sphinxunichar{2500}}
  \sphinxDUC{2502}{\sphinxunichar{2502}}
  \sphinxDUC{2514}{\sphinxunichar{2514}}
  \sphinxDUC{251C}{\sphinxunichar{251C}}
  \sphinxDUC{2572}{\textbackslash}
\fi
\usepackage{cmap}
\usepackage[T1]{fontenc}
\usepackage{amsmath,amssymb,amstext}
\usepackage{babel}



\usepackage{tgtermes}
\usepackage{tgheros}
\renewcommand{\ttdefault}{txtt}



\usepackage[Bjarne]{fncychap}
\usepackage{sphinx}

\fvset{fontsize=auto}
\usepackage{geometry}


% Include hyperref last.
\usepackage{hyperref}
% Fix anchor placement for figures with captions.
\usepackage{hypcap}% it must be loaded after hyperref.
% Set up styles of URL: it should be placed after hyperref.
\urlstyle{same}

\addto\captionsenglish{\renewcommand{\contentsname}{Contents:}}

\usepackage{sphinxmessages}
\setcounter{tocdepth}{1}



\title{CalSciPy}
\date{2023}
\release{0.3.0}
\author{Darik A.\@{} O\textquotesingle{}Neil}
\newcommand{\sphinxlogo}{\vbox{}}
\renewcommand{\releasename}{Release}
\makeindex
\begin{document}

\ifdefined\shorthandoff
  \ifnum\catcode`\=\string=\active\shorthandoff{=}\fi
  \ifnum\catcode`\"=\active\shorthandoff{"}\fi
\fi

\pagestyle{empty}
\sphinxmaketitle
\pagestyle{plain}
\sphinxtableofcontents
\pagestyle{normal}
\phantomsection\label{\detokenize{index::doc}}


\sphinxstepscope


\chapter{Introduction}
\label{\detokenize{Introduction:introduction}}\label{\detokenize{Introduction::doc}}
\sphinxAtStartPar
\sphinxstylestrong{CalSciPy} contains a variety of useful methods for handling, processing, and visualizing calcium imaging data.
It’s intended to be a collection of useful, well\sphinxhyphen{}documented functions often used in boilerplate code alongside software
packages such as \sphinxhref{https://github.com/flatironinstitute/CaImAn}{Caiman}, \sphinxhref{https://github.com/losonczylab/sima}{SIMA},
and \sphinxhref{https://github.com/MouseLand/suite2p}{Suite2P}.


\section{Motivation}
\label{\detokenize{Introduction:motivation}}
\sphinxAtStartPar
I noticed I was often re\sphinxhyphen{}writing or copy/pasting a lot of code between environments when working with calcium imaging
data. I started this package \textasciitilde{}so I don’t have to\textasciitilde{} so you don’t have to. No more wasting time writing 6 lines to simply
preview your tiff stack, extract a particular channel, or bin some spikes. No more vague exceptions or incomplete
documentation when re\sphinxhyphen{}using a hastily\sphinxhyphen{}made function from 2 months ago. Alongside these time\sphinxhyphen{}savers, I’ve also included
some more non\sphinxhyphen{}trivial methods that are particularly useful.


\section{Limitations}
\label{\detokenize{Introduction:limitations}}
\sphinxAtStartPar
The current distribution for the package is incomplete and partially tested

\sphinxstepscope


\chapter{Installation}
\label{\detokenize{Installation:installation}}\label{\detokenize{Installation::doc}}

\section{Full Install}
\label{\detokenize{Installation:full-install}}
\sphinxAtStartPar
Enter \sphinxcode{\sphinxupquote{pip install CalSciPy}} in your terminal.


\section{Partial Install}
\label{\detokenize{Installation:partial-install}}
\sphinxAtStartPar
Enter \sphinxcode{\sphinxupquote{pip install CalSciPy\sphinxhyphen{}\textless{}subpackage\textgreater{}}} in your terminal.

\sphinxstepscope


\chapter{Overview}
\label{\detokenize{Sub-Packages:overview}}\label{\detokenize{Sub-Packages::doc}}\begin{itemize}
\item {} 
\sphinxAtStartPar
{\hyperref[\detokenize{Sub-Packages:bruker-module}]{\sphinxcrossref{\DUrole{std,std-ref}{Bruker}}}}

\item {} 
\sphinxAtStartPar
{\hyperref[\detokenize{Sub-Packages:coloring-module}]{\sphinxcrossref{\DUrole{std,std-ref}{Coloring}}}}

\item {} 
\sphinxAtStartPar
{\hyperref[\detokenize{Sub-Packages:event-processing-module}]{\sphinxcrossref{\DUrole{std,std-ref}{Event Processing}}}}

\item {} 
\sphinxAtStartPar
{\hyperref[\detokenize{Sub-Packages:io-module}]{\sphinxcrossref{\DUrole{std,std-ref}{Input/Output Tools (I/O)}}}}

\item {} 
\sphinxAtStartPar
{\hyperref[\detokenize{Sub-Packages:image-processing-module}]{\sphinxcrossref{\DUrole{std,std-ref}{Image Processing}}}}

\item {} 
\sphinxAtStartPar
{\hyperref[\detokenize{Sub-Packages:interactive-visuals-module}]{\sphinxcrossref{\DUrole{std,std-ref}{Interactive Visuals}}}}

\item {} 
\sphinxAtStartPar
{\hyperref[\detokenize{Sub-Packages:miscellaneous-module}]{\sphinxcrossref{\DUrole{std,std-ref}{Miscellaneous}}}}

\item {} 
\sphinxAtStartPar
{\hyperref[\detokenize{Sub-Packages:reorganization-module}]{\sphinxcrossref{\DUrole{std,std-ref}{Reorganization}}}}

\item {} 
\sphinxAtStartPar
{\hyperref[\detokenize{Sub-Packages:trace-processing-module}]{\sphinxcrossref{\DUrole{std,std-ref}{Trace Processing}}}}

\item {} 
\sphinxAtStartPar
{\hyperref[\detokenize{Sub-Packages:version-module}]{\sphinxcrossref{\DUrole{std,std-ref}{Version}}}}

\end{itemize}


\section{Bruker}
\label{\detokenize{Sub-Packages:bruker}}\label{\detokenize{Sub-Packages:bruker-module}}
\begin{DUlineblock}{0em}
\item[] Write me
\item[] Write me
\item[] Write me
\item[] Write me
\end{DUlineblock}

\sphinxstepscope


\subsection{CalSciPy.bruker module}
\label{\detokenize{CalSciPy.bruker:module-CalSciPy.bruker}}\label{\detokenize{CalSciPy.bruker:calscipy-bruker-module}}\label{\detokenize{CalSciPy.bruker::doc}}\index{module@\spxentry{module}!CalSciPy.bruker@\spxentry{CalSciPy.bruker}}\index{CalSciPy.bruker@\spxentry{CalSciPy.bruker}!module@\spxentry{module}}\index{determine\_imaging\_content() (in module CalSciPy.bruker)@\spxentry{determine\_imaging\_content()}\spxextra{in module CalSciPy.bruker}}

\begin{fulllineitems}
\phantomsection\label{\detokenize{CalSciPy.bruker:CalSciPy.bruker.determine_imaging_content}}
\pysigstartsignatures
\pysiglinewithargsret{\sphinxcode{\sphinxupquote{CalSciPy.bruker.}}\sphinxbfcode{\sphinxupquote{determine\_imaging\_content}}}{\emph{\DUrole{n}{folder}}}{}
\pysigstopsignatures
\sphinxAtStartPar
This function determines the number of channels and planes within a folder containing .tif files
exported by Bruker’s Prairieview software. It also determines the size of the images (frames, y\sphinxhyphen{}pixels, x\sphinxhyphen{}pixels).
It’s a quick / fast alternative to parsing its respective xml. Dependent on the naming conventions of PrairieView.
\begin{quote}\begin{description}
\sphinxlineitem{Parameters}
\sphinxAtStartPar
\sphinxstyleliteralstrong{\sphinxupquote{folder}} (\sphinxhref{https://docs.python.org/3/library/stdtypes.html\#str}{\sphinxstyleliteralemphasis{\sphinxupquote{str}}}\sphinxstyleliteralemphasis{\sphinxupquote{ or }}\sphinxhref{https://docs.python.org/3/library/pathlib.html\#pathlib.Path}{\sphinxstyleliteralemphasis{\sphinxupquote{pathlib.Path}}}) \textendash{} folder containing bruker imaging data

\sphinxlineitem{Returns}
\sphinxAtStartPar
channels, planes, frames, height, width

\sphinxlineitem{Return type}
\sphinxAtStartPar
\sphinxhref{https://docs.python.org/3/library/stdtypes.html\#tuple}{tuple}{[}\sphinxhref{https://docs.python.org/3/library/functions.html\#int}{int}, \sphinxhref{https://docs.python.org/3/library/functions.html\#int}{int}, \sphinxhref{https://docs.python.org/3/library/functions.html\#int}{int}, \sphinxhref{https://docs.python.org/3/library/functions.html\#int}{int}, \sphinxhref{https://docs.python.org/3/library/functions.html\#int}{int}{]}

\end{description}\end{quote}

\end{fulllineitems}

\index{generate\_bruker\_naming\_convention() (in module CalSciPy.bruker)@\spxentry{generate\_bruker\_naming\_convention()}\spxextra{in module CalSciPy.bruker}}

\begin{fulllineitems}
\phantomsection\label{\detokenize{CalSciPy.bruker:CalSciPy.bruker.generate_bruker_naming_convention}}
\pysigstartsignatures
\pysiglinewithargsret{\sphinxcode{\sphinxupquote{CalSciPy.bruker.}}\sphinxbfcode{\sphinxupquote{generate\_bruker\_naming\_convention}}}{\emph{\DUrole{n}{channel}\DUrole{o}{=}\DUrole{default_value}{0}}, \emph{\DUrole{n}{plane}\DUrole{o}{=}\DUrole{default_value}{0}}, \emph{\DUrole{n}{num\_channels}\DUrole{o}{=}\DUrole{default_value}{1}}, \emph{\DUrole{n}{num\_planes}\DUrole{o}{=}\DUrole{default_value}{1}}}{}
\pysigstopsignatures
\sphinxAtStartPar
Generates the expected bruker naming convention for images collected with an arbitrary number of cycles \& channels

\sphinxAtStartPar
This function expects that the naming convention is \{experiment\_name\}\_Cycle00000\_Ch0\_000000.ome.tiff
where the channel is one\sphinxhyphen{}indexed. The 5\sphinxhyphen{}digit cycle id represents the frame if using multiplane imaging and
the 6\sphinxhyphen{}digit tag represents the plane. Otherwise, the 5\sphinxhyphen{}digit tag is static and the 6\sphinxhyphen{}digit tag represents the frame.
Channel and plane are \sphinxstyleemphasis{zero\sphinxhyphen{}indexed}.
\begin{quote}\begin{description}
\sphinxlineitem{Parameters}\begin{itemize}
\item {} 
\sphinxAtStartPar
\sphinxstyleliteralstrong{\sphinxupquote{channel}} (\sphinxstyleliteralemphasis{\sphinxupquote{int = 0}}) \textendash{} channel to produce name for

\item {} 
\sphinxAtStartPar
\sphinxstyleliteralstrong{\sphinxupquote{plane}} (\sphinxstyleliteralemphasis{\sphinxupquote{int = 0}}) \textendash{} plane to produce name for

\item {} 
\sphinxAtStartPar
\sphinxstyleliteralstrong{\sphinxupquote{num\_channels}} (\sphinxstyleliteralemphasis{\sphinxupquote{int = 1}}) \textendash{} number of channels

\item {} 
\sphinxAtStartPar
\sphinxstyleliteralstrong{\sphinxupquote{num\_planes}} (\sphinxstyleliteralemphasis{\sphinxupquote{int = 1}}) \textendash{} number of planes

\end{itemize}

\sphinxlineitem{Return type}
\sphinxAtStartPar
\sphinxhref{https://docs.python.org/3/library/stdtypes.html\#str}{\sphinxcode{\sphinxupquote{str}}}

\sphinxlineitem{Returns}
\sphinxAtStartPar


\end{description}\end{quote}

\end{fulllineitems}

\index{load\_bruker\_tifs() (in module CalSciPy.bruker)@\spxentry{load\_bruker\_tifs()}\spxextra{in module CalSciPy.bruker}}

\begin{fulllineitems}
\phantomsection\label{\detokenize{CalSciPy.bruker:CalSciPy.bruker.load_bruker_tifs}}
\pysigstartsignatures
\pysiglinewithargsret{\sphinxcode{\sphinxupquote{CalSciPy.bruker.}}\sphinxbfcode{\sphinxupquote{load\_bruker\_tifs}}}{\emph{\DUrole{n}{folder}}, \emph{\DUrole{n}{channel}\DUrole{o}{=}\DUrole{default_value}{None}}, \emph{\DUrole{n}{plane}\DUrole{o}{=}\DUrole{default_value}{None}}}{}
\pysigstopsignatures
\sphinxAtStartPar
Load images collected and exported to .tif by Bruker’s Prairieview software to a tuple of numpy arrays.
If multiple channels or multiple planes exist, each channel and plane combination is loaded to a separate
numpy array. Identification of multiple channels / planes is dependent on {\hyperref[\detokenize{CalSciPy.bruker:CalSciPy.bruker.determine_imaging_content}]{\sphinxcrossref{\sphinxcode{\sphinxupquote{determine\_imaging\_content()}}}}}.
Images are loaded as unsigned 16\sphinxhyphen{}bit, though note that raw bruker files are natively 12 or 13\sphinxhyphen{}bit.
\begin{quote}\begin{description}
\sphinxlineitem{Parameters}\begin{itemize}
\item {} 
\sphinxAtStartPar
\sphinxstyleliteralstrong{\sphinxupquote{folder}} (\sphinxhref{https://docs.python.org/3/library/stdtypes.html\#str}{\sphinxstyleliteralemphasis{\sphinxupquote{str}}}\sphinxstyleliteralemphasis{\sphinxupquote{ or }}\sphinxhref{https://docs.python.org/3/library/pathlib.html\#pathlib.Path}{\sphinxstyleliteralemphasis{\sphinxupquote{pathlib.Path}}}) \textendash{} folder containing a sequence of single frame tiff files

\item {} 
\sphinxAtStartPar
\sphinxstyleliteralstrong{\sphinxupquote{channel}} (\sphinxstyleliteralemphasis{\sphinxupquote{Optional}}\sphinxstyleliteralemphasis{\sphinxupquote{{[}}}\sphinxhref{https://docs.python.org/3/library/functions.html\#int}{\sphinxstyleliteralemphasis{\sphinxupquote{int}}}\sphinxstyleliteralemphasis{\sphinxupquote{{]} }}\sphinxstyleliteralemphasis{\sphinxupquote{= None}}) \textendash{} specific channel to load from dataset (zero\sphinxhyphen{}indexed)

\item {} 
\sphinxAtStartPar
\sphinxstyleliteralstrong{\sphinxupquote{plane}} (\sphinxstyleliteralemphasis{\sphinxupquote{Optional}}\sphinxstyleliteralemphasis{\sphinxupquote{{[}}}\sphinxhref{https://docs.python.org/3/library/functions.html\#int}{\sphinxstyleliteralemphasis{\sphinxupquote{int}}}\sphinxstyleliteralemphasis{\sphinxupquote{{]} }}\sphinxstyleliteralemphasis{\sphinxupquote{= None}}) \textendash{} specific plane to load from dataset (zero\sphinxhyphen{}indexed)

\end{itemize}

\sphinxlineitem{Returns}
\sphinxAtStartPar
All .tif files in the directory loaded to a tuple of numpy arrays
(frames, y\sphinxhyphen{}pixels, x\sphinxhyphen{}pixels, \sphinxcode{\sphinxupquote{np.uint16}})

\sphinxlineitem{Return type}
\sphinxAtStartPar
\sphinxhref{https://docs.python.org/3/library/stdtypes.html\#tuple}{tuple}{[}\sphinxhref{https://numpy.org/doc/1.24/reference/generated/numpy.ndarray.html\#numpy.ndarray}{numpy.ndarray}{]}

\end{description}\end{quote}

\end{fulllineitems}

\index{repackage\_bruker\_tifs() (in module CalSciPy.bruker)@\spxentry{repackage\_bruker\_tifs()}\spxextra{in module CalSciPy.bruker}}

\begin{fulllineitems}
\phantomsection\label{\detokenize{CalSciPy.bruker:CalSciPy.bruker.repackage_bruker_tifs}}
\pysigstartsignatures
\pysiglinewithargsret{\sphinxcode{\sphinxupquote{CalSciPy.bruker.}}\sphinxbfcode{\sphinxupquote{repackage\_bruker\_tifs}}}{\emph{\DUrole{n}{input\_folder}}, \emph{\DUrole{n}{output\_folder}}, \emph{\DUrole{n}{channel}\DUrole{o}{=}\DUrole{default_value}{0}}, \emph{\DUrole{n}{plane}\DUrole{o}{=}\DUrole{default_value}{0}}}{}
\pysigstopsignatures
\sphinxAtStartPar
Repackages a folder containing .tif files exported by Bruker’s Prairieview software into a sequence of \textless{}4 GB .tif
stacks. Channels are planes are \sphinxstylestrong{zero\sphinxhyphen{}indexed}.
\begin{quote}\begin{description}
\sphinxlineitem{Parameters}\begin{itemize}
\item {} 
\sphinxAtStartPar
\sphinxstyleliteralstrong{\sphinxupquote{input\_folder}} (\sphinxhref{https://docs.python.org/3/library/stdtypes.html\#str}{\sphinxstyleliteralemphasis{\sphinxupquote{str}}}\sphinxstyleliteralemphasis{\sphinxupquote{ or }}\sphinxhref{https://docs.python.org/3/library/pathlib.html\#pathlib.Path}{\sphinxstyleliteralemphasis{\sphinxupquote{pathlib.Path}}}) \textendash{} folder containing a sequence of single frame .tif files exported by Bruker’s Prairieview

\item {} 
\sphinxAtStartPar
\sphinxstyleliteralstrong{\sphinxupquote{output\_folder}} (\sphinxhref{https://docs.python.org/3/library/stdtypes.html\#str}{\sphinxstyleliteralemphasis{\sphinxupquote{str}}}\sphinxstyleliteralemphasis{\sphinxupquote{ or }}\sphinxhref{https://docs.python.org/3/library/pathlib.html\#pathlib.Path}{\sphinxstyleliteralemphasis{\sphinxupquote{pathlib.Path}}}) \textendash{} empty folder where .tif stacks will be saved

\item {} 
\sphinxAtStartPar
\sphinxstyleliteralstrong{\sphinxupquote{channel}} (\sphinxstyleliteralemphasis{\sphinxupquote{int = 0}}) \textendash{} optional input specifying channel

\item {} 
\sphinxAtStartPar
\sphinxstyleliteralstrong{\sphinxupquote{plane}} (\sphinxstyleliteralemphasis{\sphinxupquote{int = 0}}) \textendash{} optional input specifying plane

\end{itemize}

\sphinxlineitem{Return type}
\sphinxAtStartPar
None

\end{description}\end{quote}

\end{fulllineitems}



\section{Coloring}
\label{\detokenize{Sub-Packages:coloring}}\label{\detokenize{Sub-Packages:coloring-module}}
\begin{DUlineblock}{0em}
\item[] Write me
\item[] Write me
\item[] Write me
\item[] Write me
\end{DUlineblock}


\subsection{Coloring Methods}
\label{\detokenize{Sub-Packages:coloring-methods}}
\begin{DUlineblock}{0em}
\item[] Import me
\end{DUlineblock}


\section{Event Processing}
\label{\detokenize{Sub-Packages:event-processing}}\label{\detokenize{Sub-Packages:event-processing-module}}
\begin{DUlineblock}{0em}
\item[] Write me
\item[] Write me
\item[] Write me
\item[] Write me
\end{DUlineblock}

\sphinxstepscope


\subsection{CalSciPy.event\_processing module}
\label{\detokenize{CalSciPy.event_processing:module-CalSciPy.event_processing}}\label{\detokenize{CalSciPy.event_processing:calscipy-event-processing-module}}\label{\detokenize{CalSciPy.event_processing::doc}}\index{module@\spxentry{module}!CalSciPy.event\_processing@\spxentry{CalSciPy.event\_processing}}\index{CalSciPy.event\_processing@\spxentry{CalSciPy.event\_processing}!module@\spxentry{module}}\index{calculate\_firing\_rates() (in module CalSciPy.event\_processing)@\spxentry{calculate\_firing\_rates()}\spxextra{in module CalSciPy.event\_processing}}

\begin{fulllineitems}
\phantomsection\label{\detokenize{CalSciPy.event_processing:CalSciPy.event_processing.calculate_firing_rates}}
\pysigstartsignatures
\pysiglinewithargsret{\sphinxcode{\sphinxupquote{CalSciPy.event\_processing.}}\sphinxbfcode{\sphinxupquote{calculate\_firing\_rates}}}{\emph{\DUrole{n}{spike\_probability\_matrix}}, \emph{\DUrole{n}{frame\_rate}\DUrole{o}{=}\DUrole{default_value}{30.0}}, \emph{\DUrole{n}{in\_place}\DUrole{o}{=}\DUrole{default_value}{False}}}{}
\pysigstopsignatures
\sphinxAtStartPar
Calculate firing rates
\begin{quote}\begin{description}
\sphinxlineitem{Parameters}\begin{itemize}
\item {} 
\sphinxAtStartPar
\sphinxstyleliteralstrong{\sphinxupquote{spike\_probability\_matrix}} (\sphinxhref{https://numpy.org/doc/1.24/reference/generated/numpy.ndarray.html\#numpy.ndarray}{\sphinxstyleliteralemphasis{\sphinxupquote{numpy.ndarray}}}) \textendash{} matrix of n neuron x m samples where each element is the probability of a spike

\item {} 
\sphinxAtStartPar
\sphinxstyleliteralstrong{\sphinxupquote{frame\_rate}} (\sphinxstyleliteralemphasis{\sphinxupquote{float = 30}}) \textendash{} frame rate of dataset

\item {} 
\sphinxAtStartPar
\sphinxstyleliteralstrong{\sphinxupquote{in\_place}} (\sphinxstyleliteralemphasis{\sphinxupquote{bool = False}}) \textendash{} boolean indicating whether to perform calculation in\sphinxhyphen{}place

\end{itemize}

\sphinxlineitem{Returns}
\sphinxAtStartPar
firing matrix of n neurons x m samples where each element is a binary indicating presence of spike event

\sphinxlineitem{Return type}
\sphinxAtStartPar
\sphinxhref{https://numpy.org/doc/1.24/reference/generated/numpy.ndarray.html\#numpy.ndarray}{numpy.ndarray}

\end{description}\end{quote}

\end{fulllineitems}

\index{calculate\_mean\_firing\_rates() (in module CalSciPy.event\_processing)@\spxentry{calculate\_mean\_firing\_rates()}\spxextra{in module CalSciPy.event\_processing}}

\begin{fulllineitems}
\phantomsection\label{\detokenize{CalSciPy.event_processing:CalSciPy.event_processing.calculate_mean_firing_rates}}
\pysigstartsignatures
\pysiglinewithargsret{\sphinxcode{\sphinxupquote{CalSciPy.event\_processing.}}\sphinxbfcode{\sphinxupquote{calculate\_mean\_firing\_rates}}}{\emph{\DUrole{n}{firing\_matrix}}}{}
\pysigstopsignatures
\sphinxAtStartPar
Calculate mean firing rate
\begin{quote}\begin{description}
\sphinxlineitem{Parameters}
\sphinxAtStartPar
\sphinxstyleliteralstrong{\sphinxupquote{firing\_matrix}} (\sphinxhref{https://numpy.org/doc/1.24/reference/generated/numpy.ndarray.html\#numpy.ndarray}{\sphinxstyleliteralemphasis{\sphinxupquote{numpy.ndarray}}}) \textendash{} matrix of n neuron x m samples where each element is either a spike or an
instantaneous firing rate

\sphinxlineitem{Returns}
\sphinxAtStartPar
1\sphinxhyphen{}D vector of mean firing rates

\sphinxlineitem{Return type}
\sphinxAtStartPar
\sphinxhref{https://numpy.org/doc/1.24/reference/generated/numpy.ndarray.html\#numpy.ndarray}{numpy.ndarray}

\end{description}\end{quote}

\end{fulllineitems}

\index{collect\_waveforms() (in module CalSciPy.event\_processing)@\spxentry{collect\_waveforms()}\spxextra{in module CalSciPy.event\_processing}}

\begin{fulllineitems}
\phantomsection\label{\detokenize{CalSciPy.event_processing:CalSciPy.event_processing.collect_waveforms}}
\pysigstartsignatures
\pysiglinewithargsret{\sphinxcode{\sphinxupquote{CalSciPy.event\_processing.}}\sphinxbfcode{\sphinxupquote{collect\_waveforms}}}{\emph{\DUrole{n}{traces}}, \emph{\DUrole{n}{event\_indices}}, \emph{\DUrole{n}{pre}\DUrole{o}{=}\DUrole{default_value}{150}}, \emph{\DUrole{n}{post}\DUrole{o}{=}\DUrole{default_value}{450}}}{}
\pysigstopsignatures
\sphinxAtStartPar
Collect waveforms for each event
\begin{quote}\begin{description}
\sphinxlineitem{Parameters}\begin{itemize}
\item {} 
\sphinxAtStartPar
\sphinxstyleliteralstrong{\sphinxupquote{traces}} (\sphinxhref{https://numpy.org/doc/1.24/reference/generated/numpy.ndarray.html\#numpy.ndarray}{\sphinxstyleliteralemphasis{\sphinxupquote{numpy.ndarray}}}) \textendash{} a matrix of M neurons x N samples

\item {} 
\sphinxAtStartPar
\sphinxstyleliteralstrong{\sphinxupquote{event\_indices}} (\sphinxstyleliteralemphasis{\sphinxupquote{Iterable}}\sphinxstyleliteralemphasis{\sphinxupquote{{[}}}\sphinxstyleliteralemphasis{\sphinxupquote{Iterable}}\sphinxstyleliteralemphasis{\sphinxupquote{{[}}}\sphinxhref{https://docs.python.org/3/library/functions.html\#int}{\sphinxstyleliteralemphasis{\sphinxupquote{int}}}\sphinxstyleliteralemphasis{\sphinxupquote{{]}}}\sphinxstyleliteralemphasis{\sphinxupquote{{]}}}) \textendash{} a list of events

\item {} 
\sphinxAtStartPar
\sphinxstyleliteralstrong{\sphinxupquote{pre}} (\sphinxhref{https://docs.python.org/3/library/functions.html\#int}{\sphinxstyleliteralemphasis{\sphinxupquote{int}}}) \textendash{} number of pre\sphinxhyphen{}event frames

\item {} 
\sphinxAtStartPar
\sphinxstyleliteralstrong{\sphinxupquote{post}} (\sphinxhref{https://docs.python.org/3/library/functions.html\#int}{\sphinxstyleliteralemphasis{\sphinxupquote{int}}}) \textendash{} number of post\sphinxhyphen{}event frames

\end{itemize}

\sphinxlineitem{Returns}
\sphinxAtStartPar
a matrix of M events x N samples

\sphinxlineitem{Return type}
\sphinxAtStartPar
Tuple{[}\sphinxhref{https://numpy.org/doc/1.24/reference/generated/numpy.ndarray.html\#numpy.ndarray}{numpy.ndarray}{]}

\end{description}\end{quote}

\end{fulllineitems}

\index{convert\_tau() (in module CalSciPy.event\_processing)@\spxentry{convert\_tau()}\spxextra{in module CalSciPy.event\_processing}}

\begin{fulllineitems}
\phantomsection\label{\detokenize{CalSciPy.event_processing:CalSciPy.event_processing.convert_tau}}
\pysigstartsignatures
\pysiglinewithargsret{\sphinxcode{\sphinxupquote{CalSciPy.event\_processing.}}\sphinxbfcode{\sphinxupquote{convert\_tau}}}{\emph{\DUrole{n}{tau}}, \emph{\DUrole{n}{dt}}}{}
\pysigstopsignatures
\sphinxAtStartPar
Converts a discrete tau to a continuous tau
\begin{quote}\begin{description}
\sphinxlineitem{Parameters}\begin{itemize}
\item {} 
\sphinxAtStartPar
\sphinxstyleliteralstrong{\sphinxupquote{tau}} (\sphinxhref{https://docs.python.org/3/library/functions.html\#float}{\sphinxcode{\sphinxupquote{float}}}) \textendash{} decay constant

\item {} 
\sphinxAtStartPar
\sphinxstyleliteralstrong{\sphinxupquote{dt}} (\sphinxhref{https://docs.python.org/3/library/functions.html\#float}{\sphinxstyleliteralemphasis{\sphinxupquote{float}}}) \textendash{} time step (s)

\end{itemize}

\sphinxlineitem{Returns}
\sphinxAtStartPar
continuous tau (s)

\sphinxlineitem{Return type}
\sphinxAtStartPar
\sphinxhref{https://docs.python.org/3/library/functions.html\#float}{float}

\end{description}\end{quote}

\end{fulllineitems}

\index{get\_event\_onset\_intensities() (in module CalSciPy.event\_processing)@\spxentry{get\_event\_onset\_intensities()}\spxextra{in module CalSciPy.event\_processing}}

\begin{fulllineitems}
\phantomsection\label{\detokenize{CalSciPy.event_processing:CalSciPy.event_processing.get_event_onset_intensities}}
\pysigstartsignatures
\pysiglinewithargsret{\sphinxcode{\sphinxupquote{CalSciPy.event\_processing.}}\sphinxbfcode{\sphinxupquote{get\_event\_onset\_intensities}}}{\emph{\DUrole{n}{traces}}, \emph{\DUrole{n}{event\_indices}}}{}
\pysigstopsignatures
\sphinxAtStartPar
Retrieve the signal intensity at event onset for each neuron in the event indices
\begin{quote}\begin{description}
\sphinxlineitem{Parameters}\begin{itemize}
\item {} 
\sphinxAtStartPar
\sphinxstyleliteralstrong{\sphinxupquote{traces}} (\sphinxhref{https://numpy.org/doc/1.24/reference/generated/numpy.ndarray.html\#numpy.ndarray}{\sphinxstyleliteralemphasis{\sphinxupquote{numpy.ndarray}}}) \textendash{} An M neuron by N sample matrix

\item {} 
\sphinxAtStartPar
\sphinxstyleliteralstrong{\sphinxupquote{event\_indices}} (\sphinxstyleliteralemphasis{\sphinxupquote{Iterable}}\sphinxstyleliteralemphasis{\sphinxupquote{{[}}}\sphinxstyleliteralemphasis{\sphinxupquote{Iterable}}\sphinxstyleliteralemphasis{\sphinxupquote{{[}}}\sphinxhref{https://docs.python.org/3/library/functions.html\#int}{\sphinxstyleliteralemphasis{\sphinxupquote{int}}}\sphinxstyleliteralemphasis{\sphinxupquote{{]}}}\sphinxstyleliteralemphasis{\sphinxupquote{{]}}}) \textendash{} An iterable of length M containing a sequence with a duration for each event

\end{itemize}

\sphinxlineitem{Returns}
\sphinxAtStartPar
An iterable of length M neurons containing the onset intensities for each event in the sequence

\sphinxlineitem{Return type}
\sphinxAtStartPar
Tuple{[}\sphinxhref{https://numpy.org/doc/1.24/reference/generated/numpy.ndarray.html\#numpy.ndarray}{numpy.ndarray}{]}

\end{description}\end{quote}

\end{fulllineitems}

\index{get\_inter\_event\_intervals() (in module CalSciPy.event\_processing)@\spxentry{get\_inter\_event\_intervals()}\spxextra{in module CalSciPy.event\_processing}}

\begin{fulllineitems}
\phantomsection\label{\detokenize{CalSciPy.event_processing:CalSciPy.event_processing.get_inter_event_intervals}}
\pysigstartsignatures
\pysiglinewithargsret{\sphinxcode{\sphinxupquote{CalSciPy.event\_processing.}}\sphinxbfcode{\sphinxupquote{get\_inter\_event\_intervals}}}{\emph{\DUrole{n}{event\_indices}}, \emph{\DUrole{n}{frame\_rate}\DUrole{o}{=}\DUrole{default_value}{30.0}}}{}
\pysigstopsignatures
\sphinxAtStartPar
Calculate the inter event intervals for each neuron in the event indices
\begin{quote}\begin{description}
\sphinxlineitem{Parameters}\begin{itemize}
\item {} 
\sphinxAtStartPar
\sphinxstyleliteralstrong{\sphinxupquote{event\_indices}} (\sphinxstyleliteralemphasis{\sphinxupquote{Iterable}}\sphinxstyleliteralemphasis{\sphinxupquote{{[}}}\sphinxstyleliteralemphasis{\sphinxupquote{Iterable}}\sphinxstyleliteralemphasis{\sphinxupquote{{[}}}\sphinxhref{https://docs.python.org/3/library/functions.html\#int}{\sphinxstyleliteralemphasis{\sphinxupquote{int}}}\sphinxstyleliteralemphasis{\sphinxupquote{{]}}}\sphinxstyleliteralemphasis{\sphinxupquote{{]}}}) \textendash{} An iterable of length M containing a sequence with a duration for each event

\item {} 
\sphinxAtStartPar
\sphinxstyleliteralstrong{\sphinxupquote{frame\_rate}} (\sphinxhref{https://docs.python.org/3/library/functions.html\#float}{\sphinxstyleliteralemphasis{\sphinxupquote{float}}}) \textendash{} frame\_rate for trace matrix

\end{itemize}

\sphinxlineitem{Returns}
\sphinxAtStartPar
An iterable of length M neurons containing the inter\sphinxhyphen{}event intervals for each event in the sequence

\sphinxlineitem{Return type}
\sphinxAtStartPar
Tuple{[}\sphinxhref{https://numpy.org/doc/1.24/reference/generated/numpy.ndarray.html\#numpy.ndarray}{numpy.ndarray}{]}

\end{description}\end{quote}

\end{fulllineitems}

\index{get\_num\_events() (in module CalSciPy.event\_processing)@\spxentry{get\_num\_events()}\spxextra{in module CalSciPy.event\_processing}}

\begin{fulllineitems}
\phantomsection\label{\detokenize{CalSciPy.event_processing:CalSciPy.event_processing.get_num_events}}
\pysigstartsignatures
\pysiglinewithargsret{\sphinxcode{\sphinxupquote{CalSciPy.event\_processing.}}\sphinxbfcode{\sphinxupquote{get\_num\_events}}}{\emph{\DUrole{n}{event\_indices}}}{}
\pysigstopsignatures
\sphinxAtStartPar
Determines the number of events for each neuron in the event indices
\begin{quote}\begin{description}
\sphinxlineitem{Parameters}
\sphinxAtStartPar
\sphinxstyleliteralstrong{\sphinxupquote{event\_indices}} (\sphinxstyleliteralemphasis{\sphinxupquote{Iterable}}\sphinxstyleliteralemphasis{\sphinxupquote{{[}}}\sphinxstyleliteralemphasis{\sphinxupquote{Iterable}}\sphinxstyleliteralemphasis{\sphinxupquote{{[}}}\sphinxhref{https://docs.python.org/3/library/functions.html\#int}{\sphinxstyleliteralemphasis{\sphinxupquote{int}}}\sphinxstyleliteralemphasis{\sphinxupquote{{]}}}\sphinxstyleliteralemphasis{\sphinxupquote{{]}}}) \textendash{} An iterable of length M neurons containing a sequence with a duration for each event

\sphinxlineitem{Returns}
\sphinxAtStartPar
A 1\sphinxhyphen{}D vector of length M neurons containing the number of events for each neuron

\sphinxlineitem{Return type}
\sphinxAtStartPar
\sphinxhref{https://numpy.org/doc/1.24/reference/generated/numpy.ndarray.html\#numpy.ndarray}{numpy.ndarray}

\end{description}\end{quote}

\end{fulllineitems}

\index{identify\_events() (in module CalSciPy.event\_processing)@\spxentry{identify\_events()}\spxextra{in module CalSciPy.event\_processing}}

\begin{fulllineitems}
\phantomsection\label{\detokenize{CalSciPy.event_processing:CalSciPy.event_processing.identify_events}}
\pysigstartsignatures
\pysiglinewithargsret{\sphinxcode{\sphinxupquote{CalSciPy.event\_processing.}}\sphinxbfcode{\sphinxupquote{identify\_events}}}{\emph{\DUrole{n}{traces}}, \emph{\DUrole{n}{timeout}\DUrole{o}{=}\DUrole{default_value}{15}}, \emph{\DUrole{n}{frame\_rate}\DUrole{o}{=}\DUrole{default_value}{30.0}}, \emph{\DUrole{n}{smooth}\DUrole{o}{=}\DUrole{default_value}{True}}, \emph{\DUrole{n}{force\_nonneg}\DUrole{o}{=}\DUrole{default_value}{True}}}{}
\pysigstopsignatures
\sphinxAtStartPar
Identify event onset for each neuron using the smoothed, non\sphinxhyphen{}negative first\sphinxhyphen{}time derivative. The threshold for noise
is considered 1/2th the standard deviation of the derivative.
\begin{quote}\begin{description}
\sphinxlineitem{Parameters}\begin{itemize}
\item {} 
\sphinxAtStartPar
\sphinxstyleliteralstrong{\sphinxupquote{traces}} (\sphinxhref{https://numpy.org/doc/1.24/reference/generated/numpy.ndarray.html\#numpy.ndarray}{\sphinxstyleliteralemphasis{\sphinxupquote{numpy.ndarray}}}) \textendash{} An M neuron by N sample matrix

\item {} 
\sphinxAtStartPar
\sphinxstyleliteralstrong{\sphinxupquote{timeout}} (\sphinxhref{https://docs.python.org/3/library/functions.html\#int}{\sphinxstyleliteralemphasis{\sphinxupquote{int}}}) \textendash{} timeout distance for peak finding (frames)

\item {} 
\sphinxAtStartPar
\sphinxstyleliteralstrong{\sphinxupquote{frame\_rate}} (\sphinxhref{https://docs.python.org/3/library/functions.html\#float}{\sphinxstyleliteralemphasis{\sphinxupquote{float}}}) \textendash{} frame rate / time step for trace matrix

\item {} 
\sphinxAtStartPar
\sphinxstyleliteralstrong{\sphinxupquote{smooth}} (\sphinxstyleliteralemphasis{\sphinxupquote{bool = True}}) \textendash{} boolean indicating whether to smooth first\sphinxhyphen{}time derivative

\item {} 
\sphinxAtStartPar
\sphinxstyleliteralstrong{\sphinxupquote{force\_nonneg}} (\sphinxstyleliteralemphasis{\sphinxupquote{bool = True}}) \textendash{} boolean indicating whether to enforce non\sphinxhyphen{}negativity constraint on first\sphinxhyphen{}time derivative

\end{itemize}

\sphinxlineitem{Returns}
\sphinxAtStartPar
An iterable where each element contains a sequence of frames identified as event onsets

\sphinxlineitem{Return type}
\sphinxAtStartPar
Tuple{[}List{[}\sphinxhref{https://docs.python.org/3/library/functions.html\#int}{int}{]}{]}

\end{description}\end{quote}

\end{fulllineitems}

\index{normalize\_firing\_rates() (in module CalSciPy.event\_processing)@\spxentry{normalize\_firing\_rates()}\spxextra{in module CalSciPy.event\_processing}}

\begin{fulllineitems}
\phantomsection\label{\detokenize{CalSciPy.event_processing:CalSciPy.event_processing.normalize_firing_rates}}
\pysigstartsignatures
\pysiglinewithargsret{\sphinxcode{\sphinxupquote{CalSciPy.event\_processing.}}\sphinxbfcode{\sphinxupquote{normalize\_firing\_rates}}}{\emph{\DUrole{n}{firing\_matrix}}, \emph{\DUrole{n}{in\_place}\DUrole{o}{=}\DUrole{default_value}{False}}}{}
\pysigstopsignatures
\sphinxAtStartPar
Normalize firing rates by scaling to a max of 1.0. Non\sphinxhyphen{}negativity constrained.
\begin{quote}\begin{description}
\sphinxlineitem{Parameters}\begin{itemize}
\item {} 
\sphinxAtStartPar
\sphinxstyleliteralstrong{\sphinxupquote{firing\_matrix}} (\sphinxhref{https://numpy.org/doc/1.24/reference/generated/numpy.ndarray.html\#numpy.ndarray}{\sphinxstyleliteralemphasis{\sphinxupquote{numpy.ndarray}}}) \textendash{} matrix of n neuron x m samples where each element is either a spike or an
instantaneous firing rate

\item {} 
\sphinxAtStartPar
\sphinxstyleliteralstrong{\sphinxupquote{in\_place}} (\sphinxstyleliteralemphasis{\sphinxupquote{bool = False}}) \textendash{} boolean indicating whether to perform calculation in\sphinxhyphen{}place

\end{itemize}

\sphinxlineitem{Returns}
\sphinxAtStartPar
normalized firing rate matrix of n neurons x m samples

\sphinxlineitem{Return type}
\sphinxAtStartPar
\sphinxhref{https://numpy.org/doc/1.24/reference/generated/numpy.ndarray.html\#numpy.ndarray}{numpy.ndarray}

\end{description}\end{quote}

\end{fulllineitems}

\index{scale\_waveforms() (in module CalSciPy.event\_processing)@\spxentry{scale\_waveforms()}\spxextra{in module CalSciPy.event\_processing}}

\begin{fulllineitems}
\phantomsection\label{\detokenize{CalSciPy.event_processing:CalSciPy.event_processing.scale_waveforms}}
\pysigstartsignatures
\pysiglinewithargsret{\sphinxcode{\sphinxupquote{CalSciPy.event\_processing.}}\sphinxbfcode{\sphinxupquote{scale\_waveforms}}}{\emph{\DUrole{n}{waveforms}}, \emph{\DUrole{n}{scaler=\textless{}class \textquotesingle{}sklearn.preprocessing.\_data.StandardScaler\textquotesingle{}\textgreater{}}}}{}
\pysigstopsignatures
\sphinxAtStartPar
Scale waveforms for cross\sphinxhyphen{}neuron comparisons
\begin{quote}\begin{description}
\sphinxlineitem{Parameters}\begin{itemize}
\item {} 
\sphinxAtStartPar
\sphinxstyleliteralstrong{\sphinxupquote{waveforms}} (\sphinxhref{https://numpy.org/doc/1.24/reference/generated/numpy.ndarray.html\#numpy.ndarray}{\sphinxstyleliteralemphasis{\sphinxupquote{numpy.ndarray}}}) \textendash{} An Iterable of M events by N samples matrices of waveforms

\item {} 
\sphinxAtStartPar
\sphinxstyleliteralstrong{\sphinxupquote{scaler}} (\sphinxstyleliteralemphasis{\sphinxupquote{Callable}}) \textendash{} sklearn preprocessing object

\end{itemize}

\sphinxlineitem{Returns}
\sphinxAtStartPar
An Iterable of M event by N samples scaled matrices of waveforms

\sphinxlineitem{Return type}
\sphinxAtStartPar
Iterable{[}\sphinxhref{https://numpy.org/doc/1.24/reference/generated/numpy.ndarray.html\#numpy.ndarray}{numpy.ndarray}{]}

\end{description}\end{quote}

\end{fulllineitems}



\section{Input/Output (I/O)}
\label{\detokenize{Sub-Packages:input-output-i-o}}\label{\detokenize{Sub-Packages:io-module}}
\begin{DUlineblock}{0em}
\item[] Write me
\item[] Write me
\item[] Write me
\item[] Write me
\end{DUlineblock}

\sphinxstepscope


\subsection{CalSciPy.io\_tools module}
\label{\detokenize{CalSciPy.io_tools:module-CalSciPy.io_tools}}\label{\detokenize{CalSciPy.io_tools:calscipy-io-tools-module}}\label{\detokenize{CalSciPy.io_tools::doc}}\index{module@\spxentry{module}!CalSciPy.io\_tools@\spxentry{CalSciPy.io\_tools}}\index{CalSciPy.io\_tools@\spxentry{CalSciPy.io\_tools}!module@\spxentry{module}}\index{load\_binary() (in module CalSciPy.io\_tools)@\spxentry{load\_binary()}\spxextra{in module CalSciPy.io\_tools}}

\begin{fulllineitems}
\phantomsection\label{\detokenize{CalSciPy.io_tools:CalSciPy.io_tools.load_binary}}
\pysigstartsignatures
\pysiglinewithargsret{\sphinxcode{\sphinxupquote{CalSciPy.io\_tools.}}\sphinxbfcode{\sphinxupquote{load\_binary}}}{\emph{\DUrole{n}{path}}, \emph{\DUrole{n}{mapped}\DUrole{o}{=}\DUrole{default_value}{False}}}{}
\pysigstopsignatures\begin{quote}\begin{description}
\sphinxlineitem{Return type}
\sphinxAtStartPar
\sphinxhref{https://docs.python.org/3/library/typing.html\#typing.Union}{\sphinxcode{\sphinxupquote{Union}}}{[}\sphinxhref{https://numpy.org/doc/1.24/reference/generated/numpy.ndarray.html\#numpy.ndarray}{\sphinxcode{\sphinxupquote{ndarray}}}, \sphinxhref{https://numpy.org/doc/1.24/reference/generated/numpy.memmap.html\#numpy.memmap}{\sphinxcode{\sphinxupquote{memmap}}}{]}

\end{description}\end{quote}

\end{fulllineitems}

\index{load\_images() (in module CalSciPy.io\_tools)@\spxentry{load\_images()}\spxextra{in module CalSciPy.io\_tools}}

\begin{fulllineitems}
\phantomsection\label{\detokenize{CalSciPy.io_tools:CalSciPy.io_tools.load_images}}
\pysigstartsignatures
\pysiglinewithargsret{\sphinxcode{\sphinxupquote{CalSciPy.io\_tools.}}\sphinxbfcode{\sphinxupquote{load\_images}}}{\emph{\DUrole{n}{path}}}{}
\pysigstopsignatures
\sphinxAtStartPar
Load images into a numpy array. If path is a folder, all .tif files found non\sphinxhyphen{}recursively in the directory will be
compiled to a single array
\begin{quote}\begin{description}
\sphinxlineitem{Parameters}
\sphinxAtStartPar
\sphinxstyleliteralstrong{\sphinxupquote{path}} (\sphinxhref{https://docs.python.org/3/library/stdtypes.html\#str}{\sphinxstyleliteralemphasis{\sphinxupquote{str}}}\sphinxstyleliteralemphasis{\sphinxupquote{ or }}\sphinxhref{https://docs.python.org/3/library/pathlib.html\#pathlib.Path}{\sphinxstyleliteralemphasis{\sphinxupquote{pathlib.Path}}}) \textendash{} a file containing images or a folder containing several imaging stacks

\sphinxlineitem{Returns}
\sphinxAtStartPar
numpy array (frames, y\sphinxhyphen{}pixels, x\sphinxhyphen{}pixels)

\sphinxlineitem{Return type}
\sphinxAtStartPar
\sphinxhref{https://numpy.org/doc/1.24/reference/generated/numpy.ndarray.html\#numpy.ndarray}{numpy.ndarray}

\end{description}\end{quote}

\end{fulllineitems}

\index{save\_binary() (in module CalSciPy.io\_tools)@\spxentry{save\_binary()}\spxextra{in module CalSciPy.io\_tools}}

\begin{fulllineitems}
\phantomsection\label{\detokenize{CalSciPy.io_tools:CalSciPy.io_tools.save_binary}}
\pysigstartsignatures
\pysiglinewithargsret{\sphinxcode{\sphinxupquote{CalSciPy.io\_tools.}}\sphinxbfcode{\sphinxupquote{save\_binary}}}{\emph{\DUrole{n}{path}}, \emph{\DUrole{n}{images}}}{}
\pysigstopsignatures
\sphinxAtStartPar
Save images to language\sphinxhyphen{}agnostic binary format. Ideal for optimal read/write speeds and highly\sphinxhyphen{}robust to corruption.
However, the downside is that the images and their metadata are split into two separate files. Images are saved with
the \sphinxstyleemphasis{.bin} extension, while metadata is saved with extension \sphinxstyleemphasis{.json}. If for some reason you lose the metadata, you
can still load the binary if you know three of the following: number of frames, y\sphinxhyphen{}pixels, x\sphinxhyphen{}pixels, and the
datatype. The datatype is almost always unsigned 16\sphinxhyphen{}bit for all modern imaging systems\textendash{}even if they are collected
at 12 or 13\sphinxhyphen{}bit.
\begin{quote}\begin{description}
\sphinxlineitem{Parameters}
\sphinxAtStartPar
\sphinxstyleliteralstrong{\sphinxupquote{path}} (\sphinxhref{https://docs.python.org/3/library/typing.html\#typing.Union}{\sphinxcode{\sphinxupquote{Union}}}{[}\sphinxhref{https://docs.python.org/3/library/stdtypes.html\#str}{\sphinxcode{\sphinxupquote{str}}}, \sphinxhref{https://docs.python.org/3/library/pathlib.html\#pathlib.Path}{\sphinxcode{\sphinxupquote{Path}}}{]}) \textendash{} path to save images to. The path stem is considered the filename if it doesn’t have any extension.

\end{description}\end{quote}

\sphinxAtStartPar
If no filename is provided then the default filename is \sphinxstyleemphasis{binary\_video}.
:type path: str or pathlib.Path
:type images: \sphinxhref{https://numpy.org/doc/1.24/reference/generated/numpy.ndarray.html\#numpy.ndarray}{\sphinxcode{\sphinxupquote{ndarray}}}
:param images: images to save (frames, y\sphinxhyphen{}pixels, x\sphinxhyphen{}pixels)
:type images: numpy.ndarray
:return: 0 if successful
:rtype: int

\end{fulllineitems}

\index{save\_images() (in module CalSciPy.io\_tools)@\spxentry{save\_images()}\spxextra{in module CalSciPy.io\_tools}}

\begin{fulllineitems}
\phantomsection\label{\detokenize{CalSciPy.io_tools:CalSciPy.io_tools.save_images}}
\pysigstartsignatures
\pysiglinewithargsret{\sphinxcode{\sphinxupquote{CalSciPy.io\_tools.}}\sphinxbfcode{\sphinxupquote{save\_images}}}{\emph{\DUrole{n}{path}}, \emph{\DUrole{n}{images}}, \emph{\DUrole{n}{size\_cap}\DUrole{o}{=}\DUrole{default_value}{3.9}}}{}
\pysigstopsignatures
\sphinxAtStartPar
Save a numpy array to a single .tif file. If size \textgreater{} 4GB then saved as a series of files. If path is not a file and
already exists the default filename will be \sphinxstyleemphasis{images}.
\begin{quote}\begin{description}
\sphinxlineitem{Parameters}\begin{itemize}
\item {} 
\sphinxAtStartPar
\sphinxstyleliteralstrong{\sphinxupquote{path}} (\sphinxhref{https://docs.python.org/3/library/stdtypes.html\#str}{\sphinxstyleliteralemphasis{\sphinxupquote{str}}}\sphinxstyleliteralemphasis{\sphinxupquote{ or }}\sphinxhref{https://docs.python.org/3/library/pathlib.html\#pathlib.Path}{\sphinxstyleliteralemphasis{\sphinxupquote{pathlib.Path}}}) \textendash{} filename or absolute path

\item {} 
\sphinxAtStartPar
\sphinxstyleliteralstrong{\sphinxupquote{images}} (\sphinxhref{https://numpy.org/doc/1.24/reference/generated/numpy.ndarray.html\#numpy.ndarray}{\sphinxstyleliteralemphasis{\sphinxupquote{numpy.ndarray}}}) \textendash{} numpy array (frames, y pixels, x pixels)

\item {} 
\sphinxAtStartPar
\sphinxstyleliteralstrong{\sphinxupquote{size\_cap}} (\sphinxstyleliteralemphasis{\sphinxupquote{float = 3.9}}) \textendash{} maximum size per file

\end{itemize}

\sphinxlineitem{Returns}
\sphinxAtStartPar
returns 0 if successful

\sphinxlineitem{Return type}
\sphinxAtStartPar
\sphinxhref{https://docs.python.org/3/library/functions.html\#int}{int}

\end{description}\end{quote}

\end{fulllineitems}



\section{Image Processing}
\label{\detokenize{Sub-Packages:image-processing}}\label{\detokenize{Sub-Packages:image-processing-module}}
\begin{DUlineblock}{0em}
\item[] Write me
\item[] Write me
\item[] Write me
\item[] Write me
\end{DUlineblock}

\sphinxstepscope


\subsection{CalSciPy.image\_processing module}
\label{\detokenize{CalSciPy.image_processing:module-CalSciPy.image_processing}}\label{\detokenize{CalSciPy.image_processing:calscipy-image-processing-module}}\label{\detokenize{CalSciPy.image_processing::doc}}\index{module@\spxentry{module}!CalSciPy.image\_processing@\spxentry{CalSciPy.image\_processing}}\index{CalSciPy.image\_processing@\spxentry{CalSciPy.image\_processing}!module@\spxentry{module}}\index{gaussian\_filter() (in module CalSciPy.image\_processing)@\spxentry{gaussian\_filter()}\spxextra{in module CalSciPy.image\_processing}}

\begin{fulllineitems}
\phantomsection\label{\detokenize{CalSciPy.image_processing:CalSciPy.image_processing.gaussian_filter}}
\pysigstartsignatures
\pysiglinewithargsret{\sphinxcode{\sphinxupquote{CalSciPy.image\_processing.}}\sphinxbfcode{\sphinxupquote{gaussian\_filter}}}{\emph{\DUrole{n}{images}}, \emph{\DUrole{n}{sigma}\DUrole{o}{=}\DUrole{default_value}{1.0}}, \emph{\DUrole{n}{block\_size}\DUrole{o}{=}\DUrole{default_value}{None}}, \emph{\DUrole{n}{block\_buffer}\DUrole{o}{=}\DUrole{default_value}{0}}, \emph{\DUrole{n}{in\_place}\DUrole{o}{=}\DUrole{default_value}{False}}}{}
\pysigstopsignatures
\sphinxAtStartPar
GPU\sphinxhyphen{}parallelized multidimensional gaussian filter. Optional arguments for in\sphinxhyphen{}place calculation. Can be calculated
blockwise with overlapping or non\sphinxhyphen{}overlapping blocks.

\sphinxAtStartPar
Designed for use on arrays larger than the available memory capacity.

\sphinxAtStartPar
Footprint is of the form np.ones((frames, y pixels, x pixels)) with the origin in the center
\begin{quote}\begin{description}
\sphinxlineitem{Parameters}\begin{itemize}
\item {} 
\sphinxAtStartPar
\sphinxstyleliteralstrong{\sphinxupquote{images}} (\sphinxhref{https://numpy.org/doc/1.24/reference/generated/numpy.ndarray.html\#numpy.ndarray}{\sphinxstyleliteralemphasis{\sphinxupquote{numpy.ndarray}}}) \textendash{} images stack to be filtered

\item {} 
\sphinxAtStartPar
\sphinxstyleliteralstrong{\sphinxupquote{sigma}} (\sphinxstyleliteralemphasis{\sphinxupquote{Number}}\sphinxstyleliteralemphasis{\sphinxupquote{ or }}\sphinxhref{https://numpy.org/doc/1.24/reference/generated/numpy.ndarray.html\#numpy.ndarray}{\sphinxstyleliteralemphasis{\sphinxupquote{numpy.ndarray}}}) \textendash{} sigma for gaussian filter

\item {} 
\sphinxAtStartPar
\sphinxstyleliteralstrong{\sphinxupquote{block\_size}} (\sphinxstyleliteralemphasis{\sphinxupquote{int = None}}) \textendash{} the size of each block. Must fit within memory

\item {} 
\sphinxAtStartPar
\sphinxstyleliteralstrong{\sphinxupquote{block\_buffer}} (\sphinxstyleliteralemphasis{\sphinxupquote{int = 0}}) \textendash{} the size of the overlapping region between block

\item {} 
\sphinxAtStartPar
\sphinxstyleliteralstrong{\sphinxupquote{in\_place}} (\sphinxstyleliteralemphasis{\sphinxupquote{bool = False}}) \textendash{} whether to calculate in\sphinxhyphen{}place

\end{itemize}

\sphinxlineitem{Returns}
\sphinxAtStartPar
images: numpy array (frames, y pixels, x pixels)

\sphinxlineitem{Return type}
\sphinxAtStartPar
\sphinxhref{https://numpy.org/doc/1.24/reference/generated/numpy.ndarray.html\#numpy.ndarray}{numpy.ndarray}

\end{description}\end{quote}

\end{fulllineitems}

\index{median\_filter() (in module CalSciPy.image\_processing)@\spxentry{median\_filter()}\spxextra{in module CalSciPy.image\_processing}}

\begin{fulllineitems}
\phantomsection\label{\detokenize{CalSciPy.image_processing:CalSciPy.image_processing.median_filter}}
\pysigstartsignatures
\pysiglinewithargsret{\sphinxcode{\sphinxupquote{CalSciPy.image\_processing.}}\sphinxbfcode{\sphinxupquote{median\_filter}}}{\emph{\DUrole{n}{images}}, \emph{\DUrole{n}{mask}\DUrole{o}{=}\DUrole{default_value}{array({[}{[}{[}1., 1., 1.{]}, {[}1., 1., 1.{]}, {[}1., 1., 1.{]}{]}, {[}{[}1., 1., 1.{]}, {[}1., 1., 1.{]}, {[}1., 1., 1.{]}{]}, {[}{[}1., 1., 1.{]}, {[}1., 1., 1.{]}, {[}1., 1., 1.{]}{]}{]})}}, \emph{\DUrole{n}{block\_size}\DUrole{o}{=}\DUrole{default_value}{None}}, \emph{\DUrole{n}{block\_buffer}\DUrole{o}{=}\DUrole{default_value}{0}}, \emph{\DUrole{n}{in\_place}\DUrole{o}{=}\DUrole{default_value}{False}}}{}
\pysigstopsignatures
\sphinxAtStartPar
GPU\sphinxhyphen{}parallelized multidimensional median filter. Optional arguments for in\sphinxhyphen{}place calculation. Can be calculated
blockwise with overlapping or non\sphinxhyphen{}overlapping blocks.

\sphinxAtStartPar
Designed for use on arrays larger than the available memory capacity.

\sphinxAtStartPar
Footprint is of the form np.ones((frames, y pixels, x pixels)) with the origin in the center
\begin{quote}\begin{description}
\sphinxlineitem{Parameters}\begin{itemize}
\item {} 
\sphinxAtStartPar
\sphinxstyleliteralstrong{\sphinxupquote{images}} (\sphinxhref{https://numpy.org/doc/1.24/reference/generated/numpy.ndarray.html\#numpy.ndarray}{\sphinxstyleliteralemphasis{\sphinxupquote{numpy.ndarray}}}) \textendash{} images stack to be filtered

\item {} 
\sphinxAtStartPar
\sphinxstyleliteralstrong{\sphinxupquote{mask}} (\sphinxstyleliteralemphasis{\sphinxupquote{numpy.ndarray = np.ones}}\sphinxstyleliteralemphasis{\sphinxupquote{(}}\sphinxstyleliteralemphasis{\sphinxupquote{(}}\sphinxstyleliteralemphasis{\sphinxupquote{3}}\sphinxstyleliteralemphasis{\sphinxupquote{, }}\sphinxstyleliteralemphasis{\sphinxupquote{3}}\sphinxstyleliteralemphasis{\sphinxupquote{, }}\sphinxstyleliteralemphasis{\sphinxupquote{3}}\sphinxstyleliteralemphasis{\sphinxupquote{)}}\sphinxstyleliteralemphasis{\sphinxupquote{)}}) \textendash{} mask of the median filter

\item {} 
\sphinxAtStartPar
\sphinxstyleliteralstrong{\sphinxupquote{block\_size}} (\sphinxstyleliteralemphasis{\sphinxupquote{int = None}}) \textendash{} the size of each block. Must fit within memory

\item {} 
\sphinxAtStartPar
\sphinxstyleliteralstrong{\sphinxupquote{block\_buffer}} (\sphinxstyleliteralemphasis{\sphinxupquote{int = 0}}) \textendash{} the size of the overlapping region between block

\item {} 
\sphinxAtStartPar
\sphinxstyleliteralstrong{\sphinxupquote{in\_place}} (\sphinxstyleliteralemphasis{\sphinxupquote{bool = False}}) \textendash{} whether to calculate in\sphinxhyphen{}place

\end{itemize}

\sphinxlineitem{Returns}
\sphinxAtStartPar
images: numpy array (frames, y pixels, x pixels)

\sphinxlineitem{Return type}
\sphinxAtStartPar
\sphinxhref{https://numpy.org/doc/1.24/reference/generated/numpy.ndarray.html\#numpy.ndarray}{numpy.ndarray}

\end{description}\end{quote}

\end{fulllineitems}



\section{Interactive Visuals}
\label{\detokenize{Sub-Packages:interactive-visuals}}\label{\detokenize{Sub-Packages:interactive-visuals-module}}
\begin{DUlineblock}{0em}
\item[] Write me
\item[] Write me
\item[] Write me
\item[] Write me
\end{DUlineblock}


\subsection{Interactive Visuals Methods}
\label{\detokenize{Sub-Packages:interactive-visuals-methods}}
\begin{DUlineblock}{0em}
\item[] Import me
\end{DUlineblock}


\subsubsection{Miscellaneous}
\label{\detokenize{Sub-Packages:miscellaneous}}\label{\detokenize{Sub-Packages:miscellaneous-module}}
\begin{DUlineblock}{0em}
\item[] Write me
\item[] Write me
\item[] Write me
\item[] Write me
\end{DUlineblock}


\subsection{Miscellaneous Methods}
\label{\detokenize{Sub-Packages:miscellaneous-methods}}
\begin{DUlineblock}{0em}
\item[] Import me
\end{DUlineblock}


\section{Reorganization}
\label{\detokenize{Sub-Packages:reorganization}}\label{\detokenize{Sub-Packages:reorganization-module}}
\begin{DUlineblock}{0em}
\item[] Write me
\item[] Write me
\item[] Write me
\item[] Write me
\end{DUlineblock}


\subsection{Reorganization Methods}
\label{\detokenize{Sub-Packages:reorganization-methods}}
\sphinxstepscope


\subsubsection{CalSciPy.reorganization module}
\label{\detokenize{CalSciPy.reorganization:module-CalSciPy.reorganization}}\label{\detokenize{CalSciPy.reorganization:calscipy-reorganization-module}}\label{\detokenize{CalSciPy.reorganization::doc}}\index{module@\spxentry{module}!CalSciPy.reorganization@\spxentry{CalSciPy.reorganization}}\index{CalSciPy.reorganization@\spxentry{CalSciPy.reorganization}!module@\spxentry{module}}\index{generate\_raster() (in module CalSciPy.reorganization)@\spxentry{generate\_raster()}\spxextra{in module CalSciPy.reorganization}}

\begin{fulllineitems}
\phantomsection\label{\detokenize{CalSciPy.reorganization:CalSciPy.reorganization.generate_raster}}
\pysigstartsignatures
\pysiglinewithargsret{\sphinxcode{\sphinxupquote{CalSciPy.reorganization.}}\sphinxbfcode{\sphinxupquote{generate\_raster}}}{\emph{\DUrole{n}{event\_frames}}, \emph{\DUrole{n}{total\_frames}\DUrole{o}{=}\DUrole{default_value}{None}}}{}
\pysigstopsignatures
\sphinxAtStartPar
Generate raster from an iterable of iterables containing the spike or event times for each neuron
\begin{quote}\begin{description}
\sphinxlineitem{Parameters}\begin{itemize}
\item {} 
\sphinxAtStartPar
\sphinxstyleliteralstrong{\sphinxupquote{event\_frames}} (\sphinxstyleliteralemphasis{\sphinxupquote{Iterable}}\sphinxstyleliteralemphasis{\sphinxupquote{{[}}}\sphinxstyleliteralemphasis{\sphinxupquote{Iterable}}\sphinxstyleliteralemphasis{\sphinxupquote{{[}}}\sphinxhref{https://docs.python.org/3/library/functions.html\#int}{\sphinxstyleliteralemphasis{\sphinxupquote{int}}}\sphinxstyleliteralemphasis{\sphinxupquote{{]}}}\sphinxstyleliteralemphasis{\sphinxupquote{{]}}}) \textendash{} iterable containing an iterable identifying the event frames for each neuron

\item {} 
\sphinxAtStartPar
\sphinxstyleliteralstrong{\sphinxupquote{total\_frames}} (\sphinxstyleliteralemphasis{\sphinxupquote{Optional}}\sphinxstyleliteralemphasis{\sphinxupquote{{[}}}\sphinxhref{https://docs.python.org/3/library/functions.html\#int}{\sphinxstyleliteralemphasis{\sphinxupquote{int}}}\sphinxstyleliteralemphasis{\sphinxupquote{{]} }}\sphinxstyleliteralemphasis{\sphinxupquote{= None}}) \textendash{} total number of frames

\end{itemize}

\sphinxlineitem{Returns}
\sphinxAtStartPar
event matrix of neurons x total frames

\sphinxlineitem{Return type}
\sphinxAtStartPar
\sphinxhref{https://numpy.org/doc/1.24/reference/generated/numpy.ndarray.html\#numpy.ndarray}{numpy.ndarray}

\end{description}\end{quote}

\end{fulllineitems}

\index{generate\_tensor() (in module CalSciPy.reorganization)@\spxentry{generate\_tensor()}\spxextra{in module CalSciPy.reorganization}}

\begin{fulllineitems}
\phantomsection\label{\detokenize{CalSciPy.reorganization:CalSciPy.reorganization.generate_tensor}}
\pysigstartsignatures
\pysiglinewithargsret{\sphinxcode{\sphinxupquote{CalSciPy.reorganization.}}\sphinxbfcode{\sphinxupquote{generate\_tensor}}}{\emph{\DUrole{n}{traces\_as\_matrix}}, \emph{\DUrole{n}{chunk\_size}}}{}
\pysigstopsignatures
\sphinxAtStartPar
Generates a tensor given chunk / trial indices
\begin{quote}\begin{description}
\sphinxlineitem{Parameters}\begin{itemize}
\item {} 
\sphinxAtStartPar
\sphinxstyleliteralstrong{\sphinxupquote{traces\_as\_matrix}} (\sphinxhref{https://numpy.org/doc/1.24/reference/generated/numpy.ndarray.html\#numpy.ndarray}{\sphinxstyleliteralemphasis{\sphinxupquote{numpy.ndarray}}}) \textendash{} traces in matrix form (neurons x frames)

\item {} 
\sphinxAtStartPar
\sphinxstyleliteralstrong{\sphinxupquote{chunk\_size}} (\sphinxhref{https://docs.python.org/3/library/functions.html\#int}{\sphinxstyleliteralemphasis{\sphinxupquote{int}}}) \textendash{} size of each chunk

\end{itemize}

\sphinxlineitem{Returns}
\sphinxAtStartPar
traces\_as\_tensor

\sphinxlineitem{Return type}
\sphinxAtStartPar
\sphinxhref{https://numpy.org/doc/1.24/reference/generated/numpy.ndarray.html\#numpy.ndarray}{numpy.ndarray}

\end{description}\end{quote}

\end{fulllineitems}

\index{merge\_factorized\_matrices() (in module CalSciPy.reorganization)@\spxentry{merge\_factorized\_matrices()}\spxextra{in module CalSciPy.reorganization}}

\begin{fulllineitems}
\phantomsection\label{\detokenize{CalSciPy.reorganization:CalSciPy.reorganization.merge_factorized_matrices}}
\pysigstartsignatures
\pysiglinewithargsret{\sphinxcode{\sphinxupquote{CalSciPy.reorganization.}}\sphinxbfcode{\sphinxupquote{merge\_factorized\_matrices}}}{\emph{\DUrole{n}{factorized\_traces}}, \emph{\DUrole{n}{component}\DUrole{o}{=}\DUrole{default_value}{0}}}{}
\pysigstopsignatures
\sphinxAtStartPar
Concatenate a neuron x chunk or trial array in which each element is a component x frame factorization of the
original trace:
\begin{quote}\begin{description}
\sphinxlineitem{Parameters}\begin{itemize}
\item {} 
\sphinxAtStartPar
\sphinxstyleliteralstrong{\sphinxupquote{factorized\_traces}} (\sphinxhref{https://numpy.org/doc/1.24/reference/generated/numpy.ndarray.html\#numpy.ndarray}{\sphinxstyleliteralemphasis{\sphinxupquote{numpy.ndarray}}}) \textendash{} neurons x chunks (trial, tiff, etc) containing the neuron’s trace factorized
into several components

\item {} 
\sphinxAtStartPar
\sphinxstyleliteralstrong{\sphinxupquote{component}} (\sphinxhref{https://docs.python.org/3/library/functions.html\#int}{\sphinxstyleliteralemphasis{\sphinxupquote{int}}}) \textendash{} specific component to extract

\end{itemize}

\sphinxlineitem{Returns}
\sphinxAtStartPar
traces of specific component in matrix form

\sphinxlineitem{Return type}
\sphinxAtStartPar
\sphinxhref{https://numpy.org/doc/1.24/reference/generated/numpy.ndarray.html\#numpy.ndarray}{numpy.ndarray}

\end{description}\end{quote}

\end{fulllineitems}

\index{merge\_tensor() (in module CalSciPy.reorganization)@\spxentry{merge\_tensor()}\spxextra{in module CalSciPy.reorganization}}

\begin{fulllineitems}
\phantomsection\label{\detokenize{CalSciPy.reorganization:CalSciPy.reorganization.merge_tensor}}
\pysigstartsignatures
\pysiglinewithargsret{\sphinxcode{\sphinxupquote{CalSciPy.reorganization.}}\sphinxbfcode{\sphinxupquote{merge\_tensor}}}{\emph{\DUrole{n}{traces\_as\_tensor}}}{}
\pysigstopsignatures
\sphinxAtStartPar
Concatenate multiple trials or tiffs into single matrix:
\begin{quote}\begin{description}
\sphinxlineitem{Parameters}
\sphinxAtStartPar
\sphinxstyleliteralstrong{\sphinxupquote{traces\_as\_tensor}} (\sphinxhref{https://numpy.org/doc/1.24/reference/generated/numpy.ndarray.html\#numpy.ndarray}{\sphinxstyleliteralemphasis{\sphinxupquote{numpy.ndarray}}}) \textendash{} chunk (trial, tiff, etc) x neurons x frames

\sphinxlineitem{Returns}
\sphinxAtStartPar
traces in matrix form

\sphinxlineitem{Return type}
\sphinxAtStartPar
\sphinxhref{https://numpy.org/doc/1.24/reference/generated/numpy.ndarray.html\#numpy.ndarray}{numpy.ndarray}

\end{description}\end{quote}

\end{fulllineitems}



\section{Trace Processing}
\label{\detokenize{Sub-Packages:trace-processing}}\label{\detokenize{Sub-Packages:trace-processing-module}}
\begin{DUlineblock}{0em}
\item[] Write me
\item[] Write me
\item[] Write me
\item[] Write me
\end{DUlineblock}

\sphinxstepscope


\subsection{CalSciPy.trace\_processing module}
\label{\detokenize{CalSciPy.trace_processing:module-CalSciPy.trace_processing}}\label{\detokenize{CalSciPy.trace_processing:calscipy-trace-processing-module}}\label{\detokenize{CalSciPy.trace_processing::doc}}\index{module@\spxentry{module}!CalSciPy.trace\_processing@\spxentry{CalSciPy.trace\_processing}}\index{CalSciPy.trace\_processing@\spxentry{CalSciPy.trace\_processing}!module@\spxentry{module}}\index{calculate\_dfof() (in module CalSciPy.trace\_processing)@\spxentry{calculate\_dfof()}\spxextra{in module CalSciPy.trace\_processing}}

\begin{fulllineitems}
\phantomsection\label{\detokenize{CalSciPy.trace_processing:CalSciPy.trace_processing.calculate_dfof}}
\pysigstartsignatures
\pysiglinewithargsret{\sphinxcode{\sphinxupquote{CalSciPy.trace\_processing.}}\sphinxbfcode{\sphinxupquote{calculate\_dfof}}}{\emph{\DUrole{n}{traces}}, \emph{\DUrole{n}{frame\_rate}\DUrole{o}{=}\DUrole{default_value}{30.0}}, \emph{\DUrole{n}{in\_place}\DUrole{o}{=}\DUrole{default_value}{False}}, \emph{\DUrole{n}{offset}\DUrole{o}{=}\DUrole{default_value}{0.0}}, \emph{\DUrole{n}{external\_reference}\DUrole{o}{=}\DUrole{default_value}{None}}}{}
\pysigstopsignatures
\sphinxAtStartPar
Calculates Δf/f0 (fold fluorescence over baseline). Baseline is defined as the 5th percentile of the signal
after a 1Hz low\sphinxhyphen{}pass filter using a Hamming window. Baseline can be calculated using an external reference using the
raw argument or adjusted by using the offset argument. Supports in\sphinxhyphen{}place calculation (off by default).
\begin{quote}\begin{description}
\sphinxlineitem{Parameters}\begin{itemize}
\item {} 
\sphinxAtStartPar
\sphinxstyleliteralstrong{\sphinxupquote{traces}} (\sphinxhref{https://numpy.org/doc/1.24/reference/generated/numpy.ndarray.html\#numpy.ndarray}{\sphinxstyleliteralemphasis{\sphinxupquote{numpy.ndarray}}}) \textendash{} matrix of traces in the form of neurons x frames

\item {} 
\sphinxAtStartPar
\sphinxstyleliteralstrong{\sphinxupquote{frame\_rate}} (\sphinxstyleliteralemphasis{\sphinxupquote{float = 30.0}}) \textendash{} frame rate of dataset

\item {} 
\sphinxAtStartPar
\sphinxstyleliteralstrong{\sphinxupquote{in\_place}} (\sphinxstyleliteralemphasis{\sphinxupquote{bool = False}}) \textendash{} boolean indicating whether to perform calculation in\sphinxhyphen{}place

\item {} 
\sphinxAtStartPar
\sphinxstyleliteralstrong{\sphinxupquote{offset}} (\sphinxstyleliteralemphasis{\sphinxupquote{float = 0.0}}) \textendash{} offset added to baseline; useful if traces are non\sphinxhyphen{}negative

\item {} 
\sphinxAtStartPar
\sphinxstyleliteralstrong{\sphinxupquote{external\_reference}} (\sphinxstyleliteralemphasis{\sphinxupquote{numpy.ndarray = None}}) \textendash{} secondary dataset used to calculate baseline; useful if traces have been factorized

\end{itemize}

\sphinxlineitem{Returns}
\sphinxAtStartPar
Δf/f0 matrix of n neurons x m samples

\sphinxlineitem{Return type}
\sphinxAtStartPar
\sphinxhref{https://numpy.org/doc/1.24/reference/generated/numpy.ndarray.html\#numpy.ndarray}{numpy.ndarray}

\end{description}\end{quote}

\end{fulllineitems}

\index{calculate\_standardized\_noise() (in module CalSciPy.trace\_processing)@\spxentry{calculate\_standardized\_noise()}\spxextra{in module CalSciPy.trace\_processing}}

\begin{fulllineitems}
\phantomsection\label{\detokenize{CalSciPy.trace_processing:CalSciPy.trace_processing.calculate_standardized_noise}}
\pysigstartsignatures
\pysiglinewithargsret{\sphinxcode{\sphinxupquote{CalSciPy.trace\_processing.}}\sphinxbfcode{\sphinxupquote{calculate\_standardized\_noise}}}{\emph{\DUrole{n}{fold\_fluorescence\_over\_baseline}}, \emph{\DUrole{n}{frame\_rate}\DUrole{o}{=}\DUrole{default_value}{30.0}}}{}
\pysigstopsignatures\begin{description}
\sphinxlineitem{Calculates a frame\sphinxhyphen{}rate independent standardized noise as defined as:}
\begin{DUlineblock}{0em}
\item[] \(v = \frac{\sigma \frac{\Delta F}F}\sqrt{f}\)
\end{DUlineblock}

\end{description}

\sphinxAtStartPar
It is robust against outliers and approximates the standard deviation of Δf/f0 baseline fluctuations.
For comparison, the more exquisite of the Allen Brain Institute’s public datasets are approximately 1*\%Hz\textasciicircum{}(\sphinxhyphen{}1/2)
\begin{quote}\begin{description}
\sphinxlineitem{Parameters}\begin{itemize}
\item {} 
\sphinxAtStartPar
\sphinxstyleliteralstrong{\sphinxupquote{fold\_fluorescence\_over\_baseline}} (\sphinxhref{https://numpy.org/doc/1.24/reference/generated/numpy.ndarray.html\#numpy.ndarray}{\sphinxstyleliteralemphasis{\sphinxupquote{numpy.ndarray}}}) \textendash{} fold fluorescence over baseline (i.e., Δf/f0)

\item {} 
\sphinxAtStartPar
\sphinxstyleliteralstrong{\sphinxupquote{frame\_rate}} (\sphinxstyleliteralemphasis{\sphinxupquote{float = 30}}) \textendash{} frame rate of dataset

\end{itemize}

\sphinxlineitem{Returns}
\sphinxAtStartPar
standardized noise (units are  1*\%Hz\textasciicircum{}(\sphinxhyphen{}1/2) ) for each neuron

\sphinxlineitem{Return type}
\sphinxAtStartPar
\sphinxhref{https://numpy.org/doc/1.24/reference/generated/numpy.ndarray.html\#numpy.ndarray}{numpy.ndarray}

\end{description}\end{quote}

\end{fulllineitems}

\index{detrend\_polynomial() (in module CalSciPy.trace\_processing)@\spxentry{detrend\_polynomial()}\spxextra{in module CalSciPy.trace\_processing}}

\begin{fulllineitems}
\phantomsection\label{\detokenize{CalSciPy.trace_processing:CalSciPy.trace_processing.detrend_polynomial}}
\pysigstartsignatures
\pysiglinewithargsret{\sphinxcode{\sphinxupquote{CalSciPy.trace\_processing.}}\sphinxbfcode{\sphinxupquote{detrend\_polynomial}}}{\emph{\DUrole{n}{traces}}, \emph{\DUrole{n}{in\_place}\DUrole{o}{=}\DUrole{default_value}{False}}}{}
\pysigstopsignatures
\sphinxAtStartPar
Detrend traces using a fourth\sphinxhyphen{}order polynomial
\begin{quote}\begin{description}
\sphinxlineitem{Parameters}\begin{itemize}
\item {} 
\sphinxAtStartPar
\sphinxstyleliteralstrong{\sphinxupquote{traces}} (\sphinxhref{https://numpy.org/doc/1.24/reference/generated/numpy.ndarray.html\#numpy.ndarray}{\sphinxstyleliteralemphasis{\sphinxupquote{numpy.ndarray}}}) \textendash{} matrix of traces in the form of neurons x frames

\item {} 
\sphinxAtStartPar
\sphinxstyleliteralstrong{\sphinxupquote{in\_place}} (\sphinxstyleliteralemphasis{\sphinxupquote{bool = False}}) \textendash{} boolean indicating whether to perform calculation in\sphinxhyphen{}place

\end{itemize}

\sphinxlineitem{Returns}
\sphinxAtStartPar
detrended traces

\sphinxlineitem{Return type}
\sphinxAtStartPar
\sphinxhref{https://numpy.org/doc/1.24/reference/generated/numpy.ndarray.html\#numpy.ndarray}{numpy.ndarray}

\end{description}\end{quote}

\end{fulllineitems}

\index{perona\_malik\_diffusion() (in module CalSciPy.trace\_processing)@\spxentry{perona\_malik\_diffusion()}\spxextra{in module CalSciPy.trace\_processing}}

\begin{fulllineitems}
\phantomsection\label{\detokenize{CalSciPy.trace_processing:CalSciPy.trace_processing.perona_malik_diffusion}}
\pysigstartsignatures
\pysiglinewithargsret{\sphinxcode{\sphinxupquote{CalSciPy.trace\_processing.}}\sphinxbfcode{\sphinxupquote{perona\_malik\_diffusion}}}{\emph{\DUrole{n}{traces}}, \emph{\DUrole{n}{iters}\DUrole{o}{=}\DUrole{default_value}{25}}, \emph{\DUrole{n}{kappa}\DUrole{o}{=}\DUrole{default_value}{0.15}}, \emph{\DUrole{n}{gamma}\DUrole{o}{=}\DUrole{default_value}{0.25}}, \emph{\DUrole{n}{in\_place}\DUrole{o}{=}\DUrole{default_value}{False}}}{}
\pysigstopsignatures
\sphinxAtStartPar
Edge\sphinxhyphen{}preserving smoothing using perona malik diffusion. This is a non\sphinxhyphen{}linear smoothing technique that avoids the
temporal distortion introduced onto traces by standard gaussian smoothing.

\sphinxAtStartPar
The parameter \sphinxtitleref{kappa} controls the level of smoothing (“diffusion”) as a function of the derivative of the trace
(or “gradient” in the case of 2D images where this algorithm is often used). This function is known as the
diffusion coefficient. When the derivative for some portion of the trace is low, the algorithm will encourage
smoothing to reduce noise. If the derivative is large like during a burst of activity, the algorithm will discourage
smoothing to maintain its structure. Here, the argument \sphinxtitleref{kappa} is multiplied by the dynamic range to generate the
true kappa.

\sphinxAtStartPar
represents the percentile used to calculate the true
kappa

\sphinxAtStartPar
The diffusion coefficient implemented here is e\textasciicircum{}(\sphinxhyphen{}(derivative/kappa)\textasciicircum{}2).

\sphinxAtStartPar
Perona\sphinxhyphen{}Malik diffusion is an iterative process. The parameter \sphinxtitleref{gamma} controls the rate of diffusion, while
parameter \sphinxtitleref{iters} sets the number of iterations to perform.

\sphinxAtStartPar
This implementation is currently situated to handle 1\sphinxhyphen{}D vectors because it gives us some performance benefits.
\begin{quote}\begin{description}
\sphinxlineitem{Parameters}\begin{itemize}
\item {} 
\sphinxAtStartPar
\sphinxstyleliteralstrong{\sphinxupquote{traces}} (\sphinxhref{https://numpy.org/doc/1.24/reference/generated/numpy.ndarray.html\#numpy.ndarray}{\sphinxstyleliteralemphasis{\sphinxupquote{numpy.ndarray}}}) \textendash{} matrix of M neurons by N samples

\item {} 
\sphinxAtStartPar
\sphinxstyleliteralstrong{\sphinxupquote{iters}} (\sphinxstyleliteralemphasis{\sphinxupquote{int = 25}}) \textendash{} number of iterations

\item {} 
\sphinxAtStartPar
\sphinxstyleliteralstrong{\sphinxupquote{kappa}} (\sphinxstyleliteralemphasis{\sphinxupquote{Number = 15}}) \textendash{} used to calculate the true kappa, where true kappa = kappa * dynamic range. range 0\sphinxhyphen{}1

\item {} 
\sphinxAtStartPar
\sphinxstyleliteralstrong{\sphinxupquote{gamma}} (\sphinxstyleliteralemphasis{\sphinxupquote{float = 0.25}}) \textendash{} rate of diffusion for each iter. range 0\sphinxhyphen{}1

\item {} 
\sphinxAtStartPar
\sphinxstyleliteralstrong{\sphinxupquote{in\_place}} (\sphinxstyleliteralemphasis{\sphinxupquote{bool = False}}) \textendash{} whether to calculate in\sphinxhyphen{}place

\end{itemize}

\sphinxlineitem{Returns}
\sphinxAtStartPar
smoothed traces

\sphinxlineitem{Return type}
\sphinxAtStartPar
\sphinxhref{https://numpy.org/doc/1.24/reference/generated/numpy.ndarray.html\#numpy.ndarray}{numpy.ndarray}

\end{description}\end{quote}

\end{fulllineitems}

\phantomsection\label{\detokenize{Sub-Packages:version-module}}
\begin{DUlineblock}{0em}
\item[] Write me
\item[] Write me
\item[] Write me
\item[] Write me
\end{DUlineblock}


\subsection{Version Methods}
\label{\detokenize{Sub-Packages:version-methods}}
\begin{DUlineblock}{0em}
\item[] Import Me
\end{DUlineblock}


\chapter{Indices and tables}
\label{\detokenize{index:indices-and-tables}}\begin{itemize}
\item {} 
\sphinxAtStartPar
\DUrole{xref,std,std-ref}{genindex}

\item {} 
\sphinxAtStartPar
\DUrole{xref,std,std-ref}{modindex}

\item {} 
\sphinxAtStartPar
\DUrole{xref,std,std-ref}{search}

\end{itemize}


\renewcommand{\indexname}{Python Module Index}
\begin{sphinxtheindex}
\let\bigletter\sphinxstyleindexlettergroup
\bigletter{c}
\item\relax\sphinxstyleindexentry{CalSciPy.bruker}\sphinxstyleindexpageref{CalSciPy.bruker:\detokenize{module-CalSciPy.bruker}}
\item\relax\sphinxstyleindexentry{CalSciPy.event\_processing}\sphinxstyleindexpageref{CalSciPy.event_processing:\detokenize{module-CalSciPy.event_processing}}
\item\relax\sphinxstyleindexentry{CalSciPy.image\_processing}\sphinxstyleindexpageref{CalSciPy.image_processing:\detokenize{module-CalSciPy.image_processing}}
\item\relax\sphinxstyleindexentry{CalSciPy.io\_tools}\sphinxstyleindexpageref{CalSciPy.io_tools:\detokenize{module-CalSciPy.io_tools}}
\item\relax\sphinxstyleindexentry{CalSciPy.reorganization}\sphinxstyleindexpageref{CalSciPy.reorganization:\detokenize{module-CalSciPy.reorganization}}
\item\relax\sphinxstyleindexentry{CalSciPy.trace\_processing}\sphinxstyleindexpageref{CalSciPy.trace_processing:\detokenize{module-CalSciPy.trace_processing}}
\end{sphinxtheindex}

\renewcommand{\indexname}{Index}
\printindex
\end{document}