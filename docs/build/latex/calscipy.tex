%% Generated by Sphinx.
\def\sphinxdocclass{report}
\documentclass[letterpaper,10pt,english]{sphinxmanual}
\ifdefined\pdfpxdimen
   \let\sphinxpxdimen\pdfpxdimen\else\newdimen\sphinxpxdimen
\fi \sphinxpxdimen=.75bp\relax
\ifdefined\pdfimageresolution
    \pdfimageresolution= \numexpr \dimexpr1in\relax/\sphinxpxdimen\relax
\fi
%% let collapsible pdf bookmarks panel have high depth per default
\PassOptionsToPackage{bookmarksdepth=5}{hyperref}


\PassOptionsToPackage{warn}{textcomp}
\usepackage[utf8]{inputenc}
\ifdefined\DeclareUnicodeCharacter
% support both utf8 and utf8x syntaxes
  \ifdefined\DeclareUnicodeCharacterAsOptional
    \def\sphinxDUC#1{\DeclareUnicodeCharacter{"#1}}
  \else
    \let\sphinxDUC\DeclareUnicodeCharacter
  \fi
  \sphinxDUC{00A0}{\nobreakspace}
  \sphinxDUC{2500}{\sphinxunichar{2500}}
  \sphinxDUC{2502}{\sphinxunichar{2502}}
  \sphinxDUC{2514}{\sphinxunichar{2514}}
  \sphinxDUC{251C}{\sphinxunichar{251C}}
  \sphinxDUC{2572}{\textbackslash}
\fi
\usepackage{cmap}
\usepackage[T1]{fontenc}
\usepackage{amsmath,amssymb,amstext}
\usepackage{babel}



\usepackage{tgtermes}
\usepackage{tgheros}
\renewcommand{\ttdefault}{txtt}



\usepackage[Sonny]{fncychap}
\ChNameVar{\Large\normalfont\sffamily}
\ChTitleVar{\Large\normalfont\sffamily}
\usepackage{sphinx}

\fvset{fontsize=auto}
\usepackage{geometry}


% Include hyperref last.
\usepackage{hyperref}
% Fix anchor placement for figures with captions.
\usepackage{hypcap}% it must be loaded after hyperref.
% Set up styles of URL: it should be placed after hyperref.
\urlstyle{same}

\addto\captionsenglish{\renewcommand{\contentsname}{Contents:}}

\usepackage{sphinxmessages}
\setcounter{tocdepth}{1}



\title{CalSciPy}
\date{Feb 15, 2023}
\release{0.1.5}
\author{Darik A.\@{} O\textquotesingle{}Neil}
\newcommand{\sphinxlogo}{\vbox{}}
\renewcommand{\releasename}{Release}
\makeindex
\begin{document}

\ifdefined\shorthandoff
  \ifnum\catcode`\=\string=\active\shorthandoff{=}\fi
  \ifnum\catcode`\"=\active\shorthandoff{"}\fi
\fi

\pagestyle{empty}
\sphinxmaketitle
\pagestyle{plain}
\sphinxtableofcontents
\pagestyle{normal}
\phantomsection\label{\detokenize{index::doc}}


\sphinxstepscope


\chapter{Introduction}
\label{\detokenize{Introduction:introduction}}\label{\detokenize{Introduction::doc}}
\sphinxAtStartPar
\sphinxstylestrong{CalSciPy} contains a variety of useful methods for handling, processing, and visualizing calcium imaging data.
It\textquotesingle{}s intended to be a collection of useful, well\sphinxhyphen{}documented functions often used in boilerplate code alongside software
packages such as \sphinxhref{https://github.com/flatironinstitute/CaImAn}{Caiman}, \sphinxhref{https://github.com/losonczylab/sima}{SIMA},
and \sphinxhref{https://github.com/MouseLand/suite2p}{Suite2P}.


\section{Motivation}
\label{\detokenize{Introduction:motivation}}
\sphinxAtStartPar
I noticed I was often re\sphinxhyphen{}writing or copy/pasting a lot of code between environments when working with calcium imaging
data. I started this package so you don\textquotesingle{}t have to. No more wasting time writing 6 lines to simply preview your tiff
stack, extract a particular channel, or bin some spikes. No more vague exceptions or incomplete documentation when re\sphinxhyphen{}using
a hastily\sphinxhyphen{}made function from 2 months ago. Alongside these time\sphinxhyphen{}savers, I\textquotesingle{}ve also included some more non\sphinxhyphen{}trivial methods
that are particularly useful.


\section{Limitations}
\label{\detokenize{Introduction:limitations}}
\sphinxAtStartPar
The current distribution for the package is incomplete. When each module has its associated unit tests complete, it will
be pushed.

\sphinxstepscope


\chapter{Installation}
\label{\detokenize{Installation:installation}}\label{\detokenize{Installation::doc}}

\section{Full Install}
\label{\detokenize{Installation:full-install}}
\sphinxAtStartPar
Enter \sphinxcode{\sphinxupquote{pip install CalSciPy}} in your terminal.


\section{Partial Install}
\label{\detokenize{Installation:partial-install}}
\sphinxAtStartPar
Enter \sphinxcode{\sphinxupquote{pip install CalSciPy\sphinxhyphen{}\textless{}subpackage\textgreater{}}} in your terminal.

\sphinxstepscope


\chapter{Overview}
\label{\detokenize{Sub-Packages:overview}}\label{\detokenize{Sub-Packages::doc}}\begin{itemize}
\item {} 
\sphinxAtStartPar
{\hyperref[\detokenize{Sub-Packages:bruker-module}]{\sphinxcrossref{\DUrole{std,std-ref}{Bruker}}}}

\item {} 
\sphinxAtStartPar
{\hyperref[\detokenize{Sub-Packages:coloring-module}]{\sphinxcrossref{\DUrole{std,std-ref}{Coloring}}}}

\item {} 
\sphinxAtStartPar
{\hyperref[\detokenize{Sub-Packages:event-processing-module}]{\sphinxcrossref{\DUrole{std,std-ref}{Event Processing}}}}

\item {} 
\sphinxAtStartPar
{\hyperref[\detokenize{Sub-Packages:io-module}]{\sphinxcrossref{\DUrole{std,std-ref}{Input/Output (I/O)}}}}

\item {} 
\sphinxAtStartPar
{\hyperref[\detokenize{Sub-Packages:image-processing-module}]{\sphinxcrossref{\DUrole{std,std-ref}{Image Processing}}}}

\item {} 
\sphinxAtStartPar
{\hyperref[\detokenize{Sub-Packages:interactive-visuals-module}]{\sphinxcrossref{\DUrole{std,std-ref}{Interactive Visuals}}}}

\item {} 
\sphinxAtStartPar
{\hyperref[\detokenize{Sub-Packages:reorganization-module}]{\sphinxcrossref{\DUrole{std,std-ref}{Reorganization}}}}

\item {} 
\sphinxAtStartPar
{\hyperref[\detokenize{Sub-Packages:trace-processing-module}]{\sphinxcrossref{\DUrole{std,std-ref}{Trace Processing}}}}

\item {} 
\sphinxAtStartPar
{\hyperref[\detokenize{Sub-Packages:static-visuals-module}]{\sphinxcrossref{\DUrole{std,std-ref}{Static Visuals}}}}

\end{itemize}


\section{Bruker}
\label{\detokenize{Sub-Packages:bruker}}\label{\detokenize{Sub-Packages:bruker-module}}
\begin{DUlineblock}{0em}
\item[] Write me
\item[] Write me
\item[] Write me
\item[] Write me
\end{DUlineblock}

\sphinxstepscope


\subsection{CalSciPy.bruker module}
\label{\detokenize{CalSciPy.bruker:module-CalSciPy.bruker}}\label{\detokenize{CalSciPy.bruker:calscipy-bruker-module}}\label{\detokenize{CalSciPy.bruker::doc}}\index{module@\spxentry{module}!CalSciPy.bruker@\spxentry{CalSciPy.bruker}}\index{CalSciPy.bruker@\spxentry{CalSciPy.bruker}!module@\spxentry{module}}\index{determine\_bruker\_folder\_contents() (in module CalSciPy.bruker)@\spxentry{determine\_bruker\_folder\_contents()}\spxextra{in module CalSciPy.bruker}}

\begin{fulllineitems}
\phantomsection\label{\detokenize{CalSciPy.bruker:CalSciPy.bruker.determine_bruker_folder_contents}}
\pysigstartsignatures
\pysiglinewithargsret{\sphinxcode{\sphinxupquote{CalSciPy.bruker.}}\sphinxbfcode{\sphinxupquote{determine\_bruker\_folder\_contents}}}{\emph{\DUrole{n}{folder}}}{}
\pysigstopsignatures
\sphinxAtStartPar
This function determines the number of channels and planes within a folder containing .tif files
exported by Bruker\textquotesingle{}s Prairieview software. It also determines the size of the images (frames, y\sphinxhyphen{}pixels, x\sphinxhyphen{}pixels)
\begin{quote}\begin{description}
\sphinxlineitem{Parameters}
\sphinxAtStartPar
\sphinxstyleliteralstrong{\sphinxupquote{folder}} (\sphinxhref{https://docs.python.org/3/library/stdtypes.html\#str}{\sphinxstyleliteralemphasis{\sphinxupquote{str}}}\sphinxstyleliteralemphasis{\sphinxupquote{ or }}\sphinxhref{https://docs.python.org/3/library/pathlib.html\#pathlib.Path}{\sphinxstyleliteralemphasis{\sphinxupquote{pathlib.Path}}}) \sphinxhyphen{}\sphinxhyphen{} folder containing bruker imaging data

\sphinxlineitem{Returns}
\sphinxAtStartPar
channels, planes, frames, height, width

\sphinxlineitem{Return type}
\sphinxAtStartPar
\sphinxhref{https://docs.python.org/3/library/stdtypes.html\#tuple}{tuple}{[}\sphinxhref{https://docs.python.org/3/library/functions.html\#int}{int}, \sphinxhref{https://docs.python.org/3/library/functions.html\#int}{int}, \sphinxhref{https://docs.python.org/3/library/functions.html\#int}{int}, \sphinxhref{https://docs.python.org/3/library/functions.html\#int}{int}, \sphinxhref{https://docs.python.org/3/library/functions.html\#int}{int}{]}

\end{description}\end{quote}

\end{fulllineitems}

\index{load\_bruker\_tiffs() (in module CalSciPy.bruker)@\spxentry{load\_bruker\_tiffs()}\spxextra{in module CalSciPy.bruker}}

\begin{fulllineitems}
\phantomsection\label{\detokenize{CalSciPy.bruker:CalSciPy.bruker.load_bruker_tiffs}}
\pysigstartsignatures
\pysiglinewithargsret{\sphinxcode{\sphinxupquote{CalSciPy.bruker.}}\sphinxbfcode{\sphinxupquote{load\_bruker\_tiffs}}}{\emph{\DUrole{n}{folder}}, \emph{\DUrole{n}{channels}\DUrole{o}{=}\DUrole{default_value}{None}}, \emph{\DUrole{n}{planes}\DUrole{o}{=}\DUrole{default_value}{None}}}{}
\pysigstopsignatures
\sphinxAtStartPar
Load a sequence of .tif files from a directory containing .tif files exported by Bruker\textquotesingle{}s Prairieview software to a
numpy array. If multiple channels or multiple planes exist, each channel and plane combination is loaded to a
separate numpy array.
\begin{quote}\begin{description}
\sphinxlineitem{Parameters}\begin{itemize}
\item {} 
\sphinxAtStartPar
\sphinxstyleliteralstrong{\sphinxupquote{folder}} (\sphinxhref{https://docs.python.org/3/library/stdtypes.html\#str}{\sphinxstyleliteralemphasis{\sphinxupquote{str}}}\sphinxstyleliteralemphasis{\sphinxupquote{ or }}\sphinxhref{https://docs.python.org/3/library/pathlib.html\#pathlib.Path}{\sphinxstyleliteralemphasis{\sphinxupquote{pathlib.Path}}}) \sphinxhyphen{}\sphinxhyphen{} folder containing a sequence of single frame tiff files

\item {} 
\sphinxAtStartPar
\sphinxstyleliteralstrong{\sphinxupquote{channels}} (\sphinxstyleliteralemphasis{\sphinxupquote{Optional}}\sphinxstyleliteralemphasis{\sphinxupquote{{[}}}\sphinxhref{https://docs.python.org/3/library/functions.html\#int}{\sphinxstyleliteralemphasis{\sphinxupquote{int}}}\sphinxstyleliteralemphasis{\sphinxupquote{{]} }}\sphinxstyleliteralemphasis{\sphinxupquote{= None}}) \sphinxhyphen{}\sphinxhyphen{} specific channel to load from dataset (zero\sphinxhyphen{}indexed)

\item {} 
\sphinxAtStartPar
\sphinxstyleliteralstrong{\sphinxupquote{planes}} (\sphinxstyleliteralemphasis{\sphinxupquote{Optional}}\sphinxstyleliteralemphasis{\sphinxupquote{{[}}}\sphinxhref{https://docs.python.org/3/library/functions.html\#int}{\sphinxstyleliteralemphasis{\sphinxupquote{int}}}\sphinxstyleliteralemphasis{\sphinxupquote{{]} }}\sphinxstyleliteralemphasis{\sphinxupquote{= None}}) \sphinxhyphen{}\sphinxhyphen{} specific plane to load from dataset (zero\sphinxhyphen{}indexed)

\end{itemize}

\sphinxlineitem{Returns}
\sphinxAtStartPar
All .tif files in the directory loaded to a tuple of numpy arrays
(frames, y\sphinxhyphen{}pixels, x\sphinxhyphen{}pixels, \sphinxcode{\sphinxupquote{np.uint16}})

\sphinxlineitem{Return type}
\sphinxAtStartPar
\sphinxhref{https://docs.python.org/3/library/stdtypes.html\#tuple}{tuple}{[}\sphinxhref{https://numpy.org/doc/1.24/reference/generated/numpy.ndarray.html\#numpy.ndarray}{numpy.ndarray}{]}

\end{description}\end{quote}

\end{fulllineitems}

\index{pretty\_print\_image\_description() (in module CalSciPy.bruker)@\spxentry{pretty\_print\_image\_description()}\spxextra{in module CalSciPy.bruker}}

\begin{fulllineitems}
\phantomsection\label{\detokenize{CalSciPy.bruker:CalSciPy.bruker.pretty_print_image_description}}
\pysigstartsignatures
\pysiglinewithargsret{\sphinxcode{\sphinxupquote{CalSciPy.bruker.}}\sphinxbfcode{\sphinxupquote{pretty\_print\_image\_description}}}{\emph{\DUrole{n}{channels}}, \emph{\DUrole{n}{planes}}, \emph{\DUrole{n}{frames}}, \emph{\DUrole{n}{height}}, \emph{\DUrole{n}{width}}}{}
\pysigstopsignatures
\sphinxAtStartPar
Function prints the description of an imaging dataset as a table.
\begin{quote}\begin{description}
\sphinxlineitem{Parameters}\begin{itemize}
\item {} 
\sphinxAtStartPar
\sphinxstyleliteralstrong{\sphinxupquote{channels}} (\sphinxhref{https://docs.python.org/3/library/functions.html\#int}{\sphinxstyleliteralemphasis{\sphinxupquote{int}}}) \sphinxhyphen{}\sphinxhyphen{} number of channels

\item {} 
\sphinxAtStartPar
\sphinxstyleliteralstrong{\sphinxupquote{planes}} (\sphinxhref{https://docs.python.org/3/library/functions.html\#int}{\sphinxstyleliteralemphasis{\sphinxupquote{int}}}) \sphinxhyphen{}\sphinxhyphen{} number of planes

\item {} 
\sphinxAtStartPar
\sphinxstyleliteralstrong{\sphinxupquote{frames}} (\sphinxhref{https://docs.python.org/3/library/functions.html\#int}{\sphinxstyleliteralemphasis{\sphinxupquote{int}}}) \sphinxhyphen{}\sphinxhyphen{} number of frames

\item {} 
\sphinxAtStartPar
\sphinxstyleliteralstrong{\sphinxupquote{height}} (\sphinxhref{https://docs.python.org/3/library/functions.html\#int}{\sphinxstyleliteralemphasis{\sphinxupquote{int}}}) \sphinxhyphen{}\sphinxhyphen{} y\sphinxhyphen{}pixels

\item {} 
\sphinxAtStartPar
\sphinxstyleliteralstrong{\sphinxupquote{width}} (\sphinxhref{https://docs.python.org/3/library/functions.html\#int}{\sphinxstyleliteralemphasis{\sphinxupquote{int}}}) \sphinxhyphen{}\sphinxhyphen{} x\sphinxhyphen{}pixels

\end{itemize}

\sphinxlineitem{Return type}
\sphinxAtStartPar
None

\end{description}\end{quote}

\end{fulllineitems}

\index{repackage\_bruker\_tiffs() (in module CalSciPy.bruker)@\spxentry{repackage\_bruker\_tiffs()}\spxextra{in module CalSciPy.bruker}}

\begin{fulllineitems}
\phantomsection\label{\detokenize{CalSciPy.bruker:CalSciPy.bruker.repackage_bruker_tiffs}}
\pysigstartsignatures
\pysiglinewithargsret{\sphinxcode{\sphinxupquote{CalSciPy.bruker.}}\sphinxbfcode{\sphinxupquote{repackage\_bruker\_tiffs}}}{\emph{\DUrole{n}{input\_folder}}, \emph{\DUrole{n}{output\_folder}}, \emph{\DUrole{o}{*}\DUrole{n}{args}}}{}
\pysigstopsignatures
\sphinxAtStartPar
Repackages a folder containing .tif files exported by Bruker\textquotesingle{}s Prairieview software into a sequence of \textless{}4 GB .tif
stacks.
\begin{quote}\begin{description}
\sphinxlineitem{Parameters}\begin{itemize}
\item {} 
\sphinxAtStartPar
\sphinxstyleliteralstrong{\sphinxupquote{input\_folder}} (\sphinxhref{https://docs.python.org/3/library/stdtypes.html\#str}{\sphinxstyleliteralemphasis{\sphinxupquote{str}}}\sphinxstyleliteralemphasis{\sphinxupquote{ or }}\sphinxhref{https://docs.python.org/3/library/pathlib.html\#pathlib.Path}{\sphinxstyleliteralemphasis{\sphinxupquote{pathlib.Path}}}) \sphinxhyphen{}\sphinxhyphen{} folder containing a sequence of single frame .tif files exported by Bruker\textquotesingle{}s Prairieview

\item {} 
\sphinxAtStartPar
\sphinxstyleliteralstrong{\sphinxupquote{output\_folder}} (\sphinxhref{https://docs.python.org/3/library/stdtypes.html\#str}{\sphinxstyleliteralemphasis{\sphinxupquote{str}}}\sphinxstyleliteralemphasis{\sphinxupquote{ or }}\sphinxhref{https://docs.python.org/3/library/pathlib.html\#pathlib.Path}{\sphinxstyleliteralemphasis{\sphinxupquote{pathlib.Path}}}) \sphinxhyphen{}\sphinxhyphen{} empty folder where .tif stacks will be saved

\item {} 
\sphinxAtStartPar
\sphinxstyleliteralstrong{\sphinxupquote{args}} (\sphinxhref{https://docs.python.org/3/library/functions.html\#int}{\sphinxstyleliteralemphasis{\sphinxupquote{int}}}) \sphinxhyphen{}\sphinxhyphen{} optional argument to indicate the repackaging of a specific channel and/or plane

\end{itemize}

\sphinxlineitem{Return type}
\sphinxAtStartPar
None

\end{description}\end{quote}

\end{fulllineitems}



\section{Coloring}
\label{\detokenize{Sub-Packages:coloring}}\label{\detokenize{Sub-Packages:coloring-module}}
\begin{DUlineblock}{0em}
\item[] Write me
\item[] Write me
\item[] Write me
\item[] Write me
\end{DUlineblock}


\subsection{Coloring Methods}
\label{\detokenize{Sub-Packages:coloring-methods}}
\begin{DUlineblock}{0em}
\item[] Import me
\end{DUlineblock}


\section{Event Processing}
\label{\detokenize{Sub-Packages:event-processing}}\label{\detokenize{Sub-Packages:event-processing-module}}
\begin{DUlineblock}{0em}
\item[] Write me
\item[] Write me
\item[] Write me
\item[] Write me
\end{DUlineblock}

\sphinxstepscope


\subsection{CalSciPy.event\_processing module}
\label{\detokenize{CalSciPy.event_processing:module-CalSciPy.event_processing}}\label{\detokenize{CalSciPy.event_processing:calscipy-event-processing-module}}\label{\detokenize{CalSciPy.event_processing::doc}}\index{module@\spxentry{module}!CalSciPy.event\_processing@\spxentry{CalSciPy.event\_processing}}\index{CalSciPy.event\_processing@\spxentry{CalSciPy.event\_processing}!module@\spxentry{module}}\index{bin\_events() (in module CalSciPy.event\_processing)@\spxentry{bin\_events()}\spxextra{in module CalSciPy.event\_processing}}

\begin{fulllineitems}
\phantomsection\label{\detokenize{CalSciPy.event_processing:CalSciPy.event_processing.bin_events}}
\pysigstartsignatures
\pysiglinewithargsret{\sphinxcode{\sphinxupquote{CalSciPy.event\_processing.}}\sphinxbfcode{\sphinxupquote{bin\_events}}}{\emph{\DUrole{n}{matrix}}, \emph{\DUrole{n}{bin\_length}}}{}
\pysigstopsignatures
\sphinxAtStartPar
Bin events (e.g., spikes) using specified bin length
\begin{quote}\begin{description}
\sphinxlineitem{Parameters}\begin{itemize}
\item {} 
\sphinxAtStartPar
\sphinxstyleliteralstrong{\sphinxupquote{matrix}} (\sphinxhref{https://numpy.org/doc/1.24/reference/generated/numpy.ndarray.html\#numpy.ndarray}{\sphinxstyleliteralemphasis{\sphinxupquote{numpy.ndarray}}}) \sphinxhyphen{}\sphinxhyphen{} matrix of n features x m samples

\item {} 
\sphinxAtStartPar
\sphinxstyleliteralstrong{\sphinxupquote{bin\_length}} (\sphinxhref{https://docs.python.org/3/library/functions.html\#int}{\sphinxstyleliteralemphasis{\sphinxupquote{int}}}) \sphinxhyphen{}\sphinxhyphen{} length of bin

\end{itemize}

\sphinxlineitem{Returns}
\sphinxAtStartPar
binned\_matrix of n features x m bins

\sphinxlineitem{Return type}
\sphinxAtStartPar
\sphinxhref{https://numpy.org/doc/1.24/reference/generated/numpy.ndarray.html\#numpy.ndarray}{numpy.ndarray}

\end{description}\end{quote}

\end{fulllineitems}

\index{calculate\_firing\_rates() (in module CalSciPy.event\_processing)@\spxentry{calculate\_firing\_rates()}\spxextra{in module CalSciPy.event\_processing}}

\begin{fulllineitems}
\phantomsection\label{\detokenize{CalSciPy.event_processing:CalSciPy.event_processing.calculate_firing_rates}}
\pysigstartsignatures
\pysiglinewithargsret{\sphinxcode{\sphinxupquote{CalSciPy.event\_processing.}}\sphinxbfcode{\sphinxupquote{calculate\_firing\_rates}}}{\emph{\DUrole{n}{spike\_probability\_matrix}}, \emph{\DUrole{n}{frame\_rate}\DUrole{o}{=}\DUrole{default_value}{30}}, \emph{\DUrole{n}{in\_place}\DUrole{o}{=}\DUrole{default_value}{False}}}{}
\pysigstopsignatures
\sphinxAtStartPar
Calculate firing rates
\begin{quote}\begin{description}
\sphinxlineitem{Parameters}\begin{itemize}
\item {} 
\sphinxAtStartPar
\sphinxstyleliteralstrong{\sphinxupquote{spike\_probability\_matrix}} (\sphinxhref{https://numpy.org/doc/1.24/reference/generated/numpy.ndarray.html\#numpy.ndarray}{\sphinxstyleliteralemphasis{\sphinxupquote{numpy.ndarray}}}) \sphinxhyphen{}\sphinxhyphen{} matrix of n neuron x m samples where each element is the probability of a spike

\item {} 
\sphinxAtStartPar
\sphinxstyleliteralstrong{\sphinxupquote{frame\_rate}} (\sphinxstyleliteralemphasis{\sphinxupquote{float = 30}}) \sphinxhyphen{}\sphinxhyphen{} frame rate of dataset

\item {} 
\sphinxAtStartPar
\sphinxstyleliteralstrong{\sphinxupquote{in\_place}} (\sphinxstyleliteralemphasis{\sphinxupquote{bool = False}}) \sphinxhyphen{}\sphinxhyphen{} boolean indicating whether to perform calculation in\sphinxhyphen{}place

\end{itemize}

\sphinxlineitem{Returns}
\sphinxAtStartPar
firing matrix of n neurons x m samples where each element is a binary indicating presence of spike event

\sphinxlineitem{Return type}
\sphinxAtStartPar
\sphinxhref{https://numpy.org/doc/1.24/reference/generated/numpy.ndarray.html\#numpy.ndarray}{numpy.ndarray}

\end{description}\end{quote}

\end{fulllineitems}

\index{calculate\_mean\_firing\_rates() (in module CalSciPy.event\_processing)@\spxentry{calculate\_mean\_firing\_rates()}\spxextra{in module CalSciPy.event\_processing}}

\begin{fulllineitems}
\phantomsection\label{\detokenize{CalSciPy.event_processing:CalSciPy.event_processing.calculate_mean_firing_rates}}
\pysigstartsignatures
\pysiglinewithargsret{\sphinxcode{\sphinxupquote{CalSciPy.event\_processing.}}\sphinxbfcode{\sphinxupquote{calculate\_mean\_firing\_rates}}}{\emph{\DUrole{n}{firing\_matrix}}}{}
\pysigstopsignatures
\sphinxAtStartPar
Calculate mean firing rate
\begin{quote}\begin{description}
\sphinxlineitem{Parameters}
\sphinxAtStartPar
\sphinxstyleliteralstrong{\sphinxupquote{firing\_matrix}} (\sphinxhref{https://numpy.org/doc/1.24/reference/generated/numpy.ndarray.html\#numpy.ndarray}{\sphinxstyleliteralemphasis{\sphinxupquote{numpy.ndarray}}}) \sphinxhyphen{}\sphinxhyphen{} matrix of n neuron x m samples where each element is either a spike or an
instantaneous firing rate

\sphinxlineitem{Returns}
\sphinxAtStartPar
1\sphinxhyphen{}D vector of mean firing rates

\sphinxlineitem{Return type}
\sphinxAtStartPar
\sphinxhref{https://numpy.org/doc/1.24/reference/generated/numpy.ndarray.html\#numpy.ndarray}{numpy.ndarray}

\end{description}\end{quote}

\end{fulllineitems}

\index{gaussian\_smooth\_firing\_rates() (in module CalSciPy.event\_processing)@\spxentry{gaussian\_smooth\_firing\_rates()}\spxextra{in module CalSciPy.event\_processing}}

\begin{fulllineitems}
\phantomsection\label{\detokenize{CalSciPy.event_processing:CalSciPy.event_processing.gaussian_smooth_firing_rates}}
\pysigstartsignatures
\pysiglinewithargsret{\sphinxcode{\sphinxupquote{CalSciPy.event\_processing.}}\sphinxbfcode{\sphinxupquote{gaussian\_smooth\_firing\_rates}}}{\emph{\DUrole{n}{firing\_matrix}}, \emph{\DUrole{n}{sigma}}, \emph{\DUrole{n}{in\_place}\DUrole{o}{=}\DUrole{default_value}{False}}}{}
\pysigstopsignatures
\sphinxAtStartPar
Normalize firing rates using a 1\sphinxhyphen{}D gaussian filter
\begin{quote}\begin{description}
\sphinxlineitem{Parameters}\begin{itemize}
\item {} 
\sphinxAtStartPar
\sphinxstyleliteralstrong{\sphinxupquote{firing\_matrix}} (\sphinxhref{https://numpy.org/doc/1.24/reference/generated/numpy.ndarray.html\#numpy.ndarray}{\sphinxstyleliteralemphasis{\sphinxupquote{numpy.ndarray}}}) \sphinxhyphen{}\sphinxhyphen{} matrix of n neuron x m samples where each element is either a spike or an
instantaneous firing rate

\item {} 
\sphinxAtStartPar
\sphinxstyleliteralstrong{\sphinxupquote{sigma}} (\sphinxhref{https://docs.python.org/3/library/functions.html\#float}{\sphinxstyleliteralemphasis{\sphinxupquote{float}}}) \sphinxhyphen{}\sphinxhyphen{} standard deviation of gaussian kernel

\item {} 
\sphinxAtStartPar
\sphinxstyleliteralstrong{\sphinxupquote{in\_place}} (\sphinxstyleliteralemphasis{\sphinxupquote{bool = False}}) \sphinxhyphen{}\sphinxhyphen{} boolean indicating whether to perform calculation in\sphinxhyphen{}place

\end{itemize}

\sphinxlineitem{Returns}
\sphinxAtStartPar
gaussian\sphinxhyphen{}smoothed firing rate matrix of n neurons x m samples

\sphinxlineitem{Return type}
\sphinxAtStartPar
\sphinxhref{https://numpy.org/doc/1.24/reference/generated/numpy.ndarray.html\#numpy.ndarray}{numpy.ndarray}

\end{description}\end{quote}

\end{fulllineitems}

\index{normalize\_firing\_rates() (in module CalSciPy.event\_processing)@\spxentry{normalize\_firing\_rates()}\spxextra{in module CalSciPy.event\_processing}}

\begin{fulllineitems}
\phantomsection\label{\detokenize{CalSciPy.event_processing:CalSciPy.event_processing.normalize_firing_rates}}
\pysigstartsignatures
\pysiglinewithargsret{\sphinxcode{\sphinxupquote{CalSciPy.event\_processing.}}\sphinxbfcode{\sphinxupquote{normalize\_firing\_rates}}}{\emph{\DUrole{n}{firing\_matrix}}, \emph{\DUrole{n}{in\_place}\DUrole{o}{=}\DUrole{default_value}{False}}}{}
\pysigstopsignatures
\sphinxAtStartPar
Normalize firing rates by scaling to a max of 1.0. Non\sphinxhyphen{}negativity constrained.
\begin{quote}\begin{description}
\sphinxlineitem{Parameters}\begin{itemize}
\item {} 
\sphinxAtStartPar
\sphinxstyleliteralstrong{\sphinxupquote{firing\_matrix}} (\sphinxhref{https://numpy.org/doc/1.24/reference/generated/numpy.ndarray.html\#numpy.ndarray}{\sphinxstyleliteralemphasis{\sphinxupquote{numpy.ndarray}}}) \sphinxhyphen{}\sphinxhyphen{} matrix of n neuron x m samples where each element is either a spike or an
instantaneous firing rate

\item {} 
\sphinxAtStartPar
\sphinxstyleliteralstrong{\sphinxupquote{in\_place}} (\sphinxstyleliteralemphasis{\sphinxupquote{bool = False}}) \sphinxhyphen{}\sphinxhyphen{} boolean indicating whether to perform calculation in\sphinxhyphen{}place

\end{itemize}

\sphinxlineitem{Returns}
\sphinxAtStartPar
normalized firing rate matrix of n neurons x m samples

\sphinxlineitem{Return type}
\sphinxAtStartPar
\sphinxhref{https://numpy.org/doc/1.24/reference/generated/numpy.ndarray.html\#numpy.ndarray}{numpy.ndarray}

\end{description}\end{quote}

\end{fulllineitems}



\section{Input/Output (I/O)}
\label{\detokenize{Sub-Packages:input-output-i-o}}\label{\detokenize{Sub-Packages:io-module}}
\begin{DUlineblock}{0em}
\item[] Write me
\item[] Write me
\item[] Write me
\item[] Write me
\end{DUlineblock}

\sphinxstepscope


\subsection{CalSciPy.io\_tools module}
\label{\detokenize{CalSciPy.io_tools:module-CalSciPy.io_tools}}\label{\detokenize{CalSciPy.io_tools:calscipy-io-tools-module}}\label{\detokenize{CalSciPy.io_tools::doc}}\index{module@\spxentry{module}!CalSciPy.io\_tools@\spxentry{CalSciPy.io\_tools}}\index{CalSciPy.io\_tools@\spxentry{CalSciPy.io\_tools}!module@\spxentry{module}}\index{load\_all\_tiffs() (in module CalSciPy.io\_tools)@\spxentry{load\_all\_tiffs()}\spxextra{in module CalSciPy.io\_tools}}

\begin{fulllineitems}
\phantomsection\label{\detokenize{CalSciPy.io_tools:CalSciPy.io_tools.load_all_tiffs}}
\pysigstartsignatures
\pysiglinewithargsret{\sphinxcode{\sphinxupquote{CalSciPy.io\_tools.}}\sphinxbfcode{\sphinxupquote{load\_all\_tiffs}}}{\emph{\DUrole{n}{folder}}}{}
\pysigstopsignatures
\sphinxAtStartPar
Loads all .tif\textquotesingle{}s within a folder into a single numpy array. Assumes .tif files are the standard unsigned 16\sphinxhyphen{}bit
exported by the majority (all?) of imaging software.
\begin{quote}\begin{description}
\sphinxlineitem{Parameters}
\sphinxAtStartPar
\sphinxstyleliteralstrong{\sphinxupquote{folder}} (\sphinxhref{https://docs.python.org/3/library/stdtypes.html\#str}{\sphinxstyleliteralemphasis{\sphinxupquote{str}}}\sphinxstyleliteralemphasis{\sphinxupquote{ or }}\sphinxhref{https://docs.python.org/3/library/pathlib.html\#pathlib.Path}{\sphinxstyleliteralemphasis{\sphinxupquote{pathlib.Path}}}) \sphinxhyphen{}\sphinxhyphen{} folder containing a sequence of tiff stacks

\sphinxlineitem{Returns}
\sphinxAtStartPar
a numpy array containing the images (frames, y\sphinxhyphen{}pixels, x\sphinxhyphen{}pixels)

\sphinxlineitem{Return type}
\sphinxAtStartPar
\sphinxhref{https://numpy.org/doc/1.24/reference/generated/numpy.ndarray.html\#numpy.ndarray}{numpy.ndarray}

\end{description}\end{quote}

\end{fulllineitems}

\index{load\_binary\_meta() (in module CalSciPy.io\_tools)@\spxentry{load\_binary\_meta()}\spxextra{in module CalSciPy.io\_tools}}

\begin{fulllineitems}
\phantomsection\label{\detokenize{CalSciPy.io_tools:CalSciPy.io_tools.load_binary_meta}}
\pysigstartsignatures
\pysiglinewithargsret{\sphinxcode{\sphinxupquote{CalSciPy.io\_tools.}}\sphinxbfcode{\sphinxupquote{load\_binary\_meta}}}{\emph{\DUrole{n}{path}}}{}
\pysigstopsignatures
\sphinxAtStartPar
Loads the meta file for an associated binary video
\begin{quote}\begin{description}
\sphinxlineitem{Parameters}
\sphinxAtStartPar
\sphinxstyleliteralstrong{\sphinxupquote{path}} (\sphinxhref{https://docs.python.org/3/library/stdtypes.html\#str}{\sphinxstyleliteralemphasis{\sphinxupquote{str}}}\sphinxstyleliteralemphasis{\sphinxupquote{ or }}\sphinxhref{https://docs.python.org/3/library/pathlib.html\#pathlib.Path}{\sphinxstyleliteralemphasis{\sphinxupquote{pathlib.Path}}}) \sphinxhyphen{}\sphinxhyphen{} The meta file (.txt ext) or a directory containing metafile

\sphinxlineitem{Returns}
\sphinxAtStartPar
A tuple where (frames, y\sphinxhyphen{}pixels, x\sphinxhyphen{}pixels, \sphinxhref{https://numpy.org/doc/1.24/reference/generated/numpy.dtype.html\#numpy.dtype}{\sphinxcode{\sphinxupquote{numpy.dtype}}})

\sphinxlineitem{Return type}
\sphinxAtStartPar
\sphinxhref{https://docs.python.org/3/library/stdtypes.html\#tuple}{tuple}{[}\sphinxhref{https://docs.python.org/3/library/functions.html\#int}{int}, \sphinxhref{https://docs.python.org/3/library/functions.html\#int}{int}, \sphinxhref{https://docs.python.org/3/library/functions.html\#int}{int}, \sphinxhref{https://docs.python.org/3/library/stdtypes.html\#str}{str}{]}

\end{description}\end{quote}

\end{fulllineitems}

\index{load\_mapped\_binary() (in module CalSciPy.io\_tools)@\spxentry{load\_mapped\_binary()}\spxextra{in module CalSciPy.io\_tools}}

\begin{fulllineitems}
\phantomsection\label{\detokenize{CalSciPy.io_tools:CalSciPy.io_tools.load_mapped_binary}}
\pysigstartsignatures
\pysiglinewithargsret{\sphinxcode{\sphinxupquote{CalSciPy.io\_tools.}}\sphinxbfcode{\sphinxupquote{load\_mapped\_binary}}}{\emph{\DUrole{n}{path}}, \emph{\DUrole{n}{meta\_filename}\DUrole{o}{=}\DUrole{default_value}{None}}, \emph{\DUrole{o}{**}\DUrole{n}{kwargs}}}{}
\pysigstopsignatures
\sphinxAtStartPar
Loads a raw binary file as numpy array without loading into memory (memmap). Enter a directory to autogenerate the
default filenames (binary\_video, video\_meta.txt)
\begin{quote}\begin{description}
\sphinxlineitem{Parameters}\begin{itemize}
\item {} 
\sphinxAtStartPar
\sphinxstyleliteralstrong{\sphinxupquote{path}} (\sphinxhref{https://docs.python.org/3/library/stdtypes.html\#str}{\sphinxstyleliteralemphasis{\sphinxupquote{str}}}\sphinxstyleliteralemphasis{\sphinxupquote{ or }}\sphinxhref{https://docs.python.org/3/library/pathlib.html\#pathlib.Path}{\sphinxstyleliteralemphasis{\sphinxupquote{pathlib.Path}}}) \sphinxhyphen{}\sphinxhyphen{} absolute filepath for binary video or a folder containing a binary video with the default filename

\item {} 
\sphinxAtStartPar
\sphinxstyleliteralstrong{\sphinxupquote{meta\_filename}} (\sphinxstyleliteralemphasis{\sphinxupquote{Optional}}\sphinxstyleliteralemphasis{\sphinxupquote{{[}}}\sphinxhref{https://docs.python.org/3/library/stdtypes.html\#str}{\sphinxstyleliteralemphasis{\sphinxupquote{str}}}\sphinxstyleliteralemphasis{\sphinxupquote{{]} }}\sphinxstyleliteralemphasis{\sphinxupquote{= None}}) \sphinxhyphen{}\sphinxhyphen{} absolute path to meta file

\item {} 
\sphinxAtStartPar
\sphinxstyleliteralstrong{\sphinxupquote{mode}} \sphinxhyphen{}\sphinxhyphen{} mode used in loading numpy.memmap (str, default = "r")

\end{itemize}

\sphinxlineitem{Returns}
\sphinxAtStartPar
memmap (numpy) array (frames, y\sphinxhyphen{}pixels, x\sphinxhyphen{}pixels)

\sphinxlineitem{Return type}
\sphinxAtStartPar
\sphinxhref{https://numpy.org/doc/1.24/reference/generated/numpy.memmap.html\#numpy.memmap}{numpy.memmap}

\end{description}\end{quote}

\end{fulllineitems}

\index{load\_raw\_binary() (in module CalSciPy.io\_tools)@\spxentry{load\_raw\_binary()}\spxextra{in module CalSciPy.io\_tools}}

\begin{fulllineitems}
\phantomsection\label{\detokenize{CalSciPy.io_tools:CalSciPy.io_tools.load_raw_binary}}
\pysigstartsignatures
\pysiglinewithargsret{\sphinxcode{\sphinxupquote{CalSciPy.io\_tools.}}\sphinxbfcode{\sphinxupquote{load\_raw\_binary}}}{\emph{\DUrole{n}{path}}, \emph{\DUrole{n}{meta\_filename}\DUrole{o}{=}\DUrole{default_value}{None}}}{}
\pysigstopsignatures
\sphinxAtStartPar
Loads a raw binary file as a numpy array. Enter a directory to autogenerate the default
filenames (binary\_video, video\_meta.txt)
\begin{quote}\begin{description}
\sphinxlineitem{Parameters}\begin{itemize}
\item {} 
\sphinxAtStartPar
\sphinxstyleliteralstrong{\sphinxupquote{path}} (\sphinxhref{https://docs.python.org/3/library/stdtypes.html\#str}{\sphinxstyleliteralemphasis{\sphinxupquote{str}}}\sphinxstyleliteralemphasis{\sphinxupquote{ or }}\sphinxhref{https://docs.python.org/3/library/pathlib.html\#pathlib.Path}{\sphinxstyleliteralemphasis{\sphinxupquote{pathlib.Path}}}) \sphinxhyphen{}\sphinxhyphen{} absolute filepath for binary video or directory containing a file named binary video

\item {} 
\sphinxAtStartPar
\sphinxstyleliteralstrong{\sphinxupquote{meta\_filename}} (\sphinxstyleliteralemphasis{\sphinxupquote{Optional}}\sphinxstyleliteralemphasis{\sphinxupquote{{[}}}\sphinxhref{https://docs.python.org/3/library/stdtypes.html\#str}{\sphinxstyleliteralemphasis{\sphinxupquote{str}}}\sphinxstyleliteralemphasis{\sphinxupquote{{]} }}\sphinxstyleliteralemphasis{\sphinxupquote{= None}}) \sphinxhyphen{}\sphinxhyphen{} absolute path to meta file

\end{itemize}

\sphinxlineitem{Returns}
\sphinxAtStartPar
numpy array (frames, y\sphinxhyphen{}pixels, x\sphinxhyphen{}pixels)

\sphinxlineitem{Return type}
\sphinxAtStartPar
\sphinxhref{https://numpy.org/doc/1.24/reference/generated/numpy.ndarray.html\#numpy.ndarray}{numpy.ndarray}

\end{description}\end{quote}

\end{fulllineitems}

\index{load\_single\_tiff() (in module CalSciPy.io\_tools)@\spxentry{load\_single\_tiff()}\spxextra{in module CalSciPy.io\_tools}}

\begin{fulllineitems}
\phantomsection\label{\detokenize{CalSciPy.io_tools:CalSciPy.io_tools.load_single_tiff}}
\pysigstartsignatures
\pysiglinewithargsret{\sphinxcode{\sphinxupquote{CalSciPy.io\_tools.}}\sphinxbfcode{\sphinxupquote{load\_single\_tiff}}}{\emph{\DUrole{n}{path}}}{}
\pysigstopsignatures
\sphinxAtStartPar
Load a single .tif as a numpy array
\begin{quote}\begin{description}
\sphinxlineitem{Parameters}
\sphinxAtStartPar
\sphinxstyleliteralstrong{\sphinxupquote{path}} (\sphinxhref{https://docs.python.org/3/library/stdtypes.html\#str}{\sphinxstyleliteralemphasis{\sphinxupquote{str}}}\sphinxstyleliteralemphasis{\sphinxupquote{ or }}\sphinxhref{https://docs.python.org/3/library/pathlib.html\#pathlib.Path}{\sphinxstyleliteralemphasis{\sphinxupquote{pathlib.Path}}}) \sphinxhyphen{}\sphinxhyphen{} absolute filename

\sphinxlineitem{Returns}
\sphinxAtStartPar
numpy array (frames, y\sphinxhyphen{}pixels, x\sphinxhyphen{}pixels)

\sphinxlineitem{Return type}
\sphinxAtStartPar
\sphinxhref{https://numpy.org/doc/1.24/reference/generated/numpy.ndarray.html\#numpy.ndarray}{numpy.ndarray}

\end{description}\end{quote}

\end{fulllineitems}

\index{save\_raw\_binary() (in module CalSciPy.io\_tools)@\spxentry{save\_raw\_binary()}\spxextra{in module CalSciPy.io\_tools}}

\begin{fulllineitems}
\phantomsection\label{\detokenize{CalSciPy.io_tools:CalSciPy.io_tools.save_raw_binary}}
\pysigstartsignatures
\pysiglinewithargsret{\sphinxcode{\sphinxupquote{CalSciPy.io\_tools.}}\sphinxbfcode{\sphinxupquote{save\_raw\_binary}}}{\emph{\DUrole{n}{images}}, \emph{\DUrole{n}{path}}, \emph{\DUrole{n}{meta\_filename}}}{}
\pysigstopsignatures
\sphinxAtStartPar
Save a numpy array as a binary file with an associated meta .txt file
\begin{quote}\begin{description}
\sphinxlineitem{Parameters}\begin{itemize}
\item {} 
\sphinxAtStartPar
\sphinxstyleliteralstrong{\sphinxupquote{images}} (\sphinxhref{https://numpy.org/doc/1.24/reference/generated/numpy.ndarray.html\#numpy.ndarray}{\sphinxstyleliteralemphasis{\sphinxupquote{numpy.ndarray}}}) \sphinxhyphen{}\sphinxhyphen{} numpy array (frames, y\sphinxhyphen{}pixels, x\sphinxhyphen{}pixels)

\item {} 
\sphinxAtStartPar
\sphinxstyleliteralstrong{\sphinxupquote{path}} (\sphinxhref{https://docs.python.org/3/library/stdtypes.html\#str}{\sphinxstyleliteralemphasis{\sphinxupquote{str}}}) \sphinxhyphen{}\sphinxhyphen{} folder to save in or an absolute filepath for binary video file

\item {} 
\sphinxAtStartPar
\sphinxstyleliteralstrong{\sphinxupquote{meta\_filename}} (\sphinxhref{https://docs.python.org/3/library/stdtypes.html\#str}{\sphinxstyleliteralemphasis{\sphinxupquote{str}}}) \sphinxhyphen{}\sphinxhyphen{} absolute filepath for saving meta .txt file

\end{itemize}

\sphinxlineitem{Return type}
\sphinxAtStartPar
None

\end{description}\end{quote}

\end{fulllineitems}

\index{save\_single\_tiff() (in module CalSciPy.io\_tools)@\spxentry{save\_single\_tiff()}\spxextra{in module CalSciPy.io\_tools}}

\begin{fulllineitems}
\phantomsection\label{\detokenize{CalSciPy.io_tools:CalSciPy.io_tools.save_single_tiff}}
\pysigstartsignatures
\pysiglinewithargsret{\sphinxcode{\sphinxupquote{CalSciPy.io\_tools.}}\sphinxbfcode{\sphinxupquote{save\_single\_tiff}}}{\emph{\DUrole{n}{images}}, \emph{\DUrole{n}{path}}, \emph{\DUrole{n}{type\_=\textless{}class \textquotesingle{}numpy.uint16\textquotesingle{}\textgreater{}}}}{}
\pysigstopsignatures
\sphinxAtStartPar
Save a numpy array to a single .tif file. Size must be \textless{}4 GB.
\begin{quote}\begin{description}
\sphinxlineitem{Parameters}\begin{itemize}
\item {} 
\sphinxAtStartPar
\sphinxstyleliteralstrong{\sphinxupquote{images}} (\sphinxhref{https://numpy.org/doc/1.24/reference/generated/numpy.ndarray.html\#numpy.ndarray}{\sphinxstyleliteralemphasis{\sphinxupquote{numpy.ndarray}}}) \sphinxhyphen{}\sphinxhyphen{} numpy array {[}frames, y pixels, x pixels{]}

\item {} 
\sphinxAtStartPar
\sphinxstyleliteralstrong{\sphinxupquote{path}} (\sphinxhref{https://docs.python.org/3/library/stdtypes.html\#str}{\sphinxstyleliteralemphasis{\sphinxupquote{str}}}\sphinxstyleliteralemphasis{\sphinxupquote{ or }}\sphinxhref{https://docs.python.org/3/library/pathlib.html\#pathlib.Path}{\sphinxstyleliteralemphasis{\sphinxupquote{pathlib.Path}}}) \sphinxhyphen{}\sphinxhyphen{} filename or absolute path

\item {} 
\sphinxAtStartPar
\sphinxstyleliteralstrong{\sphinxupquote{type}} (\sphinxstyleliteralemphasis{\sphinxupquote{Optional}}\sphinxstyleliteralemphasis{\sphinxupquote{{[}}}\sphinxhref{https://numpy.org/doc/1.24/reference/generated/numpy.dtype.html\#numpy.dtype}{\sphinxstyleliteralemphasis{\sphinxupquote{numpy.dtype}}}\sphinxstyleliteralemphasis{\sphinxupquote{{]} }}\sphinxstyleliteralemphasis{\sphinxupquote{= numpy.uint16}}) \sphinxhyphen{}\sphinxhyphen{} type for saving

\end{itemize}

\sphinxlineitem{Return type}
\sphinxAtStartPar
None

\end{description}\end{quote}

\end{fulllineitems}

\index{save\_tiff\_stack() (in module CalSciPy.io\_tools)@\spxentry{save\_tiff\_stack()}\spxextra{in module CalSciPy.io\_tools}}

\begin{fulllineitems}
\phantomsection\label{\detokenize{CalSciPy.io_tools:CalSciPy.io_tools.save_tiff_stack}}
\pysigstartsignatures
\pysiglinewithargsret{\sphinxcode{\sphinxupquote{CalSciPy.io\_tools.}}\sphinxbfcode{\sphinxupquote{save\_tiff\_stack}}}{\emph{\DUrole{n}{images}}, \emph{\DUrole{n}{output\_folder}}, \emph{\DUrole{n}{type\_=\textless{}class \textquotesingle{}numpy.uint16\textquotesingle{}\textgreater{}}}}{}
\pysigstopsignatures
\sphinxAtStartPar
Save a numpy array to a sequence of .tif stacks
\begin{quote}\begin{description}
\sphinxlineitem{Parameters}\begin{itemize}
\item {} 
\sphinxAtStartPar
\sphinxstyleliteralstrong{\sphinxupquote{images}} (\sphinxhref{https://numpy.org/doc/1.24/reference/generated/numpy.ndarray.html\#numpy.ndarray}{\sphinxstyleliteralemphasis{\sphinxupquote{numpy.ndarray}}}) \sphinxhyphen{}\sphinxhyphen{} numpy array (frames, y\sphinxhyphen{}pixels, x\sphinxhyphen{}pixels)

\item {} 
\sphinxAtStartPar
\sphinxstyleliteralstrong{\sphinxupquote{output\_folder}} (\sphinxhref{https://docs.python.org/3/library/stdtypes.html\#str}{\sphinxstyleliteralemphasis{\sphinxupquote{str}}}\sphinxstyleliteralemphasis{\sphinxupquote{ or }}\sphinxhref{https://docs.python.org/3/library/pathlib.html\#pathlib.Path}{\sphinxstyleliteralemphasis{\sphinxupquote{pathlib.Path}}}) \sphinxhyphen{}\sphinxhyphen{} folder to save the sequence of .tif stacks

\item {} 
\sphinxAtStartPar
\sphinxstyleliteralstrong{\sphinxupquote{type}} (\sphinxstyleliteralemphasis{\sphinxupquote{Optional}}\sphinxstyleliteralemphasis{\sphinxupquote{{[}}}\sphinxhref{https://numpy.org/doc/1.24/reference/generated/numpy.dtype.html\#numpy.dtype}{\sphinxstyleliteralemphasis{\sphinxupquote{numpy.dtype}}}\sphinxstyleliteralemphasis{\sphinxupquote{{]} }}\sphinxstyleliteralemphasis{\sphinxupquote{= numpy.uint16}}) \sphinxhyphen{}\sphinxhyphen{} type for saving

\end{itemize}

\sphinxlineitem{Return type}
\sphinxAtStartPar
None

\end{description}\end{quote}

\end{fulllineitems}

\index{save\_video() (in module CalSciPy.io\_tools)@\spxentry{save\_video()}\spxextra{in module CalSciPy.io\_tools}}

\begin{fulllineitems}
\phantomsection\label{\detokenize{CalSciPy.io_tools:CalSciPy.io_tools.save_video}}
\pysigstartsignatures
\pysiglinewithargsret{\sphinxcode{\sphinxupquote{CalSciPy.io\_tools.}}\sphinxbfcode{\sphinxupquote{save\_video}}}{\emph{\DUrole{n}{images}}, \emph{\DUrole{n}{path}}, \emph{\DUrole{n}{fps}\DUrole{o}{=}\DUrole{default_value}{30.0}}}{}
\pysigstopsignatures
\sphinxAtStartPar
Save numpy array as an .mp4. Will be converted to uint8 if any other datatype.
\begin{quote}\begin{description}
\sphinxlineitem{Parameters}\begin{itemize}
\item {} 
\sphinxAtStartPar
\sphinxstyleliteralstrong{\sphinxupquote{images}} (\sphinxstyleliteralemphasis{\sphinxupquote{numpy.array}}) \sphinxhyphen{}\sphinxhyphen{} numpy array (frames, y\sphinxhyphen{}pixels, x\sphinxhyphen{}pixels)

\item {} 
\sphinxAtStartPar
\sphinxstyleliteralstrong{\sphinxupquote{path}} (\sphinxhref{https://docs.python.org/3/library/stdtypes.html\#str}{\sphinxstyleliteralemphasis{\sphinxupquote{str}}}\sphinxstyleliteralemphasis{\sphinxupquote{ or }}\sphinxhref{https://docs.python.org/3/library/pathlib.html\#pathlib.Path}{\sphinxstyleliteralemphasis{\sphinxupquote{pathlib.Path}}}) \sphinxhyphen{}\sphinxhyphen{} absolute filepath or filename

\item {} 
\sphinxAtStartPar
\sphinxstyleliteralstrong{\sphinxupquote{fps}} (\sphinxstyleliteralemphasis{\sphinxupquote{float = 30.0}}) \sphinxhyphen{}\sphinxhyphen{} frame rate for saved video

\end{itemize}

\sphinxlineitem{Return type}
\sphinxAtStartPar
None

\end{description}\end{quote}

\end{fulllineitems}



\section{Image Processing}
\label{\detokenize{Sub-Packages:image-processing}}\label{\detokenize{Sub-Packages:image-processing-module}}
\begin{DUlineblock}{0em}
\item[] Write me
\item[] Write me
\item[] Write me
\item[] Write me
\end{DUlineblock}

\sphinxstepscope


\subsection{CalSciPy.image\_processing module}
\label{\detokenize{CalSciPy.image_processing:module-CalSciPy.image_processing}}\label{\detokenize{CalSciPy.image_processing:calscipy-image-processing-module}}\label{\detokenize{CalSciPy.image_processing::doc}}\index{module@\spxentry{module}!CalSciPy.image\_processing@\spxentry{CalSciPy.image\_processing}}\index{CalSciPy.image\_processing@\spxentry{CalSciPy.image\_processing}!module@\spxentry{module}}\index{blockwise\_fast\_filter\_tiff() (in module CalSciPy.image\_processing)@\spxentry{blockwise\_fast\_filter\_tiff()}\spxextra{in module CalSciPy.image\_processing}}

\begin{fulllineitems}
\phantomsection\label{\detokenize{CalSciPy.image_processing:CalSciPy.image_processing.blockwise_fast_filter_tiff}}
\pysigstartsignatures
\pysiglinewithargsret{\sphinxcode{\sphinxupquote{CalSciPy.image\_processing.}}\sphinxbfcode{\sphinxupquote{blockwise\_fast\_filter\_tiff}}}{\emph{\DUrole{n}{images}}, \emph{\DUrole{n}{mask}\DUrole{o}{=}\DUrole{default_value}{array({[}{[}{[}1.0, 1.0, 1.0{]}, {[}1.0, 1.0, 1.0{]}, {[}1.0, 1.0, 1.0{]}{]}, {[}{[}1.0, 1.0, 1.0{]}, {[}1.0, 1.0, 1.0{]}, {[}1.0, 1.0, 1.0{]}{]}, {[}{[}1.0, 1.0, 1.0{]}, {[}1.0, 1.0, 1.0{]}, {[}1.0, 1.0, 1.0{]}{]}{]})}}, \emph{\DUrole{n}{block\_size}\DUrole{o}{=}\DUrole{default_value}{21000}}, \emph{\DUrole{n}{block\_buffer}\DUrole{o}{=}\DUrole{default_value}{500}}}{}
\pysigstopsignatures
\sphinxAtStartPar
GPU\sphinxhyphen{}parallelized multidimensional median filter performed in overlapping blocks.

\sphinxAtStartPar
Designed for use on arrays larger than the available memory capacity.

\sphinxAtStartPar
Footprint is of the form np.ones((frames, y pixels, x pixels)) with the origin in the cente
\begin{quote}\begin{description}
\sphinxlineitem{Parameters}\begin{itemize}
\item {} 
\sphinxAtStartPar
\sphinxstyleliteralstrong{\sphinxupquote{images}} (\sphinxhref{https://numpy.org/doc/1.24/reference/generated/numpy.ndarray.html\#numpy.ndarray}{\sphinxstyleliteralemphasis{\sphinxupquote{numpy.ndarray}}}) \sphinxhyphen{}\sphinxhyphen{} images stack to be filtered

\item {} 
\sphinxAtStartPar
\sphinxstyleliteralstrong{\sphinxupquote{mask}} (\sphinxstyleliteralemphasis{\sphinxupquote{numpy.ndarray = np.ones}}\sphinxstyleliteralemphasis{\sphinxupquote{(}}\sphinxstyleliteralemphasis{\sphinxupquote{(}}\sphinxstyleliteralemphasis{\sphinxupquote{3}}\sphinxstyleliteralemphasis{\sphinxupquote{, }}\sphinxstyleliteralemphasis{\sphinxupquote{3}}\sphinxstyleliteralemphasis{\sphinxupquote{, }}\sphinxstyleliteralemphasis{\sphinxupquote{3}}\sphinxstyleliteralemphasis{\sphinxupquote{)}}\sphinxstyleliteralemphasis{\sphinxupquote{)}}) \sphinxhyphen{}\sphinxhyphen{} mask of the median filter

\item {} 
\sphinxAtStartPar
\sphinxstyleliteralstrong{\sphinxupquote{block\_size}} (\sphinxstyleliteralemphasis{\sphinxupquote{int = 21000}}) \sphinxhyphen{}\sphinxhyphen{} the size of each block. Must fit within memory

\item {} 
\sphinxAtStartPar
\sphinxstyleliteralstrong{\sphinxupquote{block\_buffer}} (\sphinxstyleliteralemphasis{\sphinxupquote{int = 500}}) \sphinxhyphen{}\sphinxhyphen{} the size of the overlapping region between block

\end{itemize}

\sphinxlineitem{Returns}
\sphinxAtStartPar
images: numpy array (frames, y pixels, x pixels)

\sphinxlineitem{Return type}
\sphinxAtStartPar
\sphinxhref{https://numpy.org/doc/1.24/reference/generated/numpy.ndarray.html\#numpy.ndarray}{numpy.ndarray}

\end{description}\end{quote}

\end{fulllineitems}

\index{clean\_image\_stack() (in module CalSciPy.image\_processing)@\spxentry{clean\_image\_stack()}\spxextra{in module CalSciPy.image\_processing}}

\begin{fulllineitems}
\phantomsection\label{\detokenize{CalSciPy.image_processing:CalSciPy.image_processing.clean_image_stack}}
\pysigstartsignatures
\pysiglinewithargsret{\sphinxcode{\sphinxupquote{CalSciPy.image\_processing.}}\sphinxbfcode{\sphinxupquote{clean\_image\_stack}}}{\emph{\DUrole{n}{images}}, \emph{\DUrole{n}{artifact\_length}\DUrole{o}{=}\DUrole{default_value}{1000}}, \emph{\DUrole{n}{stack\_sizes}\DUrole{o}{=}\DUrole{default_value}{7000}}}{}
\pysigstopsignatures
\sphinxAtStartPar
Function to remove initial imaging frames such that any shutter artifact is removing and the resulting tensor
is evenly divisible by the desired stack size
\begin{quote}\begin{description}
\sphinxlineitem{Parameters}\begin{itemize}
\item {} 
\sphinxAtStartPar
\sphinxstyleliteralstrong{\sphinxupquote{images}} (\sphinxhref{https://numpy.org/doc/1.24/reference/generated/numpy.ndarray.html\#numpy.ndarray}{\sphinxstyleliteralemphasis{\sphinxupquote{numpy.ndarray}}}) \sphinxhyphen{}\sphinxhyphen{} images array (frames, y pixels, x pixels)

\item {} 
\sphinxAtStartPar
\sphinxstyleliteralstrong{\sphinxupquote{artifact\_length}} (\sphinxstyleliteralemphasis{\sphinxupquote{int = 1000}}) \sphinxhyphen{}\sphinxhyphen{} number of frames considered artifact

\item {} 
\sphinxAtStartPar
\sphinxstyleliteralstrong{\sphinxupquote{stack\_sizes}} (\sphinxstyleliteralemphasis{\sphinxupquote{int = 7000}}) \sphinxhyphen{}\sphinxhyphen{} number of frames per stack

\end{itemize}

\sphinxlineitem{Returns}
\sphinxAtStartPar
images

\sphinxlineitem{Return type}
\sphinxAtStartPar
\sphinxhref{https://numpy.org/doc/1.24/reference/generated/numpy.ndarray.html\#numpy.ndarray}{numpy.ndarray}

\end{description}\end{quote}

\end{fulllineitems}

\index{fast\_filter\_images() (in module CalSciPy.image\_processing)@\spxentry{fast\_filter\_images()}\spxextra{in module CalSciPy.image\_processing}}

\begin{fulllineitems}
\phantomsection\label{\detokenize{CalSciPy.image_processing:CalSciPy.image_processing.fast_filter_images}}
\pysigstartsignatures
\pysiglinewithargsret{\sphinxcode{\sphinxupquote{CalSciPy.image\_processing.}}\sphinxbfcode{\sphinxupquote{fast\_filter\_images}}}{\emph{\DUrole{n}{images}}, \emph{\DUrole{n}{mask}\DUrole{o}{=}\DUrole{default_value}{array({[}{[}{[}1.0, 1.0, 1.0{]}, {[}1.0, 1.0, 1.0{]}, {[}1.0, 1.0, 1.0{]}{]}, {[}{[}1.0, 1.0, 1.0{]}, {[}1.0, 1.0, 1.0{]}, {[}1.0, 1.0, 1.0{]}{]}, {[}{[}1.0, 1.0, 1.0{]}, {[}1.0, 1.0, 1.0{]}, {[}1.0, 1.0, 1.0{]}{]}{]})}}}{}
\pysigstopsignatures
\sphinxAtStartPar
GPU\sphinxhyphen{}parallelized multidimensional median filter

\sphinxAtStartPar
mask is of the form np.ones((frames, y pixels, x pixels)) with the origin in the center

\sphinxAtStartPar
Requires CuPy
\begin{quote}\begin{description}
\sphinxlineitem{Parameters}\begin{itemize}
\item {} 
\sphinxAtStartPar
\sphinxstyleliteralstrong{\sphinxupquote{images}} (\sphinxhref{https://numpy.org/doc/1.24/reference/generated/numpy.ndarray.html\#numpy.ndarray}{\sphinxstyleliteralemphasis{\sphinxupquote{numpy.ndarray}}}) \sphinxhyphen{}\sphinxhyphen{} image stack to be filtered (frames, y pixels, x pixels)

\item {} 
\sphinxAtStartPar
\sphinxstyleliteralstrong{\sphinxupquote{mask}} (\sphinxstyleliteralemphasis{\sphinxupquote{numpy.ndarray = np.ones}}\sphinxstyleliteralemphasis{\sphinxupquote{(}}\sphinxstyleliteralemphasis{\sphinxupquote{(}}\sphinxstyleliteralemphasis{\sphinxupquote{3}}\sphinxstyleliteralemphasis{\sphinxupquote{, }}\sphinxstyleliteralemphasis{\sphinxupquote{3}}\sphinxstyleliteralemphasis{\sphinxupquote{, }}\sphinxstyleliteralemphasis{\sphinxupquote{3}}\sphinxstyleliteralemphasis{\sphinxupquote{)}}\sphinxstyleliteralemphasis{\sphinxupquote{)}}) \sphinxhyphen{}\sphinxhyphen{} Mask of the median filter

\end{itemize}

\sphinxlineitem{Returns}
\sphinxAtStartPar
filtered\_image (frames, y pixels, x pixels)

\sphinxlineitem{Return type}
\sphinxAtStartPar
Any

\end{description}\end{quote}

\end{fulllineitems}

\index{filter\_images() (in module CalSciPy.image\_processing)@\spxentry{filter\_images()}\spxextra{in module CalSciPy.image\_processing}}

\begin{fulllineitems}
\phantomsection\label{\detokenize{CalSciPy.image_processing:CalSciPy.image_processing.filter_images}}
\pysigstartsignatures
\pysiglinewithargsret{\sphinxcode{\sphinxupquote{CalSciPy.image\_processing.}}\sphinxbfcode{\sphinxupquote{filter\_images}}}{\emph{\DUrole{n}{images}}, \emph{\DUrole{n}{mask}\DUrole{o}{=}\DUrole{default_value}{array({[}{[}{[}1.0, 1.0, 1.0{]}, {[}1.0, 1.0, 1.0{]}, {[}1.0, 1.0, 1.0{]}{]}, {[}{[}1.0, 1.0, 1.0{]}, {[}1.0, 1.0, 1.0{]}, {[}1.0, 1.0, 1.0{]}{]}, {[}{[}1.0, 1.0, 1.0{]}, {[}1.0, 1.0, 1.0{]}, {[}1.0, 1.0, 1.0{]}{]}{]})}}}{}
\pysigstopsignatures
\sphinxAtStartPar
Denoise a tiff stack using a multidimensional median filter

\sphinxAtStartPar
This function simply calls

\sphinxAtStartPar
mask is of the form np.ones((frames, y pixels, x pixels) with the origin in the center
\begin{quote}\begin{description}
\sphinxlineitem{Parameters}\begin{itemize}
\item {} 
\sphinxAtStartPar
\sphinxstyleliteralstrong{\sphinxupquote{images}} (\sphinxhref{https://numpy.org/doc/1.24/reference/generated/numpy.ndarray.html\#numpy.ndarray}{\sphinxstyleliteralemphasis{\sphinxupquote{numpy.ndarray}}}) \sphinxhyphen{}\sphinxhyphen{} images stack to be filtered (frames, y pixels, x pixels)

\item {} 
\sphinxAtStartPar
\sphinxstyleliteralstrong{\sphinxupquote{mask}} (\sphinxstyleliteralemphasis{\sphinxupquote{numpy.ndarray = np.ones}}\sphinxstyleliteralemphasis{\sphinxupquote{(}}\sphinxstyleliteralemphasis{\sphinxupquote{(}}\sphinxstyleliteralemphasis{\sphinxupquote{3}}\sphinxstyleliteralemphasis{\sphinxupquote{, }}\sphinxstyleliteralemphasis{\sphinxupquote{3}}\sphinxstyleliteralemphasis{\sphinxupquote{, }}\sphinxstyleliteralemphasis{\sphinxupquote{3}}\sphinxstyleliteralemphasis{\sphinxupquote{)}}\sphinxstyleliteralemphasis{\sphinxupquote{)}}) \sphinxhyphen{}\sphinxhyphen{} mask of the median filter

\end{itemize}

\sphinxlineitem{Returns}
\sphinxAtStartPar
filtered images (frames, y pixels, x pixels)

\sphinxlineitem{Return type}
\sphinxAtStartPar
\sphinxhref{https://numpy.org/doc/1.24/reference/generated/numpy.ndarray.html\#numpy.ndarray}{numpy.ndarray}

\end{description}\end{quote}

\end{fulllineitems}

\index{grouped\_z\_project() (in module CalSciPy.image\_processing)@\spxentry{grouped\_z\_project()}\spxextra{in module CalSciPy.image\_processing}}

\begin{fulllineitems}
\phantomsection\label{\detokenize{CalSciPy.image_processing:CalSciPy.image_processing.grouped_z_project}}
\pysigstartsignatures
\pysiglinewithargsret{\sphinxcode{\sphinxupquote{CalSciPy.image\_processing.}}\sphinxbfcode{\sphinxupquote{grouped\_z\_project}}}{\emph{\DUrole{n}{images}}, \emph{\DUrole{n}{bin\_size}}, \emph{\DUrole{n}{function=\textless{}function mean\textgreater{}}}}{}
\pysigstopsignatures
\sphinxAtStartPar
Utilize grouped z\sphinxhyphen{}project to downsample data

\sphinxAtStartPar
Downsample example function \sphinxhyphen{}\textgreater{} np.mean
\begin{quote}\begin{description}
\sphinxlineitem{Parameters}\begin{itemize}
\item {} 
\sphinxAtStartPar
\sphinxstyleliteralstrong{\sphinxupquote{images}} (\sphinxhref{https://numpy.org/doc/1.24/reference/generated/numpy.ndarray.html\#numpy.ndarray}{\sphinxstyleliteralemphasis{\sphinxupquote{numpy.ndarray}}}) \sphinxhyphen{}\sphinxhyphen{} A numpy array containing a tiff stack (frames, y pixels, x pixels)

\item {} 
\sphinxAtStartPar
\sphinxstyleliteralstrong{\sphinxupquote{bin\_size}} (\sphinxhref{https://docs.python.org/3/library/stdtypes.html\#tuple}{\sphinxstyleliteralemphasis{\sphinxupquote{tuple}}}\sphinxstyleliteralemphasis{\sphinxupquote{{[}}}\sphinxhref{https://docs.python.org/3/library/functions.html\#int}{\sphinxstyleliteralemphasis{\sphinxupquote{int}}}\sphinxstyleliteralemphasis{\sphinxupquote{, }}\sphinxhref{https://docs.python.org/3/library/functions.html\#int}{\sphinxstyleliteralemphasis{\sphinxupquote{int}}}\sphinxstyleliteralemphasis{\sphinxupquote{, }}\sphinxhref{https://docs.python.org/3/library/functions.html\#int}{\sphinxstyleliteralemphasis{\sphinxupquote{int}}}\sphinxstyleliteralemphasis{\sphinxupquote{{]} or }}\sphinxhref{https://docs.python.org/3/library/stdtypes.html\#tuple}{\sphinxstyleliteralemphasis{\sphinxupquote{tuple}}}\sphinxstyleliteralemphasis{\sphinxupquote{{[}}}\sphinxhref{https://docs.python.org/3/library/functions.html\#int}{\sphinxstyleliteralemphasis{\sphinxupquote{int}}}\sphinxstyleliteralemphasis{\sphinxupquote{{]}}}) \sphinxhyphen{}\sphinxhyphen{} size of each bin passed to downsampling function

\item {} 
\sphinxAtStartPar
\sphinxstyleliteralstrong{\sphinxupquote{function}} (\sphinxstyleliteralemphasis{\sphinxupquote{Callable = np.mean}}) \sphinxhyphen{}\sphinxhyphen{} group\sphinxhyphen{}z projecting function

\end{itemize}

\sphinxlineitem{Returns}
\sphinxAtStartPar
downsampled image (frames, y pixels, x pixels)

\sphinxlineitem{Return type}
\sphinxAtStartPar
\sphinxhref{https://numpy.org/doc/1.24/reference/generated/numpy.ndarray.html\#numpy.ndarray}{numpy.ndarray}

\end{description}\end{quote}

\end{fulllineitems}



\section{Interactive Visuals}
\label{\detokenize{Sub-Packages:interactive-visuals}}\label{\detokenize{Sub-Packages:interactive-visuals-module}}
\begin{DUlineblock}{0em}
\item[] Write me
\item[] Write me
\item[] Write me
\item[] Write me
\end{DUlineblock}


\subsection{Interactive Visuals Methods}
\label{\detokenize{Sub-Packages:interactive-visuals-methods}}
\begin{DUlineblock}{0em}
\item[] Import me
\end{DUlineblock}


\section{Reorganization}
\label{\detokenize{Sub-Packages:reorganization}}\label{\detokenize{Sub-Packages:reorganization-module}}
\begin{DUlineblock}{0em}
\item[] Write me
\item[] Write me
\item[] Write me
\item[] Write me
\end{DUlineblock}


\subsection{Reorganization Methods}
\label{\detokenize{Sub-Packages:reorganization-methods}}
\sphinxstepscope


\subsubsection{CalSciPy.reorganization module}
\label{\detokenize{CalSciPy.reorganization:module-CalSciPy.reorganization}}\label{\detokenize{CalSciPy.reorganization:calscipy-reorganization-module}}\label{\detokenize{CalSciPy.reorganization::doc}}\index{module@\spxentry{module}!CalSciPy.reorganization@\spxentry{CalSciPy.reorganization}}\index{CalSciPy.reorganization@\spxentry{CalSciPy.reorganization}!module@\spxentry{module}}\index{generate\_raster() (in module CalSciPy.reorganization)@\spxentry{generate\_raster()}\spxextra{in module CalSciPy.reorganization}}

\begin{fulllineitems}
\phantomsection\label{\detokenize{CalSciPy.reorganization:CalSciPy.reorganization.generate_raster}}
\pysigstartsignatures
\pysiglinewithargsret{\sphinxcode{\sphinxupquote{CalSciPy.reorganization.}}\sphinxbfcode{\sphinxupquote{generate\_raster}}}{\emph{\DUrole{n}{event\_frames}}, \emph{\DUrole{n}{total\_frames}\DUrole{o}{=}\DUrole{default_value}{None}}}{}
\pysigstopsignatures
\sphinxAtStartPar
Generate raster from lists of frames containing an event (e.g., spikes)
\begin{quote}\begin{description}
\sphinxlineitem{Parameters}\begin{itemize}
\item {} 
\sphinxAtStartPar
\sphinxstyleliteralstrong{\sphinxupquote{event\_frames}} (\sphinxhref{https://docs.python.org/3/library/stdtypes.html\#list}{\sphinxstyleliteralemphasis{\sphinxupquote{list}}}\sphinxstyleliteralemphasis{\sphinxupquote{{[}}}\sphinxhref{https://docs.python.org/3/library/stdtypes.html\#list}{\sphinxstyleliteralemphasis{\sphinxupquote{list}}}\sphinxstyleliteralemphasis{\sphinxupquote{{[}}}\sphinxhref{https://docs.python.org/3/library/functions.html\#int}{\sphinxstyleliteralemphasis{\sphinxupquote{int}}}\sphinxstyleliteralemphasis{\sphinxupquote{{]}}}\sphinxstyleliteralemphasis{\sphinxupquote{{]}}}) \sphinxhyphen{}\sphinxhyphen{} list of event frames (e.g., spike frames)

\item {} 
\sphinxAtStartPar
\sphinxstyleliteralstrong{\sphinxupquote{total\_frames}} (\sphinxstyleliteralemphasis{\sphinxupquote{Optional}}\sphinxstyleliteralemphasis{\sphinxupquote{{[}}}\sphinxhref{https://docs.python.org/3/library/functions.html\#int}{\sphinxstyleliteralemphasis{\sphinxupquote{int}}}\sphinxstyleliteralemphasis{\sphinxupquote{{]} }}\sphinxstyleliteralemphasis{\sphinxupquote{= None}}) \sphinxhyphen{}\sphinxhyphen{} total number of frames

\end{itemize}

\sphinxlineitem{Returns}
\sphinxAtStartPar
event matrix of neurons x total frames

\sphinxlineitem{Return type}
\sphinxAtStartPar
\sphinxhref{https://numpy.org/doc/1.24/reference/generated/numpy.ndarray.html\#numpy.ndarray}{numpy.ndarray}

\end{description}\end{quote}

\end{fulllineitems}

\index{generate\_tensor() (in module CalSciPy.reorganization)@\spxentry{generate\_tensor()}\spxextra{in module CalSciPy.reorganization}}

\begin{fulllineitems}
\phantomsection\label{\detokenize{CalSciPy.reorganization:CalSciPy.reorganization.generate_tensor}}
\pysigstartsignatures
\pysiglinewithargsret{\sphinxcode{\sphinxupquote{CalSciPy.reorganization.}}\sphinxbfcode{\sphinxupquote{generate\_tensor}}}{\emph{\DUrole{n}{traces\_as\_matrix}}, \emph{\DUrole{n}{chunk\_size}}}{}
\pysigstopsignatures
\sphinxAtStartPar
Generates a tensor given chunk / trial indices
\begin{quote}\begin{description}
\sphinxlineitem{Parameters}\begin{itemize}
\item {} 
\sphinxAtStartPar
\sphinxstyleliteralstrong{\sphinxupquote{traces\_as\_matrix}} (\sphinxhref{https://numpy.org/doc/1.24/reference/generated/numpy.ndarray.html\#numpy.ndarray}{\sphinxstyleliteralemphasis{\sphinxupquote{numpy.ndarray}}}) \sphinxhyphen{}\sphinxhyphen{} traces in matrix form (neurons x frames)

\item {} 
\sphinxAtStartPar
\sphinxstyleliteralstrong{\sphinxupquote{chunk\_size}} (\sphinxhref{https://docs.python.org/3/library/functions.html\#int}{\sphinxstyleliteralemphasis{\sphinxupquote{int}}}) \sphinxhyphen{}\sphinxhyphen{} size of each chunk

\end{itemize}

\sphinxlineitem{Returns}
\sphinxAtStartPar
traces\_as\_tensor

\sphinxlineitem{Return type}
\sphinxAtStartPar
\sphinxhref{https://numpy.org/doc/1.24/reference/generated/numpy.ndarray.html\#numpy.ndarray}{numpy.ndarray}

\end{description}\end{quote}

\end{fulllineitems}

\index{merge\_factorized\_matrices() (in module CalSciPy.reorganization)@\spxentry{merge\_factorized\_matrices()}\spxextra{in module CalSciPy.reorganization}}

\begin{fulllineitems}
\phantomsection\label{\detokenize{CalSciPy.reorganization:CalSciPy.reorganization.merge_factorized_matrices}}
\pysigstartsignatures
\pysiglinewithargsret{\sphinxcode{\sphinxupquote{CalSciPy.reorganization.}}\sphinxbfcode{\sphinxupquote{merge\_factorized\_matrices}}}{\emph{\DUrole{n}{factorized\_traces}}, \emph{\DUrole{n}{component}\DUrole{o}{=}\DUrole{default_value}{0}}}{}
\pysigstopsignatures
\sphinxAtStartPar
Concatenate a neuron x chunk or trial array in which each element is a component x frame factorization of the
original trace:
\begin{quote}\begin{description}
\sphinxlineitem{Parameters}\begin{itemize}
\item {} 
\sphinxAtStartPar
\sphinxstyleliteralstrong{\sphinxupquote{factorized\_traces}} (\sphinxhref{https://numpy.org/doc/1.24/reference/generated/numpy.ndarray.html\#numpy.ndarray}{\sphinxstyleliteralemphasis{\sphinxupquote{numpy.ndarray}}}) \sphinxhyphen{}\sphinxhyphen{} neurons x chunks (trial, tiff, etc) containing the neuron\textquotesingle{}s trace factorized
into several components

\item {} 
\sphinxAtStartPar
\sphinxstyleliteralstrong{\sphinxupquote{component}} (\sphinxhref{https://docs.python.org/3/library/functions.html\#int}{\sphinxstyleliteralemphasis{\sphinxupquote{int}}}) \sphinxhyphen{}\sphinxhyphen{} specific component to extract

\end{itemize}

\sphinxlineitem{Returns}
\sphinxAtStartPar
traces of specific component in matrix form

\sphinxlineitem{Return type}
\sphinxAtStartPar
\sphinxhref{https://numpy.org/doc/1.24/reference/generated/numpy.ndarray.html\#numpy.ndarray}{numpy.ndarray}

\end{description}\end{quote}

\end{fulllineitems}

\index{merge\_tensor() (in module CalSciPy.reorganization)@\spxentry{merge\_tensor()}\spxextra{in module CalSciPy.reorganization}}

\begin{fulllineitems}
\phantomsection\label{\detokenize{CalSciPy.reorganization:CalSciPy.reorganization.merge_tensor}}
\pysigstartsignatures
\pysiglinewithargsret{\sphinxcode{\sphinxupquote{CalSciPy.reorganization.}}\sphinxbfcode{\sphinxupquote{merge\_tensor}}}{\emph{\DUrole{n}{traces\_as\_tensor}}}{}
\pysigstopsignatures
\sphinxAtStartPar
Concatenate multiple trials or tiffs into single matrix:
\begin{quote}\begin{description}
\sphinxlineitem{Parameters}
\sphinxAtStartPar
\sphinxstyleliteralstrong{\sphinxupquote{traces\_as\_tensor}} (\sphinxhref{https://numpy.org/doc/1.24/reference/generated/numpy.ndarray.html\#numpy.ndarray}{\sphinxstyleliteralemphasis{\sphinxupquote{numpy.ndarray}}}) \sphinxhyphen{}\sphinxhyphen{} chunk (trial, tiff, etc) x neurons x frames

\sphinxlineitem{Returns}
\sphinxAtStartPar
traces in matrix form

\sphinxlineitem{Return type}
\sphinxAtStartPar
\sphinxhref{https://numpy.org/doc/1.24/reference/generated/numpy.ndarray.html\#numpy.ndarray}{numpy.ndarray}

\end{description}\end{quote}

\end{fulllineitems}



\section{Trace Processing}
\label{\detokenize{Sub-Packages:trace-processing}}\label{\detokenize{Sub-Packages:trace-processing-module}}
\begin{DUlineblock}{0em}
\item[] Write me
\item[] Write me
\item[] Write me
\item[] Write me
\end{DUlineblock}

\sphinxstepscope


\subsection{CalSciPy.trace\_processing module}
\label{\detokenize{CalSciPy.trace_processing:module-CalSciPy.trace_processing}}\label{\detokenize{CalSciPy.trace_processing:calscipy-trace-processing-module}}\label{\detokenize{CalSciPy.trace_processing::doc}}\index{module@\spxentry{module}!CalSciPy.trace\_processing@\spxentry{CalSciPy.trace\_processing}}\index{CalSciPy.trace\_processing@\spxentry{CalSciPy.trace\_processing}!module@\spxentry{module}}\index{calculate\_dfof() (in module CalSciPy.trace\_processing)@\spxentry{calculate\_dfof()}\spxextra{in module CalSciPy.trace\_processing}}

\begin{fulllineitems}
\phantomsection\label{\detokenize{CalSciPy.trace_processing:CalSciPy.trace_processing.calculate_dfof}}
\pysigstartsignatures
\pysiglinewithargsret{\sphinxcode{\sphinxupquote{CalSciPy.trace\_processing.}}\sphinxbfcode{\sphinxupquote{calculate\_dfof}}}{\emph{\DUrole{n}{traces}}, \emph{\DUrole{n}{frame\_rate}\DUrole{o}{=}\DUrole{default_value}{30}}, \emph{\DUrole{n}{in\_place}\DUrole{o}{=}\DUrole{default_value}{False}}, \emph{\DUrole{n}{offset}\DUrole{o}{=}\DUrole{default_value}{0.0}}, \emph{\DUrole{n}{raw}\DUrole{o}{=}\DUrole{default_value}{None}}}{}
\pysigstopsignatures
\sphinxAtStartPar
Calculates Δf/f0 (fold fluorescence over baseline). Baseline is defined as the 5th percentile of the signal
after a 1Hz low\sphinxhyphen{}pass filter using a Hamming window.
\begin{quote}\begin{description}
\sphinxlineitem{Parameters}\begin{itemize}
\item {} 
\sphinxAtStartPar
\sphinxstyleliteralstrong{\sphinxupquote{traces}} (\sphinxhref{https://numpy.org/doc/1.24/reference/generated/numpy.ndarray.html\#numpy.ndarray}{\sphinxstyleliteralemphasis{\sphinxupquote{numpy.ndarray}}}) \sphinxhyphen{}\sphinxhyphen{} matrix of traces in the form of neurons x frames

\item {} 
\sphinxAtStartPar
\sphinxstyleliteralstrong{\sphinxupquote{frame\_rate}} (\sphinxstyleliteralemphasis{\sphinxupquote{float = 30}}) \sphinxhyphen{}\sphinxhyphen{} frame rate of dataset

\item {} 
\sphinxAtStartPar
\sphinxstyleliteralstrong{\sphinxupquote{in\_place}} (\sphinxstyleliteralemphasis{\sphinxupquote{bool = False}}) \sphinxhyphen{}\sphinxhyphen{} boolean indicating whether to perform calculation in\sphinxhyphen{}place

\item {} 
\sphinxAtStartPar
\sphinxstyleliteralstrong{\sphinxupquote{offset}} (\sphinxhref{https://docs.python.org/3/library/functions.html\#float}{\sphinxstyleliteralemphasis{\sphinxupquote{float}}}) \sphinxhyphen{}\sphinxhyphen{} offset added to baseline; useful if traces are non\sphinxhyphen{}negative

\item {} 
\sphinxAtStartPar
\sphinxstyleliteralstrong{\sphinxupquote{raw}} (\sphinxhref{https://numpy.org/doc/1.24/reference/generated/numpy.ndarray.html\#numpy.ndarray}{\sphinxstyleliteralemphasis{\sphinxupquote{numpy.ndarray}}}\sphinxstyleliteralemphasis{\sphinxupquote{ or }}\sphinxstyleliteralemphasis{\sphinxupquote{None}}) \sphinxhyphen{}\sphinxhyphen{} raw dataset used to calculate baseline; useful if traces have been factorized

\end{itemize}

\sphinxlineitem{Returns}
\sphinxAtStartPar
Δf/f0 matrix of n neurons x m samples

\sphinxlineitem{Return type}
\sphinxAtStartPar
\sphinxhref{https://numpy.org/doc/1.24/reference/generated/numpy.ndarray.html\#numpy.ndarray}{numpy.ndarray}

\end{description}\end{quote}

\end{fulllineitems}

\index{calculate\_standardized\_noise() (in module CalSciPy.trace\_processing)@\spxentry{calculate\_standardized\_noise()}\spxextra{in module CalSciPy.trace\_processing}}

\begin{fulllineitems}
\phantomsection\label{\detokenize{CalSciPy.trace_processing:CalSciPy.trace_processing.calculate_standardized_noise}}
\pysigstartsignatures
\pysiglinewithargsret{\sphinxcode{\sphinxupquote{CalSciPy.trace\_processing.}}\sphinxbfcode{\sphinxupquote{calculate\_standardized\_noise}}}{\emph{\DUrole{n}{fold\_fluorescence\_over\_baseline}}, \emph{\DUrole{n}{frame\_rate}\DUrole{o}{=}\DUrole{default_value}{30}}}{}
\pysigstopsignatures\begin{description}
\sphinxlineitem{Calculates a frame\sphinxhyphen{}rate independent standardized noise as defined as:}
\begin{DUlineblock}{0em}
\item[] \(v = \frac{\sigma \frac{\Delta F}F}\sqrt{f}\)
\end{DUlineblock}

\end{description}

\sphinxAtStartPar
It is robust against outliers and approximates the standard deviation of Δf/f0 baseline fluctuations.
For comparison, the more exquisite of the Allen Brain Institute\textquotesingle{}s public datasets are approximately 1*\%Hz\textasciicircum{}(\sphinxhyphen{}1/2)
\begin{quote}\begin{description}
\sphinxlineitem{Parameters}\begin{itemize}
\item {} 
\sphinxAtStartPar
\sphinxstyleliteralstrong{\sphinxupquote{fold\_fluorescence\_over\_baseline}} (\sphinxhref{https://numpy.org/doc/1.24/reference/generated/numpy.ndarray.html\#numpy.ndarray}{\sphinxstyleliteralemphasis{\sphinxupquote{numpy.ndarray}}}) \sphinxhyphen{}\sphinxhyphen{} fold fluorescence over baseline (i.e., Δf/f0)

\item {} 
\sphinxAtStartPar
\sphinxstyleliteralstrong{\sphinxupquote{frame\_rate}} (\sphinxstyleliteralemphasis{\sphinxupquote{float = 30}}) \sphinxhyphen{}\sphinxhyphen{} frame rate of dataset

\end{itemize}

\sphinxlineitem{Returns}
\sphinxAtStartPar
standardized noise (units are  1*\%Hz\textasciicircum{}(\sphinxhyphen{}1/2) )

\sphinxlineitem{Return type}
\sphinxAtStartPar
\sphinxhref{https://numpy.org/doc/1.24/reference/generated/numpy.ndarray.html\#numpy.ndarray}{numpy.ndarray}

\end{description}\end{quote}

\end{fulllineitems}

\index{detrend\_polynomial() (in module CalSciPy.trace\_processing)@\spxentry{detrend\_polynomial()}\spxextra{in module CalSciPy.trace\_processing}}

\begin{fulllineitems}
\phantomsection\label{\detokenize{CalSciPy.trace_processing:CalSciPy.trace_processing.detrend_polynomial}}
\pysigstartsignatures
\pysiglinewithargsret{\sphinxcode{\sphinxupquote{CalSciPy.trace\_processing.}}\sphinxbfcode{\sphinxupquote{detrend\_polynomial}}}{\emph{\DUrole{n}{traces}}, \emph{\DUrole{n}{in\_place}\DUrole{o}{=}\DUrole{default_value}{False}}}{}
\pysigstopsignatures
\sphinxAtStartPar
Detrend traces using a fourth\sphinxhyphen{}order polynomial
\begin{quote}\begin{description}
\sphinxlineitem{Parameters}\begin{itemize}
\item {} 
\sphinxAtStartPar
\sphinxstyleliteralstrong{\sphinxupquote{traces}} (\sphinxhref{https://numpy.org/doc/1.24/reference/generated/numpy.ndarray.html\#numpy.ndarray}{\sphinxstyleliteralemphasis{\sphinxupquote{numpy.ndarray}}}) \sphinxhyphen{}\sphinxhyphen{} matrix of traces in the form of neurons x frames

\item {} 
\sphinxAtStartPar
\sphinxstyleliteralstrong{\sphinxupquote{in\_place}} (\sphinxstyleliteralemphasis{\sphinxupquote{bool = False}}) \sphinxhyphen{}\sphinxhyphen{} boolean indicating whether to perform calculation in\sphinxhyphen{}place

\end{itemize}

\sphinxlineitem{Returns}
\sphinxAtStartPar
detrended traces

\sphinxlineitem{Return type}
\sphinxAtStartPar
\sphinxhref{https://numpy.org/doc/1.24/reference/generated/numpy.ndarray.html\#numpy.ndarray}{numpy.ndarray}

\end{description}\end{quote}

\end{fulllineitems}

\index{smooth\_perona\_malik() (in module CalSciPy.trace\_processing)@\spxentry{smooth\_perona\_malik()}\spxextra{in module CalSciPy.trace\_processing}}

\begin{fulllineitems}
\phantomsection\label{\detokenize{CalSciPy.trace_processing:CalSciPy.trace_processing.smooth_perona_malik}}
\pysigstartsignatures
\pysiglinewithargsret{\sphinxcode{\sphinxupquote{CalSciPy.trace\_processing.}}\sphinxbfcode{\sphinxupquote{smooth\_perona\_malik}}}{\emph{\DUrole{n}{traces}}, \emph{\DUrole{n}{iters}\DUrole{o}{=}\DUrole{default_value}{5}}, \emph{\DUrole{n}{kappa}\DUrole{o}{=}\DUrole{default_value}{100}}, \emph{\DUrole{n}{gamma}\DUrole{o}{=}\DUrole{default_value}{0.15}}}{}
\pysigstopsignatures
\sphinxAtStartPar
Edge\sphinxhyphen{}preserving smoothing using perona malik diffusion
\begin{quote}\begin{description}
\sphinxlineitem{Parameters}\begin{itemize}
\item {} 
\sphinxAtStartPar
\sphinxstyleliteralstrong{\sphinxupquote{traces}} (\sphinxhref{https://numpy.org/doc/1.24/reference/generated/numpy.ndarray.html\#numpy.ndarray}{\sphinxstyleliteralemphasis{\sphinxupquote{numpy.ndarray}}}) \sphinxhyphen{}\sphinxhyphen{} a matrix of neurons x frames

\item {} 
\sphinxAtStartPar
\sphinxstyleliteralstrong{\sphinxupquote{iters}} (\sphinxstyleliteralemphasis{\sphinxupquote{int = 5}}) \sphinxhyphen{}\sphinxhyphen{} number of iterations

\item {} 
\sphinxAtStartPar
\sphinxstyleliteralstrong{\sphinxupquote{kappa}} (\sphinxstyleliteralemphasis{\sphinxupquote{int = 100}}) \sphinxhyphen{}\sphinxhyphen{} diffusivity conductance

\item {} 
\sphinxAtStartPar
\sphinxstyleliteralstrong{\sphinxupquote{gamma}} (\sphinxstyleliteralemphasis{\sphinxupquote{float = 0.15}}) \sphinxhyphen{}\sphinxhyphen{} step size (must be less than 1)

\end{itemize}

\sphinxlineitem{Returns}
\sphinxAtStartPar
smoothed traces

\sphinxlineitem{Return type}
\sphinxAtStartPar
\sphinxhref{https://numpy.org/doc/1.24/reference/generated/numpy.ndarray.html\#numpy.ndarray}{numpy.ndarray}

\end{description}\end{quote}

\end{fulllineitems}



\section{Static Visuals}
\label{\detokenize{Sub-Packages:static-visuals}}\label{\detokenize{Sub-Packages:static-visuals-module}}
\begin{DUlineblock}{0em}
\item[] Write me
\item[] Write me
\item[] Write me
\item[] Write me
\end{DUlineblock}


\subsection{Static Visual Methods}
\label{\detokenize{Sub-Packages:static-visual-methods}}
\begin{DUlineblock}{0em}
\item[] Import me
\end{DUlineblock}


\chapter{Indices and tables}
\label{\detokenize{index:indices-and-tables}}\begin{itemize}
\item {} 
\sphinxAtStartPar
\DUrole{xref,std,std-ref}{genindex}

\item {} 
\sphinxAtStartPar
\DUrole{xref,std,std-ref}{modindex}

\item {} 
\sphinxAtStartPar
\DUrole{xref,std,std-ref}{search}

\end{itemize}


\renewcommand{\indexname}{Python Module Index}
\begin{sphinxtheindex}
\let\bigletter\sphinxstyleindexlettergroup
\bigletter{c}
\item\relax\sphinxstyleindexentry{CalSciPy.bruker}\sphinxstyleindexpageref{CalSciPy.bruker:\detokenize{module-CalSciPy.bruker}}
\item\relax\sphinxstyleindexentry{CalSciPy.event\_processing}\sphinxstyleindexpageref{CalSciPy.event_processing:\detokenize{module-CalSciPy.event_processing}}
\item\relax\sphinxstyleindexentry{CalSciPy.image\_processing}\sphinxstyleindexpageref{CalSciPy.image_processing:\detokenize{module-CalSciPy.image_processing}}
\item\relax\sphinxstyleindexentry{CalSciPy.io\_tools}\sphinxstyleindexpageref{CalSciPy.io_tools:\detokenize{module-CalSciPy.io_tools}}
\item\relax\sphinxstyleindexentry{CalSciPy.reorganization}\sphinxstyleindexpageref{CalSciPy.reorganization:\detokenize{module-CalSciPy.reorganization}}
\item\relax\sphinxstyleindexentry{CalSciPy.trace\_processing}\sphinxstyleindexpageref{CalSciPy.trace_processing:\detokenize{module-CalSciPy.trace_processing}}
\end{sphinxtheindex}

\renewcommand{\indexname}{Index}
\printindex
\end{document}