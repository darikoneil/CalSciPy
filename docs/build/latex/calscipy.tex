%% Generated by Sphinx.
\def\sphinxdocclass{report}
\documentclass[letterpaper,10pt,english]{sphinxmanual}
\ifdefined\pdfpxdimen
   \let\sphinxpxdimen\pdfpxdimen\else\newdimen\sphinxpxdimen
\fi \sphinxpxdimen=.75bp\relax
\ifdefined\pdfimageresolution
    \pdfimageresolution= \numexpr \dimexpr1in\relax/\sphinxpxdimen\relax
\fi
%% let collapsible pdf bookmarks panel have high depth per default
\PassOptionsToPackage{bookmarksdepth=5}{hyperref}

\PassOptionsToPackage{booktabs}{sphinx}
\PassOptionsToPackage{colorrows}{sphinx}

\PassOptionsToPackage{warn}{textcomp}
\usepackage[utf8]{inputenc}
\ifdefined\DeclareUnicodeCharacter
% support both utf8 and utf8x syntaxes
  \ifdefined\DeclareUnicodeCharacterAsOptional
    \def\sphinxDUC#1{\DeclareUnicodeCharacter{"#1}}
  \else
    \let\sphinxDUC\DeclareUnicodeCharacter
  \fi
  \sphinxDUC{00A0}{\nobreakspace}
  \sphinxDUC{2500}{\sphinxunichar{2500}}
  \sphinxDUC{2502}{\sphinxunichar{2502}}
  \sphinxDUC{2514}{\sphinxunichar{2514}}
  \sphinxDUC{251C}{\sphinxunichar{251C}}
  \sphinxDUC{2572}{\textbackslash}
\fi
\usepackage{cmap}
\usepackage[T1]{fontenc}
\usepackage{amsmath,amssymb,amstext}
\usepackage{babel}



\usepackage{tgtermes}
\usepackage{tgheros}
\renewcommand{\ttdefault}{txtt}



\usepackage[Bjarne]{fncychap}
\usepackage{sphinx}

\fvset{fontsize=auto}
\usepackage{geometry}


% Include hyperref last.
\usepackage{hyperref}
% Fix anchor placement for figures with captions.
\usepackage{hypcap}% it must be loaded after hyperref.
% Set up styles of URL: it should be placed after hyperref.
\urlstyle{same}

\addto\captionsenglish{\renewcommand{\contentsname}{Contents:}}

\usepackage{sphinxmessages}
\setcounter{tocdepth}{1}



\title{CalSciPy}
\date{2023}
\release{0.3.1}
\author{Darik A.\@{} O\textquotesingle{}Neil}
\newcommand{\sphinxlogo}{\vbox{}}
\renewcommand{\releasename}{Release}
\makeindex
\begin{document}

\ifdefined\shorthandoff
  \ifnum\catcode`\=\string=\active\shorthandoff{=}\fi
  \ifnum\catcode`\"=\active\shorthandoff{"}\fi
\fi

\pagestyle{empty}
\sphinxmaketitle
\pagestyle{plain}
\sphinxtableofcontents
\pagestyle{normal}
\phantomsection\label{\detokenize{index::doc}}


\sphinxstepscope


\chapter{Introduction}
\label{\detokenize{Introduction:introduction}}\label{\detokenize{Introduction::doc}}
\sphinxAtStartPar
\sphinxstylestrong{CalSciPy} contains a variety of useful methods for handling, processing, and visualizing calcium imaging data.
It’s intended to be a collection of useful, well\sphinxhyphen{}documented functions often used in boilerplate code alongside software
packages such as \sphinxhref{https://github.com/flatironinstitute/CaImAn}{Caiman}, \sphinxhref{https://github.com/losonczylab/sima}{SIMA},
and \sphinxhref{https://github.com/MouseLand/suite2p}{Suite2P}.


\section{Motivation}
\label{\detokenize{Introduction:motivation}}
\sphinxAtStartPar
I noticed I was often re\sphinxhyphen{}writing or copy/pasting a lot of code between environments when working with calcium imaging
data. I started this package  so I don’t have to  so you don’t have to. No more wasting time writing 6 lines to simply
preview your tiff stack, extract a particular channel, or bin some spikes. No more vague exceptions or incomplete
documentation when re\sphinxhyphen{}using a hastily\sphinxhyphen{}made function from 2 months ago. Alongside these time\sphinxhyphen{}savers, I’ve also included
some more non\sphinxhyphen{}trivial methods that are particularly useful.


\section{Limitations}
\label{\detokenize{Introduction:limitations}}
\sphinxAtStartPar
The current distribution for the package is incomplete and partially tested. There may be breaking changes between versions.

\sphinxstepscope


\chapter{Installation}
\label{\detokenize{Installation:installation}}\label{\detokenize{Installation::doc}}

\section{Full Install}
\label{\detokenize{Installation:full-install}}
\sphinxAtStartPar
Enter \sphinxcode{\sphinxupquote{pip install CalSciPy}} in your terminal.


\section{GPU Installation}
\label{\detokenize{Installation:gpu-installation}}
\sphinxAtStartPar
An installation of CuPy \& CUDA are required for gpu\sphinxhyphen{}parallelized functions

\sphinxstepscope


\chapter{CalSciPy.bruker module}
\label{\detokenize{CalSciPy.bruker:module-CalSciPy.bruker}}\label{\detokenize{CalSciPy.bruker:calscipy-bruker-module}}\label{\detokenize{CalSciPy.bruker::doc}}\index{module@\spxentry{module}!CalSciPy.bruker@\spxentry{CalSciPy.bruker}}\index{CalSciPy.bruker@\spxentry{CalSciPy.bruker}!module@\spxentry{module}}\index{align\_data() (in module CalSciPy.bruker)@\spxentry{align\_data()}\spxextra{in module CalSciPy.bruker}}

\begin{fulllineitems}
\phantomsection\label{\detokenize{CalSciPy.bruker:CalSciPy.bruker.align_data}}
\pysigstartsignatures
\pysiglinewithargsret{\sphinxcode{\sphinxupquote{CalSciPy.bruker.}}\sphinxbfcode{\sphinxupquote{align\_data}}}{\emph{\DUrole{n}{analog\_data}\DUrole{p}{:}\DUrole{w}{  }\DUrole{n}{\sphinxhref{https://pandas.pydata.org/docs/reference/api/pandas.DataFrame.html\#pandas.DataFrame}{pandas.core.frame.DataFrame}}}, \emph{\DUrole{n}{frame\_times}\DUrole{p}{:}\DUrole{w}{  }\DUrole{n}{\sphinxhref{https://pandas.pydata.org/docs/reference/api/pandas.DataFrame.html\#pandas.DataFrame}{pandas.core.frame.DataFrame}}}, \emph{\DUrole{n}{fill}\DUrole{p}{:}\DUrole{w}{  }\DUrole{n}{\sphinxhref{https://docs.python.org/3/library/functions.html\#bool}{bool}}\DUrole{w}{  }\DUrole{o}{=}\DUrole{w}{  }\DUrole{default_value}{False}}}{{ $\rightarrow$ \sphinxhref{https://pandas.pydata.org/docs/reference/api/pandas.DataFrame.html\#pandas.DataFrame}{pandas.core.frame.DataFrame}}}
\pysigstopsignatures
\sphinxAtStartPar
Synchronizes analog data \& imaging frames using the timestamp of each frame. Option to generate a second column
in which the frame index is interpolated such that each analog sample matches with an associated frame.
\begin{quote}\begin{description}
\sphinxlineitem{Parameters}\begin{itemize}
\item {} 
\sphinxAtStartPar
\sphinxstyleliteralstrong{\sphinxupquote{analog\_data}} (\sphinxhref{https://pandas.pydata.org/docs/reference/api/pandas.DataFrame.html\#pandas.DataFrame}{\sphinxcode{\sphinxupquote{DataFrame}}}) \textendash{} analog data

\item {} 
\sphinxAtStartPar
\sphinxstyleliteralstrong{\sphinxupquote{frame\_times}} (\sphinxhref{https://pandas.pydata.org/docs/reference/api/pandas.DataFrame.html\#pandas.DataFrame}{\sphinxcode{\sphinxupquote{DataFrame}}}) \textendash{} frame timestamps

\item {} 
\sphinxAtStartPar
\sphinxstyleliteralstrong{\sphinxupquote{fill}} (\sphinxhref{https://docs.python.org/3/library/functions.html\#bool}{\sphinxcode{\sphinxupquote{bool}}}, default: \sphinxcode{\sphinxupquote{False}}) \textendash{} whether to include an interpolated nearest\sphinxhyphen{}neighbor column so each sample has an associated frame

\end{itemize}

\sphinxlineitem{Return type}
\sphinxAtStartPar
\sphinxhref{https://pandas.pydata.org/docs/reference/api/pandas.DataFrame.html\#pandas.DataFrame}{\sphinxcode{\sphinxupquote{DataFrame}}}

\sphinxlineitem{Returns}
\sphinxAtStartPar
a dataframe containing time (index, ms) with aligned columns of voltage recordings/analog data and imaging frame

\end{description}\end{quote}

\end{fulllineitems}

\index{determine\_imaging\_content() (in module CalSciPy.bruker)@\spxentry{determine\_imaging\_content()}\spxextra{in module CalSciPy.bruker}}

\begin{fulllineitems}
\phantomsection\label{\detokenize{CalSciPy.bruker:CalSciPy.bruker.determine_imaging_content}}
\pysigstartsignatures
\pysiglinewithargsret{\sphinxcode{\sphinxupquote{CalSciPy.bruker.}}\sphinxbfcode{\sphinxupquote{determine\_imaging\_content}}}{\emph{\DUrole{n}{folder}\DUrole{p}{:}\DUrole{w}{  }\DUrole{n}{\sphinxhref{https://docs.python.org/3/library/stdtypes.html\#str}{str}\DUrole{w}{  }\DUrole{p}{|}\DUrole{w}{  }\sphinxhref{https://docs.python.org/3/library/pathlib.html\#pathlib.Path}{pathlib.Path}}}}{{ $\rightarrow$ \sphinxhref{https://docs.python.org/3/library/typing.html\#typing.Tuple}{Tuple}\DUrole{p}{{[}}\sphinxhref{https://docs.python.org/3/library/functions.html\#int}{int}\DUrole{p}{,}\DUrole{w}{  }\sphinxhref{https://docs.python.org/3/library/functions.html\#int}{int}\DUrole{p}{,}\DUrole{w}{  }\sphinxhref{https://docs.python.org/3/library/functions.html\#int}{int}\DUrole{p}{,}\DUrole{w}{  }\sphinxhref{https://docs.python.org/3/library/functions.html\#int}{int}\DUrole{p}{,}\DUrole{w}{  }\sphinxhref{https://docs.python.org/3/library/functions.html\#int}{int}\DUrole{p}{{]}}}}
\pysigstopsignatures
\sphinxAtStartPar
This function determines the number of channels and planes within a folder containing .tif files
exported by Bruker’s Prairieview software. It also determines the size of the images (frames, y\sphinxhyphen{}pixels, x\sphinxhyphen{}pixels).
It’s a quick / fast alternative to parsing its respective xml. However, note that the function is dependent on the
naming conventions of PrairieView and will not work on arbitrary folders.
\begin{quote}\begin{description}
\sphinxlineitem{Parameters}
\sphinxAtStartPar
\sphinxstyleliteralstrong{\sphinxupquote{folder}} (\sphinxhref{https://docs.python.org/3/library/typing.html\#typing.Union}{\sphinxcode{\sphinxupquote{Union}}}{[}\sphinxhref{https://docs.python.org/3/library/stdtypes.html\#str}{\sphinxcode{\sphinxupquote{str}}}, \sphinxhref{https://docs.python.org/3/library/pathlib.html\#pathlib.Path}{\sphinxcode{\sphinxupquote{Path}}}{]}) \textendash{} folder containing bruker imaging data

\sphinxlineitem{Return type}
\sphinxAtStartPar
\sphinxhref{https://docs.python.org/3/library/typing.html\#typing.Tuple}{\sphinxcode{\sphinxupquote{Tuple}}}{[}\sphinxhref{https://docs.python.org/3/library/functions.html\#int}{\sphinxcode{\sphinxupquote{int}}}, \sphinxhref{https://docs.python.org/3/library/functions.html\#int}{\sphinxcode{\sphinxupquote{int}}}, \sphinxhref{https://docs.python.org/3/library/functions.html\#int}{\sphinxcode{\sphinxupquote{int}}}, \sphinxhref{https://docs.python.org/3/library/functions.html\#int}{\sphinxcode{\sphinxupquote{int}}}, \sphinxhref{https://docs.python.org/3/library/functions.html\#int}{\sphinxcode{\sphinxupquote{int}}}{]}

\sphinxlineitem{Returns}
\sphinxAtStartPar
channels, planes, frames, height, width

\end{description}\end{quote}

\end{fulllineitems}

\index{extract\_frame\_times() (in module CalSciPy.bruker)@\spxentry{extract\_frame\_times()}\spxextra{in module CalSciPy.bruker}}

\begin{fulllineitems}
\phantomsection\label{\detokenize{CalSciPy.bruker:CalSciPy.bruker.extract_frame_times}}
\pysigstartsignatures
\pysiglinewithargsret{\sphinxcode{\sphinxupquote{CalSciPy.bruker.}}\sphinxbfcode{\sphinxupquote{extract\_frame\_times}}}{\emph{\DUrole{n}{filename}\DUrole{p}{:}\DUrole{w}{  }\DUrole{n}{\sphinxhref{https://docs.python.org/3/library/stdtypes.html\#str}{str}\DUrole{w}{  }\DUrole{p}{|}\DUrole{w}{  }\sphinxhref{https://docs.python.org/3/library/pathlib.html\#pathlib.Path}{pathlib.Path}}}}{{ $\rightarrow$ \sphinxhref{https://pandas.pydata.org/docs/reference/api/pandas.DataFrame.html\#pandas.DataFrame}{pandas.core.frame.DataFrame}}}
\pysigstopsignatures
\sphinxAtStartPar
Function to extract the relative frame times from a PrairieView imaging session’s primary .xml file
\begin{quote}\begin{description}
\sphinxlineitem{Param}
\sphinxAtStartPar
filename

\sphinxlineitem{Return type}
\sphinxAtStartPar
\sphinxhref{https://pandas.pydata.org/docs/reference/api/pandas.DataFrame.html\#pandas.DataFrame}{\sphinxcode{\sphinxupquote{DataFrame}}}

\sphinxlineitem{Returns}
\sphinxAtStartPar
dataframe containing time (index, ms) x imaging frame (\sphinxstyleemphasis{zero\sphinxhyphen{}indexed})

\end{description}\end{quote}

\end{fulllineitems}

\index{generate\_bruker\_naming\_convention() (in module CalSciPy.bruker)@\spxentry{generate\_bruker\_naming\_convention()}\spxextra{in module CalSciPy.bruker}}

\begin{fulllineitems}
\phantomsection\label{\detokenize{CalSciPy.bruker:CalSciPy.bruker.generate_bruker_naming_convention}}
\pysigstartsignatures
\pysiglinewithargsret{\sphinxcode{\sphinxupquote{CalSciPy.bruker.}}\sphinxbfcode{\sphinxupquote{generate\_bruker\_naming\_convention}}}{\emph{\DUrole{n}{channel}\DUrole{p}{:}\DUrole{w}{  }\DUrole{n}{\sphinxhref{https://docs.python.org/3/library/functions.html\#int}{int}}}, \emph{\DUrole{n}{plane}\DUrole{p}{:}\DUrole{w}{  }\DUrole{n}{\sphinxhref{https://docs.python.org/3/library/functions.html\#int}{int}}}, \emph{\DUrole{n}{num\_channels}\DUrole{p}{:}\DUrole{w}{  }\DUrole{n}{\sphinxhref{https://docs.python.org/3/library/functions.html\#int}{int}}\DUrole{w}{  }\DUrole{o}{=}\DUrole{w}{  }\DUrole{default_value}{1}}, \emph{\DUrole{n}{num\_planes}\DUrole{p}{:}\DUrole{w}{  }\DUrole{n}{\sphinxhref{https://docs.python.org/3/library/functions.html\#int}{int}}\DUrole{w}{  }\DUrole{o}{=}\DUrole{w}{  }\DUrole{default_value}{1}}}{{ $\rightarrow$ \sphinxhref{https://docs.python.org/3/library/stdtypes.html\#str}{str}}}
\pysigstopsignatures
\sphinxAtStartPar
Generates the expected bruker naming convention for images collected with an arbitrary number of cycles \& channels

\sphinxAtStartPar
This function expects that the naming convention is \_Cycle00000\_Ch0\_000000.ome.tiff where the channel is
one\sphinxhyphen{}indexed. The 5\sphinxhyphen{}digit cycle id represents the frame if using multiplane imaging and the 6\sphinxhyphen{}digit tag represents
the plane. Otherwise, the 5\sphinxhyphen{}digit tag is static and the 6\sphinxhyphen{}digit tag represents the frame.

\sphinxAtStartPar
Please note that the parameters channel and plane are \sphinxstyleemphasis{zero\sphinxhyphen{}indexed}.
\begin{quote}\begin{description}
\sphinxlineitem{Parameters}\begin{itemize}
\item {} 
\sphinxAtStartPar
\sphinxstyleliteralstrong{\sphinxupquote{channel}} (\sphinxhref{https://docs.python.org/3/library/functions.html\#int}{\sphinxcode{\sphinxupquote{int}}}) \textendash{} channel to produce name for

\item {} 
\sphinxAtStartPar
\sphinxstyleliteralstrong{\sphinxupquote{plane}} (\sphinxhref{https://docs.python.org/3/library/functions.html\#int}{\sphinxcode{\sphinxupquote{int}}}) \textendash{} plane to produce name for

\item {} 
\sphinxAtStartPar
\sphinxstyleliteralstrong{\sphinxupquote{num\_channels}} (\sphinxhref{https://docs.python.org/3/library/functions.html\#int}{\sphinxcode{\sphinxupquote{int}}}, default: \sphinxcode{\sphinxupquote{1}}) \textendash{} number of channels

\item {} 
\sphinxAtStartPar
\sphinxstyleliteralstrong{\sphinxupquote{num\_planes}} (\sphinxhref{https://docs.python.org/3/library/functions.html\#int}{\sphinxcode{\sphinxupquote{int}}}, default: \sphinxcode{\sphinxupquote{1}}) \textendash{} number of planes

\end{itemize}

\sphinxlineitem{Return type}
\sphinxAtStartPar
\sphinxhref{https://docs.python.org/3/library/stdtypes.html\#str}{\sphinxcode{\sphinxupquote{str}}}

\sphinxlineitem{Returns}
\sphinxAtStartPar
proper naming convention

\end{description}\end{quote}

\end{fulllineitems}

\index{load\_bruker\_tifs() (in module CalSciPy.bruker)@\spxentry{load\_bruker\_tifs()}\spxextra{in module CalSciPy.bruker}}

\begin{fulllineitems}
\phantomsection\label{\detokenize{CalSciPy.bruker:CalSciPy.bruker.load_bruker_tifs}}
\pysigstartsignatures
\pysiglinewithargsret{\sphinxcode{\sphinxupquote{CalSciPy.bruker.}}\sphinxbfcode{\sphinxupquote{load\_bruker\_tifs}}}{\emph{\DUrole{n}{folder}\DUrole{p}{:}\DUrole{w}{  }\DUrole{n}{\sphinxhref{https://docs.python.org/3/library/stdtypes.html\#str}{str}\DUrole{w}{  }\DUrole{p}{|}\DUrole{w}{  }\sphinxhref{https://docs.python.org/3/library/pathlib.html\#pathlib.Path}{pathlib.Path}}}, \emph{\DUrole{n}{channel}\DUrole{p}{:}\DUrole{w}{  }\DUrole{n}{\sphinxhref{https://docs.python.org/3/library/functions.html\#int}{int}\DUrole{w}{  }\DUrole{p}{|}\DUrole{w}{  }\sphinxhref{https://docs.python.org/3/library/constants.html\#None}{None}}\DUrole{w}{  }\DUrole{o}{=}\DUrole{w}{  }\DUrole{default_value}{None}}, \emph{\DUrole{n}{plane}\DUrole{p}{:}\DUrole{w}{  }\DUrole{n}{\sphinxhref{https://docs.python.org/3/library/functions.html\#int}{int}\DUrole{w}{  }\DUrole{p}{|}\DUrole{w}{  }\sphinxhref{https://docs.python.org/3/library/constants.html\#None}{None}}\DUrole{w}{  }\DUrole{o}{=}\DUrole{w}{  }\DUrole{default_value}{None}}}{{ $\rightarrow$ \sphinxhref{https://docs.python.org/3/library/typing.html\#typing.Tuple}{Tuple}\DUrole{p}{{[}}\sphinxhref{https://numpy.org/doc/1.24/reference/generated/numpy.ndarray.html\#numpy.ndarray}{numpy.ndarray}\DUrole{p}{{]}}}}
\pysigstopsignatures
\sphinxAtStartPar
This function loads images collected and converted to .tif files by Bruker’s Prairieview software.
If multiple channels or multiple planes exist, each channel and plane combination is loaded to a separate
numpy array. Identification of multiple channels / planes is dependent on {\hyperref[\detokenize{CalSciPy.bruker:CalSciPy.bruker.determine_imaging_content}]{\sphinxcrossref{\sphinxcode{\sphinxupquote{determine\_imaging\_content()}}}}}.
Images are loaded as unsigned 16\sphinxhyphen{}bit (\sphinxcode{\sphinxupquote{numpy.uint16}}), though note that raw bruker files are
natively 12 or 13\sphinxhyphen{}bit.
\begin{quote}\begin{description}
\sphinxlineitem{Parameters}\begin{itemize}
\item {} 
\sphinxAtStartPar
\sphinxstyleliteralstrong{\sphinxupquote{folder}} (\sphinxhref{https://docs.python.org/3/library/typing.html\#typing.Union}{\sphinxcode{\sphinxupquote{Union}}}{[}\sphinxhref{https://docs.python.org/3/library/stdtypes.html\#str}{\sphinxcode{\sphinxupquote{str}}}, \sphinxhref{https://docs.python.org/3/library/pathlib.html\#pathlib.Path}{\sphinxcode{\sphinxupquote{Path}}}{]}) \textendash{} folder containing a sequence of single frame tiff files

\item {} 
\sphinxAtStartPar
\sphinxstyleliteralstrong{\sphinxupquote{channel}} (\sphinxhref{https://docs.python.org/3/library/typing.html\#typing.Optional}{\sphinxcode{\sphinxupquote{Optional}}}{[}\sphinxhref{https://docs.python.org/3/library/functions.html\#int}{\sphinxcode{\sphinxupquote{int}}}{]}, default: \sphinxcode{\sphinxupquote{None}}) \textendash{} specific channel to load from dataset (zero\sphinxhyphen{}indexed)

\item {} 
\sphinxAtStartPar
\sphinxstyleliteralstrong{\sphinxupquote{plane}} (\sphinxhref{https://docs.python.org/3/library/typing.html\#typing.Optional}{\sphinxcode{\sphinxupquote{Optional}}}{[}\sphinxhref{https://docs.python.org/3/library/functions.html\#int}{\sphinxcode{\sphinxupquote{int}}}{]}, default: \sphinxcode{\sphinxupquote{None}}) \textendash{} specific plane to load from dataset (zero\sphinxhyphen{}indexed)

\end{itemize}

\sphinxlineitem{Return type}
\sphinxAtStartPar
\sphinxhref{https://docs.python.org/3/library/typing.html\#typing.Tuple}{\sphinxcode{\sphinxupquote{Tuple}}}{[}\sphinxhref{https://numpy.org/doc/1.24/reference/generated/numpy.ndarray.html\#numpy.ndarray}{\sphinxcode{\sphinxupquote{ndarray}}}{]}

\sphinxlineitem{Returns}
\sphinxAtStartPar
a tuple of numpy arrays (frames, y\sphinxhyphen{}pixels, x\sphinxhyphen{}pixels, \sphinxcode{\sphinxupquote{numpy.uint16}})

\end{description}\end{quote}

\end{fulllineitems}

\index{load\_voltage\_recording() (in module CalSciPy.bruker)@\spxentry{load\_voltage\_recording()}\spxextra{in module CalSciPy.bruker}}

\begin{fulllineitems}
\phantomsection\label{\detokenize{CalSciPy.bruker:CalSciPy.bruker.load_voltage_recording}}
\pysigstartsignatures
\pysiglinewithargsret{\sphinxcode{\sphinxupquote{CalSciPy.bruker.}}\sphinxbfcode{\sphinxupquote{load\_voltage\_recording}}}{\emph{\DUrole{n}{path}\DUrole{p}{:}\DUrole{w}{  }\DUrole{n}{\sphinxhref{https://docs.python.org/3/library/stdtypes.html\#str}{str}\DUrole{w}{  }\DUrole{p}{|}\DUrole{w}{  }\sphinxhref{https://docs.python.org/3/library/pathlib.html\#pathlib.Path}{pathlib.Path}}}}{{ $\rightarrow$ \sphinxhref{https://pandas.pydata.org/docs/reference/api/pandas.DataFrame.html\#pandas.DataFrame}{pandas.core.frame.DataFrame}}}
\pysigstopsignatures
\sphinxAtStartPar
Import bruker analog data from an imaging folder or individual file. By PrairieView naming conventions, these
| files contain “VoltageRecording” in the name.
\begin{quote}\begin{description}
\sphinxlineitem{Parameters}
\sphinxAtStartPar
\sphinxstyleliteralstrong{\sphinxupquote{path}} (\sphinxhref{https://docs.python.org/3/library/typing.html\#typing.Union}{\sphinxcode{\sphinxupquote{Union}}}{[}\sphinxhref{https://docs.python.org/3/library/stdtypes.html\#str}{\sphinxcode{\sphinxupquote{str}}}, \sphinxhref{https://docs.python.org/3/library/pathlib.html\#pathlib.Path}{\sphinxcode{\sphinxupquote{Path}}}{]}) \textendash{} folder or filename containing analog data

\sphinxlineitem{Return type}
\sphinxAtStartPar
\sphinxhref{https://pandas.pydata.org/docs/reference/api/pandas.DataFrame.html\#pandas.DataFrame}{\sphinxcode{\sphinxupquote{DataFrame}}}

\sphinxlineitem{Returns}
\sphinxAtStartPar
dataframe containing time (index, ms) x channel data

\end{description}\end{quote}

\end{fulllineitems}

\index{repackage\_bruker\_tifs() (in module CalSciPy.bruker)@\spxentry{repackage\_bruker\_tifs()}\spxextra{in module CalSciPy.bruker}}

\begin{fulllineitems}
\phantomsection\label{\detokenize{CalSciPy.bruker:CalSciPy.bruker.repackage_bruker_tifs}}
\pysigstartsignatures
\pysiglinewithargsret{\sphinxcode{\sphinxupquote{CalSciPy.bruker.}}\sphinxbfcode{\sphinxupquote{repackage\_bruker\_tifs}}}{\emph{\DUrole{n}{input\_folder}\DUrole{p}{:}\DUrole{w}{  }\DUrole{n}{\sphinxhref{https://docs.python.org/3/library/stdtypes.html\#str}{str}\DUrole{w}{  }\DUrole{p}{|}\DUrole{w}{  }\sphinxhref{https://docs.python.org/3/library/pathlib.html\#pathlib.Path}{pathlib.Path}}}, \emph{\DUrole{n}{output\_folder}\DUrole{p}{:}\DUrole{w}{  }\DUrole{n}{\sphinxhref{https://docs.python.org/3/library/stdtypes.html\#str}{str}\DUrole{w}{  }\DUrole{p}{|}\DUrole{w}{  }\sphinxhref{https://docs.python.org/3/library/pathlib.html\#pathlib.Path}{pathlib.Path}}}, \emph{\DUrole{n}{channel}\DUrole{p}{:}\DUrole{w}{  }\DUrole{n}{\sphinxhref{https://docs.python.org/3/library/functions.html\#int}{int}}\DUrole{w}{  }\DUrole{o}{=}\DUrole{w}{  }\DUrole{default_value}{0}}, \emph{\DUrole{n}{plane}\DUrole{p}{:}\DUrole{w}{  }\DUrole{n}{\sphinxhref{https://docs.python.org/3/library/functions.html\#int}{int}}\DUrole{w}{  }\DUrole{o}{=}\DUrole{w}{  }\DUrole{default_value}{0}}}{{ $\rightarrow$ \sphinxhref{https://docs.python.org/3/library/constants.html\#None}{None}}}
\pysigstopsignatures
\sphinxAtStartPar
This function repackages a folder containing .tif files exported by Bruker’s Prairieview software into a sequence
of \textless{}4 GB .tif stacks. Note that parameters channel and plane are \sphinxstylestrong{zero\sphinxhyphen{}indexed}.
\begin{quote}\begin{description}
\sphinxlineitem{Parameters}\begin{itemize}
\item {} 
\sphinxAtStartPar
\sphinxstyleliteralstrong{\sphinxupquote{input\_folder}} (\sphinxhref{https://docs.python.org/3/library/typing.html\#typing.Union}{\sphinxcode{\sphinxupquote{Union}}}{[}\sphinxhref{https://docs.python.org/3/library/stdtypes.html\#str}{\sphinxcode{\sphinxupquote{str}}}, \sphinxhref{https://docs.python.org/3/library/pathlib.html\#pathlib.Path}{\sphinxcode{\sphinxupquote{Path}}}{]}) \textendash{} folder containing a sequence of single frame .tif files exported by Bruker’s Prairieview

\item {} 
\sphinxAtStartPar
\sphinxstyleliteralstrong{\sphinxupquote{output\_folder}} (\sphinxhref{https://docs.python.org/3/library/typing.html\#typing.Union}{\sphinxcode{\sphinxupquote{Union}}}{[}\sphinxhref{https://docs.python.org/3/library/stdtypes.html\#str}{\sphinxcode{\sphinxupquote{str}}}, \sphinxhref{https://docs.python.org/3/library/pathlib.html\#pathlib.Path}{\sphinxcode{\sphinxupquote{Path}}}{]}) \textendash{} empty folder where .tif stacks will be saved

\item {} 
\sphinxAtStartPar
\sphinxstyleliteralstrong{\sphinxupquote{channel}} (\sphinxhref{https://docs.python.org/3/library/functions.html\#int}{\sphinxcode{\sphinxupquote{int}}}, default: \sphinxcode{\sphinxupquote{0}}) \textendash{} specify channel

\item {} 
\sphinxAtStartPar
\sphinxstyleliteralstrong{\sphinxupquote{plane}} (\sphinxhref{https://docs.python.org/3/library/functions.html\#int}{\sphinxcode{\sphinxupquote{int}}}, default: \sphinxcode{\sphinxupquote{0}}) \textendash{} specify plane

\end{itemize}

\sphinxlineitem{Return type}
\sphinxAtStartPar
\sphinxhref{https://docs.python.org/3/library/constants.html\#None}{\sphinxcode{\sphinxupquote{None}}}

\end{description}\end{quote}

\end{fulllineitems}


\sphinxstepscope


\chapter{CalSciPy.coloring module}
\label{\detokenize{CalSciPy.coloring:module-CalSciPy.coloring}}\label{\detokenize{CalSciPy.coloring:calscipy-coloring-module}}\label{\detokenize{CalSciPy.coloring::doc}}\index{module@\spxentry{module}!CalSciPy.coloring@\spxentry{CalSciPy.coloring}}\index{CalSciPy.coloring@\spxentry{CalSciPy.coloring}!module@\spxentry{module}}\index{BackgroundImage (class in CalSciPy.coloring)@\spxentry{BackgroundImage}\spxextra{class in CalSciPy.coloring}}

\begin{fulllineitems}
\phantomsection\label{\detokenize{CalSciPy.coloring:CalSciPy.coloring.BackgroundImage}}
\pysigstartsignatures
\pysiglinewithargsret{\sphinxbfcode{\sphinxupquote{class\DUrole{w}{  }}}\sphinxcode{\sphinxupquote{CalSciPy.coloring.}}\sphinxbfcode{\sphinxupquote{BackgroundImage}}}{\emph{\DUrole{n}{images}\DUrole{p}{:}\DUrole{w}{  }\DUrole{n}{\sphinxhref{https://numpy.org/doc/1.24/reference/generated/numpy.ndarray.html\#numpy.ndarray}{numpy.ndarray}}}, \emph{\DUrole{n}{style}\DUrole{p}{:}\DUrole{w}{  }\DUrole{n}{\sphinxhref{https://docs.python.org/3/library/functions.html\#int}{int}}\DUrole{w}{  }\DUrole{o}{=}\DUrole{w}{  }\DUrole{default_value}{0}}, \emph{\DUrole{n}{cutoffs}\DUrole{p}{:}\DUrole{w}{  }\DUrole{n}{\sphinxhref{https://docs.python.org/3/library/typing.html\#typing.Tuple}{Tuple}\DUrole{p}{{[}}\sphinxhref{https://docs.python.org/3/library/functions.html\#float}{float}\DUrole{p}{,}\DUrole{w}{  }\sphinxhref{https://docs.python.org/3/library/functions.html\#float}{float}\DUrole{p}{{]}}}\DUrole{w}{  }\DUrole{o}{=}\DUrole{w}{  }\DUrole{default_value}{(0.0, 100.0)}}}{}
\pysigstopsignatures
\sphinxAtStartPar
Bases: \sphinxhref{https://docs.python.org/3/library/functions.html\#object}{\sphinxcode{\sphinxupquote{object}}}
\index{cast() (CalSciPy.coloring.BackgroundImage method)@\spxentry{cast()}\spxextra{CalSciPy.coloring.BackgroundImage method}}

\begin{fulllineitems}
\phantomsection\label{\detokenize{CalSciPy.coloring:CalSciPy.coloring.BackgroundImage.cast}}
\pysigstartsignatures
\pysiglinewithargsret{\sphinxbfcode{\sphinxupquote{cast}}}{}{{ $\rightarrow$ \sphinxhref{https://numpy.org/doc/1.24/reference/generated/numpy.ndarray.html\#numpy.ndarray}{numpy.ndarray}}}
\pysigstopsignatures\begin{quote}\begin{description}
\sphinxlineitem{Return type}
\sphinxAtStartPar
\sphinxhref{https://numpy.org/doc/1.24/reference/generated/numpy.ndarray.html\#numpy.ndarray}{\sphinxcode{\sphinxupquote{ndarray}}}

\end{description}\end{quote}

\end{fulllineitems}

\index{convert() (CalSciPy.coloring.BackgroundImage method)@\spxentry{convert()}\spxextra{CalSciPy.coloring.BackgroundImage method}}

\begin{fulllineitems}
\phantomsection\label{\detokenize{CalSciPy.coloring:CalSciPy.coloring.BackgroundImage.convert}}
\pysigstartsignatures
\pysiglinewithargsret{\sphinxbfcode{\sphinxupquote{convert}}}{}{{ $\rightarrow$ \sphinxhref{https://numpy.org/doc/1.24/reference/generated/numpy.ndarray.html\#numpy.ndarray}{numpy.ndarray}}}
\pysigstopsignatures\begin{quote}\begin{description}
\sphinxlineitem{Return type}
\sphinxAtStartPar
\sphinxhref{https://numpy.org/doc/1.24/reference/generated/numpy.ndarray.html\#numpy.ndarray}{\sphinxcode{\sphinxupquote{ndarray}}}

\end{description}\end{quote}

\end{fulllineitems}

\index{get (CalSciPy.coloring.BackgroundImage property)@\spxentry{get}\spxextra{CalSciPy.coloring.BackgroundImage property}}

\begin{fulllineitems}
\phantomsection\label{\detokenize{CalSciPy.coloring:CalSciPy.coloring.BackgroundImage.get}}
\pysigstartsignatures
\pysigline{\sphinxbfcode{\sphinxupquote{property\DUrole{w}{  }}}\sphinxbfcode{\sphinxupquote{get}}\sphinxbfcode{\sphinxupquote{\DUrole{p}{:}\DUrole{w}{  }\sphinxhref{https://numpy.org/doc/1.24/reference/generated/numpy.ndarray.html\#numpy.ndarray}{ndarray}}}}
\pysigstopsignatures
\end{fulllineitems}

\index{rescale() (CalSciPy.coloring.BackgroundImage method)@\spxentry{rescale()}\spxextra{CalSciPy.coloring.BackgroundImage method}}

\begin{fulllineitems}
\phantomsection\label{\detokenize{CalSciPy.coloring:CalSciPy.coloring.BackgroundImage.rescale}}
\pysigstartsignatures
\pysiglinewithargsret{\sphinxbfcode{\sphinxupquote{rescale}}}{}{{ $\rightarrow$ \sphinxhref{https://numpy.org/doc/1.24/reference/generated/numpy.ndarray.html\#numpy.ndarray}{numpy.ndarray}}}
\pysigstopsignatures\begin{quote}\begin{description}
\sphinxlineitem{Return type}
\sphinxAtStartPar
\sphinxhref{https://numpy.org/doc/1.24/reference/generated/numpy.ndarray.html\#numpy.ndarray}{\sphinxcode{\sphinxupquote{ndarray}}}

\end{description}\end{quote}

\end{fulllineitems}

\index{stylize() (CalSciPy.coloring.BackgroundImage method)@\spxentry{stylize()}\spxextra{CalSciPy.coloring.BackgroundImage method}}

\begin{fulllineitems}
\phantomsection\label{\detokenize{CalSciPy.coloring:CalSciPy.coloring.BackgroundImage.stylize}}
\pysigstartsignatures
\pysiglinewithargsret{\sphinxbfcode{\sphinxupquote{stylize}}}{}{{ $\rightarrow$ \sphinxhref{https://numpy.org/doc/1.24/reference/generated/numpy.ndarray.html\#numpy.ndarray}{numpy.ndarray}}}
\pysigstopsignatures\begin{quote}\begin{description}
\sphinxlineitem{Return type}
\sphinxAtStartPar
\sphinxhref{https://numpy.org/doc/1.24/reference/generated/numpy.ndarray.html\#numpy.ndarray}{\sphinxcode{\sphinxupquote{ndarray}}}

\end{description}\end{quote}

\end{fulllineitems}


\end{fulllineitems}

\index{color\_images() (in module CalSciPy.coloring)@\spxentry{color\_images()}\spxextra{in module CalSciPy.coloring}}

\begin{fulllineitems}
\phantomsection\label{\detokenize{CalSciPy.coloring:CalSciPy.coloring.color_images}}
\pysigstartsignatures
\pysiglinewithargsret{\sphinxcode{\sphinxupquote{CalSciPy.coloring.}}\sphinxbfcode{\sphinxupquote{color\_images}}}{\emph{\DUrole{n}{images}\DUrole{p}{:}\DUrole{w}{  }\DUrole{n}{\sphinxhref{https://numpy.org/doc/1.24/reference/generated/numpy.ndarray.html\#numpy.ndarray}{numpy.ndarray}}}, \emph{\DUrole{n}{rois}\DUrole{p}{:}\DUrole{w}{  }\DUrole{n}{\sphinxhref{https://numpy.org/doc/1.24/reference/generated/numpy.ndarray.html\#numpy.ndarray}{numpy.ndarray}}}}{{ $\rightarrow$ \sphinxhref{https://numpy.org/doc/1.24/reference/generated/numpy.ndarray.html\#numpy.ndarray}{numpy.ndarray}}}
\pysigstopsignatures\begin{quote}\begin{description}
\sphinxlineitem{Return type}
\sphinxAtStartPar
\sphinxhref{https://numpy.org/doc/1.24/reference/generated/numpy.ndarray.html\#numpy.ndarray}{\sphinxcode{\sphinxupquote{ndarray}}}

\end{description}\end{quote}

\end{fulllineitems}

\index{cutoff\_images() (in module CalSciPy.coloring)@\spxentry{cutoff\_images()}\spxextra{in module CalSciPy.coloring}}

\begin{fulllineitems}
\phantomsection\label{\detokenize{CalSciPy.coloring:CalSciPy.coloring.cutoff_images}}
\pysigstartsignatures
\pysiglinewithargsret{\sphinxcode{\sphinxupquote{CalSciPy.coloring.}}\sphinxbfcode{\sphinxupquote{cutoff\_images}}}{\emph{\DUrole{n}{images}\DUrole{p}{:}\DUrole{w}{  }\DUrole{n}{\sphinxhref{https://numpy.org/doc/1.24/reference/generated/numpy.ndarray.html\#numpy.ndarray}{numpy.ndarray}}}, \emph{\DUrole{n}{cutoffs}\DUrole{p}{:}\DUrole{w}{  }\DUrole{n}{\sphinxhref{https://docs.python.org/3/library/typing.html\#typing.Tuple}{Tuple}\DUrole{p}{{[}}\sphinxhref{https://docs.python.org/3/library/functions.html\#float}{float}\DUrole{p}{,}\DUrole{w}{  }\sphinxhref{https://docs.python.org/3/library/functions.html\#float}{float}\DUrole{p}{{]}}}\DUrole{w}{  }\DUrole{o}{=}\DUrole{w}{  }\DUrole{default_value}{(0.0, 100.0)}}, \emph{\DUrole{n}{in\_place}\DUrole{p}{:}\DUrole{w}{  }\DUrole{n}{\sphinxhref{https://docs.python.org/3/library/functions.html\#bool}{bool}}\DUrole{w}{  }\DUrole{o}{=}\DUrole{w}{  }\DUrole{default_value}{True}}}{{ $\rightarrow$ \sphinxhref{https://numpy.org/doc/1.24/reference/generated/numpy.ndarray.html\#numpy.ndarray}{numpy.ndarray}}}
\pysigstopsignatures\begin{quote}\begin{description}
\sphinxlineitem{Return type}
\sphinxAtStartPar
\sphinxhref{https://numpy.org/doc/1.24/reference/generated/numpy.ndarray.html\#numpy.ndarray}{\sphinxcode{\sphinxupquote{ndarray}}}

\end{description}\end{quote}

\end{fulllineitems}

\index{generate\_background\_images() (in module CalSciPy.coloring)@\spxentry{generate\_background\_images()}\spxextra{in module CalSciPy.coloring}}

\begin{fulllineitems}
\phantomsection\label{\detokenize{CalSciPy.coloring:CalSciPy.coloring.generate_background_images}}
\pysigstartsignatures
\pysiglinewithargsret{\sphinxcode{\sphinxupquote{CalSciPy.coloring.}}\sphinxbfcode{\sphinxupquote{generate\_background\_images}}}{\emph{\DUrole{n}{images}\DUrole{p}{:}\DUrole{w}{  }\DUrole{n}{\sphinxhref{https://numpy.org/doc/1.24/reference/generated/numpy.ndarray.html\#numpy.ndarray}{numpy.ndarray}}}, \emph{\DUrole{n}{style}\DUrole{p}{:}\DUrole{w}{  }\DUrole{n}{\sphinxhref{https://docs.python.org/3/library/functions.html\#int}{int}}\DUrole{w}{  }\DUrole{o}{=}\DUrole{w}{  }\DUrole{default_value}{0}}}{{ $\rightarrow$ \sphinxhref{https://numpy.org/doc/1.24/reference/generated/numpy.ndarray.html\#numpy.ndarray}{numpy.ndarray}}}
\pysigstopsignatures
\sphinxAtStartPar
Generates a background image
\begin{quote}\begin{description}
\sphinxlineitem{Parameters}\begin{itemize}
\item {} 
\sphinxAtStartPar
\sphinxstyleliteralstrong{\sphinxupquote{images}} (\sphinxhref{https://numpy.org/doc/1.24/reference/generated/numpy.ndarray.html\#numpy.ndarray}{\sphinxcode{\sphinxupquote{ndarray}}}) \textendash{} 

\item {} 
\sphinxAtStartPar
\sphinxstyleliteralstrong{\sphinxupquote{style}} (\sphinxhref{https://docs.python.org/3/library/functions.html\#int}{\sphinxcode{\sphinxupquote{int}}}, default: \sphinxcode{\sphinxupquote{0}}) \textendash{} 

\end{itemize}

\sphinxlineitem{Return type}
\sphinxAtStartPar
\sphinxhref{https://numpy.org/doc/1.24/reference/generated/numpy.ndarray.html\#numpy.ndarray}{\sphinxcode{\sphinxupquote{ndarray}}}

\sphinxlineitem{Returns}
\sphinxAtStartPar


\end{description}\end{quote}

\end{fulllineitems}

\index{generate\_custom\_colormap() (in module CalSciPy.coloring)@\spxentry{generate\_custom\_colormap()}\spxextra{in module CalSciPy.coloring}}

\begin{fulllineitems}
\phantomsection\label{\detokenize{CalSciPy.coloring:CalSciPy.coloring.generate_custom_colormap}}
\pysigstartsignatures
\pysiglinewithargsret{\sphinxcode{\sphinxupquote{CalSciPy.coloring.}}\sphinxbfcode{\sphinxupquote{generate\_custom\_colormap}}}{\emph{\DUrole{n}{colors}\DUrole{p}{:}\DUrole{w}{  }\DUrole{n}{\sphinxhref{https://docs.python.org/3/library/typing.html\#typing.Tuple}{Tuple}\DUrole{p}{{[}}\sphinxhref{https://docs.python.org/3/library/typing.html\#typing.Tuple}{Tuple}\DUrole{p}{{[}}\sphinxhref{https://docs.python.org/3/library/functions.html\#float}{float}\DUrole{p}{,}\DUrole{w}{  }\sphinxhref{https://docs.python.org/3/library/functions.html\#float}{float}\DUrole{p}{,}\DUrole{w}{  }\sphinxhref{https://docs.python.org/3/library/functions.html\#float}{float}\DUrole{p}{{]}}\DUrole{p}{{]}}}}}{{ $\rightarrow$ matplotlib.colors.Colormap}}
\pysigstopsignatures
\sphinxAtStartPar
Generate a custom linearized colormap from a list of rgb colors
Each color must be in the form a tuple of three floats with each float being between 0.0 \sphinxhyphen{} 1.0.
\begin{quote}\begin{description}
\sphinxlineitem{Parameters}
\sphinxAtStartPar
\sphinxstyleliteralstrong{\sphinxupquote{colors}} (\sphinxhref{https://docs.python.org/3/library/stdtypes.html\#list}{\sphinxstyleliteralemphasis{\sphinxupquote{list}}}\sphinxstyleliteralemphasis{\sphinxupquote{{[}}}\sphinxhref{https://docs.python.org/3/library/stdtypes.html\#tuple}{\sphinxstyleliteralemphasis{\sphinxupquote{tuple}}}\sphinxstyleliteralemphasis{\sphinxupquote{{[}}}\sphinxhref{https://docs.python.org/3/library/functions.html\#float}{\sphinxstyleliteralemphasis{\sphinxupquote{float}}}\sphinxstyleliteralemphasis{\sphinxupquote{, }}\sphinxhref{https://docs.python.org/3/library/functions.html\#float}{\sphinxstyleliteralemphasis{\sphinxupquote{float}}}\sphinxstyleliteralemphasis{\sphinxupquote{, }}\sphinxhref{https://docs.python.org/3/library/functions.html\#float}{\sphinxstyleliteralemphasis{\sphinxupquote{float}}}\sphinxstyleliteralemphasis{\sphinxupquote{{]}}}\sphinxstyleliteralemphasis{\sphinxupquote{{]}}}) \textendash{} a list of colors

\sphinxlineitem{Returns}
\sphinxAtStartPar
a custom linearized colormap

\sphinxlineitem{Return type}
\sphinxAtStartPar
matplotlib.pyplot.cm.colors.Colormap

\end{description}\end{quote}

\end{fulllineitems}

\index{rescale\_images() (in module CalSciPy.coloring)@\spxentry{rescale\_images()}\spxextra{in module CalSciPy.coloring}}

\begin{fulllineitems}
\phantomsection\label{\detokenize{CalSciPy.coloring:CalSciPy.coloring.rescale_images}}
\pysigstartsignatures
\pysiglinewithargsret{\sphinxcode{\sphinxupquote{CalSciPy.coloring.}}\sphinxbfcode{\sphinxupquote{rescale\_images}}}{\emph{\DUrole{n}{images}\DUrole{p}{:}\DUrole{w}{  }\DUrole{n}{\sphinxhref{https://numpy.org/doc/1.24/reference/generated/numpy.ndarray.html\#numpy.ndarray}{numpy.ndarray}}}, \emph{\DUrole{n}{new\_range}\DUrole{p}{:}\DUrole{w}{  }\DUrole{n}{\sphinxhref{https://docs.python.org/3/library/typing.html\#typing.Tuple}{Tuple}\DUrole{p}{{[}}\sphinxhref{https://docs.python.org/3/library/functions.html\#float}{float}\DUrole{p}{,}\DUrole{w}{  }\sphinxhref{https://docs.python.org/3/library/functions.html\#float}{float}\DUrole{p}{{]}}}\DUrole{w}{  }\DUrole{o}{=}\DUrole{w}{  }\DUrole{default_value}{(0.0, 255.0)}}, \emph{\DUrole{n}{in\_place}\DUrole{p}{:}\DUrole{w}{  }\DUrole{n}{\sphinxhref{https://docs.python.org/3/library/functions.html\#bool}{bool}}\DUrole{w}{  }\DUrole{o}{=}\DUrole{w}{  }\DUrole{default_value}{True}}}{{ $\rightarrow$ \sphinxhref{https://numpy.org/doc/1.24/reference/generated/numpy.ndarray.html\#numpy.ndarray}{numpy.ndarray}}}
\pysigstopsignatures\begin{quote}\begin{description}
\sphinxlineitem{Return type}
\sphinxAtStartPar
\sphinxhref{https://numpy.org/doc/1.24/reference/generated/numpy.ndarray.html\#numpy.ndarray}{\sphinxcode{\sphinxupquote{ndarray}}}

\end{description}\end{quote}

\end{fulllineitems}


\sphinxstepscope


\chapter{CalSciPy.event\_processing module}
\label{\detokenize{CalSciPy.event_processing:module-CalSciPy.event_processing}}\label{\detokenize{CalSciPy.event_processing:calscipy-event-processing-module}}\label{\detokenize{CalSciPy.event_processing::doc}}\index{module@\spxentry{module}!CalSciPy.event\_processing@\spxentry{CalSciPy.event\_processing}}\index{CalSciPy.event\_processing@\spxentry{CalSciPy.event\_processing}!module@\spxentry{module}}\index{calculate\_firing\_rates() (in module CalSciPy.event\_processing)@\spxentry{calculate\_firing\_rates()}\spxextra{in module CalSciPy.event\_processing}}

\begin{fulllineitems}
\phantomsection\label{\detokenize{CalSciPy.event_processing:CalSciPy.event_processing.calculate_firing_rates}}
\pysigstartsignatures
\pysiglinewithargsret{\sphinxcode{\sphinxupquote{CalSciPy.event\_processing.}}\sphinxbfcode{\sphinxupquote{calculate\_firing\_rates}}}{\emph{\DUrole{n}{spike\_probability\_matrix}\DUrole{p}{:}\DUrole{w}{  }\DUrole{n}{\sphinxhref{https://numpy.org/doc/1.24/reference/generated/numpy.ndarray.html\#numpy.ndarray}{numpy.ndarray}}}, \emph{\DUrole{n}{frame\_rate}\DUrole{p}{:}\DUrole{w}{  }\DUrole{n}{\sphinxhref{https://docs.python.org/3/library/functions.html\#float}{float}}\DUrole{w}{  }\DUrole{o}{=}\DUrole{w}{  }\DUrole{default_value}{30.0}}, \emph{\DUrole{n}{in\_place}\DUrole{p}{:}\DUrole{w}{  }\DUrole{n}{\sphinxhref{https://docs.python.org/3/library/functions.html\#bool}{bool}}\DUrole{w}{  }\DUrole{o}{=}\DUrole{w}{  }\DUrole{default_value}{False}}}{{ $\rightarrow$ \sphinxhref{https://numpy.org/doc/1.24/reference/generated/numpy.ndarray.html\#numpy.ndarray}{numpy.ndarray}}}
\pysigstopsignatures
\sphinxAtStartPar
Calculate firing rates
\begin{quote}\begin{description}
\sphinxlineitem{Parameters}\begin{itemize}
\item {} 
\sphinxAtStartPar
\sphinxstyleliteralstrong{\sphinxupquote{spike\_probability\_matrix}} (\sphinxhref{https://numpy.org/doc/1.24/reference/generated/numpy.ndarray.html\#numpy.ndarray}{\sphinxcode{\sphinxupquote{ndarray}}}) \textendash{} matrix of n neuron x m samples where each element is the probability of a spike

\item {} 
\sphinxAtStartPar
\sphinxstyleliteralstrong{\sphinxupquote{frame\_rate}} (\sphinxhref{https://docs.python.org/3/library/functions.html\#float}{\sphinxcode{\sphinxupquote{float}}}, default: \sphinxcode{\sphinxupquote{30.0}}) \textendash{} frame rate of dataset

\item {} 
\sphinxAtStartPar
\sphinxstyleliteralstrong{\sphinxupquote{in\_place}} (\sphinxhref{https://docs.python.org/3/library/functions.html\#bool}{\sphinxcode{\sphinxupquote{bool}}}, default: \sphinxcode{\sphinxupquote{False}}) \textendash{} boolean indicating whether to perform calculation in\sphinxhyphen{}place

\end{itemize}

\sphinxlineitem{Return type}
\sphinxAtStartPar
\sphinxhref{https://numpy.org/doc/1.24/reference/generated/numpy.ndarray.html\#numpy.ndarray}{\sphinxcode{\sphinxupquote{ndarray}}}

\sphinxlineitem{Returns}
\sphinxAtStartPar
firing matrix of n neurons x m samples where each element is a binary indicating presence of spike event

\end{description}\end{quote}

\end{fulllineitems}

\index{calculate\_mean\_firing\_rates() (in module CalSciPy.event\_processing)@\spxentry{calculate\_mean\_firing\_rates()}\spxextra{in module CalSciPy.event\_processing}}

\begin{fulllineitems}
\phantomsection\label{\detokenize{CalSciPy.event_processing:CalSciPy.event_processing.calculate_mean_firing_rates}}
\pysigstartsignatures
\pysiglinewithargsret{\sphinxcode{\sphinxupquote{CalSciPy.event\_processing.}}\sphinxbfcode{\sphinxupquote{calculate\_mean\_firing\_rates}}}{\emph{\DUrole{n}{firing\_matrix}\DUrole{p}{:}\DUrole{w}{  }\DUrole{n}{\sphinxhref{https://numpy.org/doc/1.24/reference/generated/numpy.ndarray.html\#numpy.ndarray}{numpy.ndarray}}}}{{ $\rightarrow$ \sphinxhref{https://numpy.org/doc/1.24/reference/generated/numpy.ndarray.html\#numpy.ndarray}{numpy.ndarray}}}
\pysigstopsignatures
\sphinxAtStartPar
Calculate mean firing rate
\begin{quote}\begin{description}
\sphinxlineitem{Parameters}
\sphinxAtStartPar
\sphinxstyleliteralstrong{\sphinxupquote{firing\_matrix}} (\sphinxhref{https://numpy.org/doc/1.24/reference/generated/numpy.ndarray.html\#numpy.ndarray}{\sphinxcode{\sphinxupquote{ndarray}}}) \textendash{} matrix of n neuron x m samples where each element is either a spike or an

\end{description}\end{quote}

\begin{DUlineblock}{0em}
\item[] instantaneous firing rate
\end{DUlineblock}
\begin{quote}\begin{description}
\sphinxlineitem{Return type}
\sphinxAtStartPar
\sphinxhref{https://numpy.org/doc/1.24/reference/generated/numpy.ndarray.html\#numpy.ndarray}{\sphinxcode{\sphinxupquote{ndarray}}}

\sphinxlineitem{Returns}
\sphinxAtStartPar
1\sphinxhyphen{}D vector of mean firing rates

\end{description}\end{quote}

\end{fulllineitems}

\index{collect\_waveforms() (in module CalSciPy.event\_processing)@\spxentry{collect\_waveforms()}\spxextra{in module CalSciPy.event\_processing}}

\begin{fulllineitems}
\phantomsection\label{\detokenize{CalSciPy.event_processing:CalSciPy.event_processing.collect_waveforms}}
\pysigstartsignatures
\pysiglinewithargsret{\sphinxcode{\sphinxupquote{CalSciPy.event\_processing.}}\sphinxbfcode{\sphinxupquote{collect\_waveforms}}}{\emph{\DUrole{n}{traces}\DUrole{p}{:}\DUrole{w}{  }\DUrole{n}{\sphinxhref{https://numpy.org/doc/1.24/reference/generated/numpy.ndarray.html\#numpy.ndarray}{numpy.ndarray}}}, \emph{\DUrole{n}{event\_indices}\DUrole{p}{:}\DUrole{w}{  }\DUrole{n}{\sphinxhref{https://docs.python.org/3/library/typing.html\#typing.Iterable}{Iterable}\DUrole{p}{{[}}\sphinxhref{https://docs.python.org/3/library/typing.html\#typing.Iterable}{Iterable}\DUrole{p}{{[}}\sphinxhref{https://docs.python.org/3/library/functions.html\#int}{int}\DUrole{p}{{]}}\DUrole{p}{{]}}}}, \emph{\DUrole{n}{pre}\DUrole{p}{:}\DUrole{w}{  }\DUrole{n}{\sphinxhref{https://docs.python.org/3/library/functions.html\#int}{int}}\DUrole{w}{  }\DUrole{o}{=}\DUrole{w}{  }\DUrole{default_value}{150}}, \emph{\DUrole{n}{post}\DUrole{p}{:}\DUrole{w}{  }\DUrole{n}{\sphinxhref{https://docs.python.org/3/library/functions.html\#int}{int}}\DUrole{w}{  }\DUrole{o}{=}\DUrole{w}{  }\DUrole{default_value}{450}}}{{ $\rightarrow$ \sphinxhref{https://docs.python.org/3/library/typing.html\#typing.Tuple}{Tuple}\DUrole{p}{{[}}\sphinxhref{https://numpy.org/doc/1.24/reference/generated/numpy.ndarray.html\#numpy.ndarray}{numpy.ndarray}\DUrole{p}{{]}}}}
\pysigstopsignatures
\sphinxAtStartPar
Collect waveforms for each event
\begin{quote}\begin{description}
\sphinxlineitem{Parameters}\begin{itemize}
\item {} 
\sphinxAtStartPar
\sphinxstyleliteralstrong{\sphinxupquote{traces}} (\sphinxhref{https://numpy.org/doc/1.24/reference/generated/numpy.ndarray.html\#numpy.ndarray}{\sphinxcode{\sphinxupquote{ndarray}}}) \textendash{} a matrix of M neurons x N samples

\item {} 
\sphinxAtStartPar
\sphinxstyleliteralstrong{\sphinxupquote{event\_indices}} (\sphinxhref{https://docs.python.org/3/library/typing.html\#typing.Iterable}{\sphinxcode{\sphinxupquote{Iterable}}}{[}\sphinxhref{https://docs.python.org/3/library/typing.html\#typing.Iterable}{\sphinxcode{\sphinxupquote{Iterable}}}{[}\sphinxhref{https://docs.python.org/3/library/functions.html\#int}{\sphinxcode{\sphinxupquote{int}}}{]}{]}) \textendash{} a list of events

\item {} 
\sphinxAtStartPar
\sphinxstyleliteralstrong{\sphinxupquote{pre}} (\sphinxhref{https://docs.python.org/3/library/functions.html\#int}{\sphinxcode{\sphinxupquote{int}}}, default: \sphinxcode{\sphinxupquote{150}}) \textendash{} number of pre\sphinxhyphen{}event frames

\item {} 
\sphinxAtStartPar
\sphinxstyleliteralstrong{\sphinxupquote{post}} (\sphinxhref{https://docs.python.org/3/library/functions.html\#int}{\sphinxcode{\sphinxupquote{int}}}, default: \sphinxcode{\sphinxupquote{450}}) \textendash{} number of post\sphinxhyphen{}event frames

\end{itemize}

\sphinxlineitem{Return type}
\sphinxAtStartPar
\sphinxhref{https://docs.python.org/3/library/typing.html\#typing.Tuple}{\sphinxcode{\sphinxupquote{Tuple}}}{[}\sphinxhref{https://numpy.org/doc/1.24/reference/generated/numpy.ndarray.html\#numpy.ndarray}{\sphinxcode{\sphinxupquote{ndarray}}}{]}

\sphinxlineitem{Returns}
\sphinxAtStartPar
a matrix of M events x N samples

\end{description}\end{quote}

\end{fulllineitems}

\index{convert\_tau() (in module CalSciPy.event\_processing)@\spxentry{convert\_tau()}\spxextra{in module CalSciPy.event\_processing}}

\begin{fulllineitems}
\phantomsection\label{\detokenize{CalSciPy.event_processing:CalSciPy.event_processing.convert_tau}}
\pysigstartsignatures
\pysiglinewithargsret{\sphinxcode{\sphinxupquote{CalSciPy.event\_processing.}}\sphinxbfcode{\sphinxupquote{convert\_tau}}}{\emph{\DUrole{n}{tau}\DUrole{p}{:}\DUrole{w}{  }\DUrole{n}{\sphinxhref{https://docs.python.org/3/library/functions.html\#float}{float}}}, \emph{\DUrole{n}{dt}\DUrole{p}{:}\DUrole{w}{  }\DUrole{n}{\sphinxhref{https://docs.python.org/3/library/functions.html\#float}{float}}}}{{ $\rightarrow$ \sphinxhref{https://docs.python.org/3/library/functions.html\#float}{float}}}
\pysigstopsignatures
\sphinxAtStartPar
Converts a discrete tau to a continuous tau
\begin{quote}\begin{description}
\sphinxlineitem{Parameters}\begin{itemize}
\item {} 
\sphinxAtStartPar
\sphinxstyleliteralstrong{\sphinxupquote{tau}} (\sphinxhref{https://docs.python.org/3/library/functions.html\#float}{\sphinxcode{\sphinxupquote{float}}}) \textendash{} decay constant

\item {} 
\sphinxAtStartPar
\sphinxstyleliteralstrong{\sphinxupquote{dt}} (\sphinxhref{https://docs.python.org/3/library/functions.html\#float}{\sphinxcode{\sphinxupquote{float}}}) \textendash{} time step (s)

\end{itemize}

\sphinxlineitem{Return type}
\sphinxAtStartPar
\sphinxhref{https://docs.python.org/3/library/functions.html\#float}{\sphinxcode{\sphinxupquote{float}}}

\sphinxlineitem{Returns}
\sphinxAtStartPar
continuous tau (s)

\end{description}\end{quote}

\end{fulllineitems}

\index{get\_event\_onset\_intensities() (in module CalSciPy.event\_processing)@\spxentry{get\_event\_onset\_intensities()}\spxextra{in module CalSciPy.event\_processing}}

\begin{fulllineitems}
\phantomsection\label{\detokenize{CalSciPy.event_processing:CalSciPy.event_processing.get_event_onset_intensities}}
\pysigstartsignatures
\pysiglinewithargsret{\sphinxcode{\sphinxupquote{CalSciPy.event\_processing.}}\sphinxbfcode{\sphinxupquote{get\_event\_onset\_intensities}}}{\emph{\DUrole{n}{traces}\DUrole{p}{:}\DUrole{w}{  }\DUrole{n}{\sphinxhref{https://numpy.org/doc/1.24/reference/generated/numpy.ndarray.html\#numpy.ndarray}{numpy.ndarray}}}, \emph{\DUrole{n}{event\_indices}\DUrole{p}{:}\DUrole{w}{  }\DUrole{n}{\sphinxhref{https://docs.python.org/3/library/typing.html\#typing.Iterable}{Iterable}\DUrole{p}{{[}}\sphinxhref{https://docs.python.org/3/library/typing.html\#typing.Iterable}{Iterable}\DUrole{p}{{[}}\sphinxhref{https://docs.python.org/3/library/functions.html\#int}{int}\DUrole{p}{{]}}\DUrole{p}{{]}}}}}{{ $\rightarrow$ \sphinxhref{https://docs.python.org/3/library/typing.html\#typing.Tuple}{Tuple}\DUrole{p}{{[}}\sphinxhref{https://numpy.org/doc/1.24/reference/generated/numpy.ndarray.html\#numpy.ndarray}{numpy.ndarray}\DUrole{p}{{]}}}}
\pysigstopsignatures
\sphinxAtStartPar
Retrieve the signal intensity at event onset for each neuron in the event indices
\begin{quote}\begin{description}
\sphinxlineitem{Parameters}\begin{itemize}
\item {} 
\sphinxAtStartPar
\sphinxstyleliteralstrong{\sphinxupquote{traces}} (\sphinxhref{https://numpy.org/doc/1.24/reference/generated/numpy.ndarray.html\#numpy.ndarray}{\sphinxcode{\sphinxupquote{ndarray}}}) \textendash{} An M neuron by N sample matrix

\item {} 
\sphinxAtStartPar
\sphinxstyleliteralstrong{\sphinxupquote{event\_indices}} (\sphinxhref{https://docs.python.org/3/library/typing.html\#typing.Iterable}{\sphinxcode{\sphinxupquote{Iterable}}}{[}\sphinxhref{https://docs.python.org/3/library/typing.html\#typing.Iterable}{\sphinxcode{\sphinxupquote{Iterable}}}{[}\sphinxhref{https://docs.python.org/3/library/functions.html\#int}{\sphinxcode{\sphinxupquote{int}}}{]}{]}) \textendash{} An iterable of length M containing a sequence with a duration for each event

\end{itemize}

\sphinxlineitem{Return type}
\sphinxAtStartPar
\sphinxhref{https://docs.python.org/3/library/typing.html\#typing.Tuple}{\sphinxcode{\sphinxupquote{Tuple}}}{[}\sphinxhref{https://numpy.org/doc/1.24/reference/generated/numpy.ndarray.html\#numpy.ndarray}{\sphinxcode{\sphinxupquote{ndarray}}}{]}

\sphinxlineitem{Returns}
\sphinxAtStartPar
An iterable of length M neurons containing the onset intensities for each event in the sequence

\end{description}\end{quote}

\end{fulllineitems}

\index{get\_inter\_event\_intervals() (in module CalSciPy.event\_processing)@\spxentry{get\_inter\_event\_intervals()}\spxextra{in module CalSciPy.event\_processing}}

\begin{fulllineitems}
\phantomsection\label{\detokenize{CalSciPy.event_processing:CalSciPy.event_processing.get_inter_event_intervals}}
\pysigstartsignatures
\pysiglinewithargsret{\sphinxcode{\sphinxupquote{CalSciPy.event\_processing.}}\sphinxbfcode{\sphinxupquote{get\_inter\_event\_intervals}}}{\emph{\DUrole{n}{event\_indices}\DUrole{p}{:}\DUrole{w}{  }\DUrole{n}{\sphinxhref{https://docs.python.org/3/library/typing.html\#typing.Iterable}{Iterable}\DUrole{p}{{[}}\sphinxhref{https://docs.python.org/3/library/typing.html\#typing.Iterable}{Iterable}\DUrole{p}{{[}}\sphinxhref{https://docs.python.org/3/library/functions.html\#int}{int}\DUrole{p}{{]}}\DUrole{p}{{]}}}}, \emph{\DUrole{n}{frame\_rate}\DUrole{p}{:}\DUrole{w}{  }\DUrole{n}{\sphinxhref{https://docs.python.org/3/library/functions.html\#float}{float}}\DUrole{w}{  }\DUrole{o}{=}\DUrole{w}{  }\DUrole{default_value}{30.0}}}{{ $\rightarrow$ \sphinxhref{https://docs.python.org/3/library/typing.html\#typing.Tuple}{Tuple}\DUrole{p}{{[}}\sphinxhref{https://numpy.org/doc/1.24/reference/generated/numpy.ndarray.html\#numpy.ndarray}{numpy.ndarray}\DUrole{p}{{]}}}}
\pysigstopsignatures
\sphinxAtStartPar
Calculate the inter event intervals for each neuron in the event indices
\begin{quote}\begin{description}
\sphinxlineitem{Parameters}\begin{itemize}
\item {} 
\sphinxAtStartPar
\sphinxstyleliteralstrong{\sphinxupquote{event\_indices}} (\sphinxhref{https://docs.python.org/3/library/typing.html\#typing.Iterable}{\sphinxcode{\sphinxupquote{Iterable}}}{[}\sphinxhref{https://docs.python.org/3/library/typing.html\#typing.Iterable}{\sphinxcode{\sphinxupquote{Iterable}}}{[}\sphinxhref{https://docs.python.org/3/library/functions.html\#int}{\sphinxcode{\sphinxupquote{int}}}{]}{]}) \textendash{} An iterable of length M containing a sequence with a duration for each event

\item {} 
\sphinxAtStartPar
\sphinxstyleliteralstrong{\sphinxupquote{frame\_rate}} (\sphinxhref{https://docs.python.org/3/library/functions.html\#float}{\sphinxcode{\sphinxupquote{float}}}, default: \sphinxcode{\sphinxupquote{30.0}}) \textendash{} frame\_rate for trace matrix

\end{itemize}

\sphinxlineitem{Return type}
\sphinxAtStartPar
\sphinxhref{https://docs.python.org/3/library/typing.html\#typing.Tuple}{\sphinxcode{\sphinxupquote{Tuple}}}{[}\sphinxhref{https://numpy.org/doc/1.24/reference/generated/numpy.ndarray.html\#numpy.ndarray}{\sphinxcode{\sphinxupquote{ndarray}}}{]}

\sphinxlineitem{Returns}
\sphinxAtStartPar
An iterable of length M neurons containing the inter\sphinxhyphen{}event intervals for each event in the sequence

\end{description}\end{quote}

\end{fulllineitems}

\index{get\_num\_events() (in module CalSciPy.event\_processing)@\spxentry{get\_num\_events()}\spxextra{in module CalSciPy.event\_processing}}

\begin{fulllineitems}
\phantomsection\label{\detokenize{CalSciPy.event_processing:CalSciPy.event_processing.get_num_events}}
\pysigstartsignatures
\pysiglinewithargsret{\sphinxcode{\sphinxupquote{CalSciPy.event\_processing.}}\sphinxbfcode{\sphinxupquote{get\_num\_events}}}{\emph{\DUrole{n}{event\_indices}\DUrole{p}{:}\DUrole{w}{  }\DUrole{n}{\sphinxhref{https://docs.python.org/3/library/typing.html\#typing.Iterable}{Iterable}\DUrole{p}{{[}}\sphinxhref{https://docs.python.org/3/library/typing.html\#typing.Iterable}{Iterable}\DUrole{p}{{[}}\sphinxhref{https://docs.python.org/3/library/functions.html\#int}{int}\DUrole{p}{{]}}\DUrole{p}{{]}}}}}{{ $\rightarrow$ \sphinxhref{https://numpy.org/doc/1.24/reference/generated/numpy.ndarray.html\#numpy.ndarray}{numpy.ndarray}}}
\pysigstopsignatures
\sphinxAtStartPar
Determines the number of events for each neuron in the event indices
\begin{quote}\begin{description}
\sphinxlineitem{Parameters}
\sphinxAtStartPar
\sphinxstyleliteralstrong{\sphinxupquote{event\_indices}} (\sphinxhref{https://docs.python.org/3/library/typing.html\#typing.Iterable}{\sphinxcode{\sphinxupquote{Iterable}}}{[}\sphinxhref{https://docs.python.org/3/library/typing.html\#typing.Iterable}{\sphinxcode{\sphinxupquote{Iterable}}}{[}\sphinxhref{https://docs.python.org/3/library/functions.html\#int}{\sphinxcode{\sphinxupquote{int}}}{]}{]}) \textendash{} An iterable of length M neurons containing a sequence with a duration for each event

\sphinxlineitem{Return type}
\sphinxAtStartPar
\sphinxhref{https://numpy.org/doc/1.24/reference/generated/numpy.ndarray.html\#numpy.ndarray}{\sphinxcode{\sphinxupquote{ndarray}}}

\sphinxlineitem{Returns}
\sphinxAtStartPar
A 1\sphinxhyphen{}D vector of length M neurons containing the number of events for each neuron

\end{description}\end{quote}

\end{fulllineitems}

\index{identify\_events() (in module CalSciPy.event\_processing)@\spxentry{identify\_events()}\spxextra{in module CalSciPy.event\_processing}}

\begin{fulllineitems}
\phantomsection\label{\detokenize{CalSciPy.event_processing:CalSciPy.event_processing.identify_events}}
\pysigstartsignatures
\pysiglinewithargsret{\sphinxcode{\sphinxupquote{CalSciPy.event\_processing.}}\sphinxbfcode{\sphinxupquote{identify\_events}}}{\emph{\DUrole{n}{traces}\DUrole{p}{:}\DUrole{w}{  }\DUrole{n}{\sphinxhref{https://numpy.org/doc/1.24/reference/generated/numpy.ndarray.html\#numpy.ndarray}{numpy.ndarray}}}, \emph{\DUrole{n}{timeout}\DUrole{p}{:}\DUrole{w}{  }\DUrole{n}{\sphinxhref{https://docs.python.org/3/library/functions.html\#int}{int}}\DUrole{w}{  }\DUrole{o}{=}\DUrole{w}{  }\DUrole{default_value}{15}}, \emph{\DUrole{n}{frame\_rate}\DUrole{p}{:}\DUrole{w}{  }\DUrole{n}{\sphinxhref{https://docs.python.org/3/library/functions.html\#float}{float}}\DUrole{w}{  }\DUrole{o}{=}\DUrole{w}{  }\DUrole{default_value}{30.0}}, \emph{\DUrole{n}{smooth}\DUrole{p}{:}\DUrole{w}{  }\DUrole{n}{\sphinxhref{https://docs.python.org/3/library/functions.html\#bool}{bool}}\DUrole{w}{  }\DUrole{o}{=}\DUrole{w}{  }\DUrole{default_value}{True}}, \emph{\DUrole{n}{force\_nonneg}\DUrole{p}{:}\DUrole{w}{  }\DUrole{n}{\sphinxhref{https://docs.python.org/3/library/functions.html\#bool}{bool}}\DUrole{w}{  }\DUrole{o}{=}\DUrole{w}{  }\DUrole{default_value}{True}}}{{ $\rightarrow$ \sphinxhref{https://docs.python.org/3/library/typing.html\#typing.Tuple}{Tuple}\DUrole{p}{{[}}\sphinxhref{https://docs.python.org/3/library/typing.html\#typing.List}{List}\DUrole{p}{{[}}\sphinxhref{https://docs.python.org/3/library/functions.html\#int}{int}\DUrole{p}{{]}}\DUrole{p}{{]}}}}
\pysigstopsignatures
\sphinxAtStartPar
Identify event onset for each neuron using the smoothed, non\sphinxhyphen{}negative first\sphinxhyphen{}time derivative. The threshold for noise
is considered 1/2th the standard deviation of the derivative.
\begin{quote}\begin{description}
\sphinxlineitem{Parameters}\begin{itemize}
\item {} 
\sphinxAtStartPar
\sphinxstyleliteralstrong{\sphinxupquote{traces}} (\sphinxhref{https://numpy.org/doc/1.24/reference/generated/numpy.ndarray.html\#numpy.ndarray}{\sphinxcode{\sphinxupquote{ndarray}}}) \textendash{} An M neuron by N sample matrix

\item {} 
\sphinxAtStartPar
\sphinxstyleliteralstrong{\sphinxupquote{timeout}} (\sphinxhref{https://docs.python.org/3/library/functions.html\#int}{\sphinxcode{\sphinxupquote{int}}}, default: \sphinxcode{\sphinxupquote{15}}) \textendash{} timeout distance for peak finding (frames)

\item {} 
\sphinxAtStartPar
\sphinxstyleliteralstrong{\sphinxupquote{frame\_rate}} (\sphinxhref{https://docs.python.org/3/library/functions.html\#float}{\sphinxcode{\sphinxupquote{float}}}, default: \sphinxcode{\sphinxupquote{30.0}}) \textendash{} frame rate / time step for trace matrix

\item {} 
\sphinxAtStartPar
\sphinxstyleliteralstrong{\sphinxupquote{smooth}} (\sphinxhref{https://docs.python.org/3/library/functions.html\#bool}{\sphinxcode{\sphinxupquote{bool}}}, default: \sphinxcode{\sphinxupquote{True}}) \textendash{} boolean indicating whether to smooth first\sphinxhyphen{}time derivative

\item {} 
\sphinxAtStartPar
\sphinxstyleliteralstrong{\sphinxupquote{force\_nonneg}} (\sphinxhref{https://docs.python.org/3/library/functions.html\#bool}{\sphinxcode{\sphinxupquote{bool}}}, default: \sphinxcode{\sphinxupquote{True}}) \textendash{} boolean indicating whether to enforce non\sphinxhyphen{}negativity constraint on first\sphinxhyphen{}time derivative

\end{itemize}

\sphinxlineitem{Return type}
\sphinxAtStartPar
\sphinxhref{https://docs.python.org/3/library/typing.html\#typing.Tuple}{\sphinxcode{\sphinxupquote{Tuple}}}{[}\sphinxhref{https://docs.python.org/3/library/typing.html\#typing.List}{\sphinxcode{\sphinxupquote{List}}}{[}\sphinxhref{https://docs.python.org/3/library/functions.html\#int}{\sphinxcode{\sphinxupquote{int}}}{]}{]}

\sphinxlineitem{Returns}
\sphinxAtStartPar
An iterable where each element contains a sequence of frames identified as event onsets

\end{description}\end{quote}

\end{fulllineitems}

\index{normalize\_firing\_rates() (in module CalSciPy.event\_processing)@\spxentry{normalize\_firing\_rates()}\spxextra{in module CalSciPy.event\_processing}}

\begin{fulllineitems}
\phantomsection\label{\detokenize{CalSciPy.event_processing:CalSciPy.event_processing.normalize_firing_rates}}
\pysigstartsignatures
\pysiglinewithargsret{\sphinxcode{\sphinxupquote{CalSciPy.event\_processing.}}\sphinxbfcode{\sphinxupquote{normalize\_firing\_rates}}}{\emph{\DUrole{n}{firing\_matrix}\DUrole{p}{:}\DUrole{w}{  }\DUrole{n}{\sphinxhref{https://numpy.org/doc/1.24/reference/generated/numpy.ndarray.html\#numpy.ndarray}{numpy.ndarray}}}, \emph{\DUrole{n}{in\_place}\DUrole{p}{:}\DUrole{w}{  }\DUrole{n}{\sphinxhref{https://docs.python.org/3/library/functions.html\#bool}{bool}}\DUrole{w}{  }\DUrole{o}{=}\DUrole{w}{  }\DUrole{default_value}{False}}}{{ $\rightarrow$ \sphinxhref{https://numpy.org/doc/1.24/reference/generated/numpy.ndarray.html\#numpy.ndarray}{numpy.ndarray}}}
\pysigstopsignatures
\sphinxAtStartPar
Normalize firing rates by scaling to a max of 1.0. Non\sphinxhyphen{}negativity constrained.
\begin{quote}\begin{description}
\sphinxlineitem{Parameters}\begin{itemize}
\item {} 
\sphinxAtStartPar
\sphinxstyleliteralstrong{\sphinxupquote{firing\_matrix}} (\sphinxhref{https://numpy.org/doc/1.24/reference/generated/numpy.ndarray.html\#numpy.ndarray}{\sphinxcode{\sphinxupquote{ndarray}}}) \textendash{} matrix of n neuron x m samples where each element is either a spike or an
instantaneous firing rate

\item {} 
\sphinxAtStartPar
\sphinxstyleliteralstrong{\sphinxupquote{in\_place}} (\sphinxhref{https://docs.python.org/3/library/functions.html\#bool}{\sphinxcode{\sphinxupquote{bool}}}, default: \sphinxcode{\sphinxupquote{False}}) \textendash{} boolean indicating whether to perform calculation in\sphinxhyphen{}place

\end{itemize}

\sphinxlineitem{Return type}
\sphinxAtStartPar
\sphinxhref{https://numpy.org/doc/1.24/reference/generated/numpy.ndarray.html\#numpy.ndarray}{\sphinxcode{\sphinxupquote{ndarray}}}

\sphinxlineitem{Returns}
\sphinxAtStartPar
normalized firing rate matrix of n neurons x m samples

\end{description}\end{quote}

\end{fulllineitems}

\index{scale\_waveforms() (in module CalSciPy.event\_processing)@\spxentry{scale\_waveforms()}\spxextra{in module CalSciPy.event\_processing}}

\begin{fulllineitems}
\phantomsection\label{\detokenize{CalSciPy.event_processing:CalSciPy.event_processing.scale_waveforms}}
\pysigstartsignatures
\pysiglinewithargsret{\sphinxcode{\sphinxupquote{CalSciPy.event\_processing.}}\sphinxbfcode{\sphinxupquote{scale\_waveforms}}}{\emph{waveforms: typing.Iterable{[}numpy.ndarray{]}, scaler: typing.Callable = \textless{}class \textquotesingle{}sklearn.preprocessing.\_data.StandardScaler\textquotesingle{}\textgreater{}}}{{ $\rightarrow$ \sphinxhref{https://numpy.org/doc/1.24/reference/generated/numpy.ndarray.html\#numpy.ndarray}{numpy.ndarray}}}
\pysigstopsignatures
\sphinxAtStartPar
Scale waveforms for cross\sphinxhyphen{}neuron comparisons
\begin{quote}\begin{description}
\sphinxlineitem{Parameters}\begin{itemize}
\item {} 
\sphinxAtStartPar
\sphinxstyleliteralstrong{\sphinxupquote{waveforms}} (\sphinxhref{https://docs.python.org/3/library/typing.html\#typing.Iterable}{\sphinxcode{\sphinxupquote{Iterable}}}{[}\sphinxhref{https://numpy.org/doc/1.24/reference/generated/numpy.ndarray.html\#numpy.ndarray}{\sphinxcode{\sphinxupquote{ndarray}}}{]}) \textendash{} An Iterable of M events by N samples matrices of waveforms

\item {} 
\sphinxAtStartPar
\sphinxstyleliteralstrong{\sphinxupquote{scaler}} (\sphinxhref{https://docs.python.org/3/library/typing.html\#typing.Callable}{\sphinxcode{\sphinxupquote{Callable}}}, default: \sphinxcode{\sphinxupquote{\textless{}class \textquotesingle{}sklearn.preprocessing.\_data.StandardScaler\textquotesingle{}\textgreater{}}}) \textendash{} sklearn preprocessing object

\end{itemize}

\sphinxlineitem{Return type}
\sphinxAtStartPar
\sphinxhref{https://numpy.org/doc/1.24/reference/generated/numpy.ndarray.html\#numpy.ndarray}{\sphinxcode{\sphinxupquote{ndarray}}}

\sphinxlineitem{Returns}
\sphinxAtStartPar
An Iterable of M event by N samples scaled matrices of waveforms

\end{description}\end{quote}

\end{fulllineitems}


\sphinxstepscope


\chapter{CalSciPy.image\_processing module}
\label{\detokenize{CalSciPy.image_processing:module-CalSciPy.image_processing}}\label{\detokenize{CalSciPy.image_processing:calscipy-image-processing-module}}\label{\detokenize{CalSciPy.image_processing::doc}}\index{module@\spxentry{module}!CalSciPy.image\_processing@\spxentry{CalSciPy.image\_processing}}\index{CalSciPy.image\_processing@\spxentry{CalSciPy.image\_processing}!module@\spxentry{module}}\index{gaussian\_filter() (in module CalSciPy.image\_processing)@\spxentry{gaussian\_filter()}\spxextra{in module CalSciPy.image\_processing}}

\begin{fulllineitems}
\phantomsection\label{\detokenize{CalSciPy.image_processing:CalSciPy.image_processing.gaussian_filter}}
\pysigstartsignatures
\pysiglinewithargsret{\sphinxcode{\sphinxupquote{CalSciPy.image\_processing.}}\sphinxbfcode{\sphinxupquote{gaussian\_filter}}}{\emph{\DUrole{n}{images}\DUrole{p}{:}\DUrole{w}{  }\DUrole{n}{np.ndarray}}, \emph{\DUrole{n}{sigma}\DUrole{p}{:}\DUrole{w}{  }\DUrole{n}{Number\DUrole{w}{  }\DUrole{p}{|}\DUrole{w}{  }np.ndarry}\DUrole{w}{  }\DUrole{o}{=}\DUrole{w}{  }\DUrole{default_value}{1.0}}, \emph{\DUrole{n}{block\_size}\DUrole{p}{:}\DUrole{w}{  }\DUrole{n}{\sphinxhref{https://docs.python.org/3/library/functions.html\#int}{int}}\DUrole{w}{  }\DUrole{o}{=}\DUrole{w}{  }\DUrole{default_value}{None}}, \emph{\DUrole{n}{block\_buffer}\DUrole{p}{:}\DUrole{w}{  }\DUrole{n}{\sphinxhref{https://docs.python.org/3/library/functions.html\#int}{int}}\DUrole{w}{  }\DUrole{o}{=}\DUrole{w}{  }\DUrole{default_value}{0}}, \emph{\DUrole{n}{in\_place}\DUrole{p}{:}\DUrole{w}{  }\DUrole{n}{\sphinxhref{https://docs.python.org/3/library/functions.html\#bool}{bool}}\DUrole{w}{  }\DUrole{o}{=}\DUrole{w}{  }\DUrole{default_value}{False}}}{{ $\rightarrow$ np.ndarray}}
\pysigstopsignatures
\sphinxAtStartPar
GPU\sphinxhyphen{}parallelized multidimensional gaussian filter. Optional arguments for in\sphinxhyphen{}place calculation. Can be calculated
blockwise with overlapping or non\sphinxhyphen{}overlapping blocks.

\sphinxAtStartPar
Designed for use on arrays larger than the available memory capacity.

\sphinxAtStartPar
Footprint is of the form np.ones((frames, y pixels, x pixels)) with the origin in the center
\begin{quote}\begin{description}
\sphinxlineitem{Parameters}\begin{itemize}
\item {} 
\sphinxAtStartPar
\sphinxstyleliteralstrong{\sphinxupquote{images}} \textendash{} images stack to be filtered

\item {} 
\sphinxAtStartPar
\sphinxstyleliteralstrong{\sphinxupquote{sigma}} (default: \sphinxcode{\sphinxupquote{1.0}}) \textendash{} sigma for gaussian filter

\item {} 
\sphinxAtStartPar
\sphinxstyleliteralstrong{\sphinxupquote{block\_size}} (default: \sphinxcode{\sphinxupquote{None}}) \textendash{} the size of each block. Must fit within memory

\item {} 
\sphinxAtStartPar
\sphinxstyleliteralstrong{\sphinxupquote{block\_buffer}} (default: \sphinxcode{\sphinxupquote{0}}) \textendash{} the size of the overlapping region between block

\item {} 
\sphinxAtStartPar
\sphinxstyleliteralstrong{\sphinxupquote{in\_place}} (default: \sphinxcode{\sphinxupquote{False}}) \textendash{} whether to calculate in\sphinxhyphen{}place

\end{itemize}

\sphinxlineitem{Returns}
\sphinxAtStartPar
images: numpy array (frames, y pixels, x pixels)

\end{description}\end{quote}

\end{fulllineitems}

\index{median\_filter() (in module CalSciPy.image\_processing)@\spxentry{median\_filter()}\spxextra{in module CalSciPy.image\_processing}}

\begin{fulllineitems}
\phantomsection\label{\detokenize{CalSciPy.image_processing:CalSciPy.image_processing.median_filter}}
\pysigstartsignatures
\pysiglinewithargsret{\sphinxcode{\sphinxupquote{CalSciPy.image\_processing.}}\sphinxbfcode{\sphinxupquote{median\_filter}}}{\emph{\DUrole{n}{images}\DUrole{p}{:}\DUrole{w}{  }\DUrole{n}{\sphinxhref{https://numpy.org/doc/1.24/reference/generated/numpy.ndarray.html\#numpy.ndarray}{numpy.ndarray}}}, \emph{\DUrole{n}{mask}\DUrole{p}{:}\DUrole{w}{  }\DUrole{n}{\sphinxhref{https://numpy.org/doc/1.24/reference/generated/numpy.ndarray.html\#numpy.ndarray}{numpy.ndarray}}\DUrole{w}{  }\DUrole{o}{=}\DUrole{w}{  }\DUrole{default_value}{array({[}{[}{[}1., 1., 1.{]}, {[}1., 1., 1.{]}, {[}1., 1., 1.{]}{]}, {[}{[}1., 1., 1.{]}, {[}1., 1., 1.{]}, {[}1., 1., 1.{]}{]}, {[}{[}1., 1., 1.{]}, {[}1., 1., 1.{]}, {[}1., 1., 1.{]}{]}{]})}}, \emph{\DUrole{n}{block\_size}\DUrole{p}{:}\DUrole{w}{  }\DUrole{n}{\sphinxhref{https://docs.python.org/3/library/functions.html\#int}{int}\DUrole{w}{  }\DUrole{p}{|}\DUrole{w}{  }\sphinxhref{https://docs.python.org/3/library/constants.html\#None}{None}}\DUrole{w}{  }\DUrole{o}{=}\DUrole{w}{  }\DUrole{default_value}{None}}, \emph{\DUrole{n}{block\_buffer}\DUrole{p}{:}\DUrole{w}{  }\DUrole{n}{\sphinxhref{https://docs.python.org/3/library/functions.html\#int}{int}}\DUrole{w}{  }\DUrole{o}{=}\DUrole{w}{  }\DUrole{default_value}{0}}, \emph{\DUrole{n}{in\_place}\DUrole{p}{:}\DUrole{w}{  }\DUrole{n}{\sphinxhref{https://docs.python.org/3/library/functions.html\#bool}{bool}}\DUrole{w}{  }\DUrole{o}{=}\DUrole{w}{  }\DUrole{default_value}{False}}}{{ $\rightarrow$ \sphinxhref{https://numpy.org/doc/1.24/reference/generated/numpy.ndarray.html\#numpy.ndarray}{numpy.ndarray}}}
\pysigstopsignatures
\sphinxAtStartPar
GPU\sphinxhyphen{}parallelized multidimensional median filter. Optional arguments for in\sphinxhyphen{}place calculation. Can be calculated
blockwise with overlapping or non\sphinxhyphen{}overlapping blocks.

\sphinxAtStartPar
Designed for use on arrays larger than the available memory capacity.

\sphinxAtStartPar
Footprint is of the form np.ones((frames, y pixels, x pixels)) with the origin in the center
\begin{quote}\begin{description}
\sphinxlineitem{Parameters}\begin{itemize}
\item {} 
\sphinxAtStartPar
\sphinxstyleliteralstrong{\sphinxupquote{images}} (\sphinxhref{https://numpy.org/doc/1.24/reference/generated/numpy.ndarray.html\#numpy.ndarray}{\sphinxcode{\sphinxupquote{ndarray}}}) \textendash{} images stack to be filtered

\item {} 
\sphinxAtStartPar
\sphinxstyleliteralstrong{\sphinxupquote{mask}} (\sphinxhref{https://numpy.org/doc/1.24/reference/generated/numpy.ndarray.html\#numpy.ndarray}{\sphinxcode{\sphinxupquote{ndarray}}}, default: \sphinxcode{\sphinxupquote{array({[}{[}{[}1., 1., 1.{]},
        {[}1., 1., 1.{]},
        {[}1., 1., 1.{]}{]},

       {[}{[}1., 1., 1.{]},
        {[}1., 1., 1.{]},
        {[}1., 1., 1.{]}{]},

       {[}{[}1., 1., 1.{]},
        {[}1., 1., 1.{]},
        {[}1., 1., 1.{]}{]}{]})}}) \textendash{} mask of the median filter

\item {} 
\sphinxAtStartPar
\sphinxstyleliteralstrong{\sphinxupquote{block\_size}} (\sphinxhref{https://docs.python.org/3/library/typing.html\#typing.Optional}{\sphinxcode{\sphinxupquote{Optional}}}{[}\sphinxhref{https://docs.python.org/3/library/functions.html\#int}{\sphinxcode{\sphinxupquote{int}}}{]}, default: \sphinxcode{\sphinxupquote{None}}) \textendash{} the size of each block. Must fit within memory

\item {} 
\sphinxAtStartPar
\sphinxstyleliteralstrong{\sphinxupquote{block\_buffer}} (\sphinxhref{https://docs.python.org/3/library/functions.html\#int}{\sphinxcode{\sphinxupquote{int}}}, default: \sphinxcode{\sphinxupquote{0}}) \textendash{} the size of the overlapping region between block

\item {} 
\sphinxAtStartPar
\sphinxstyleliteralstrong{\sphinxupquote{in\_place}} (\sphinxhref{https://docs.python.org/3/library/functions.html\#bool}{\sphinxcode{\sphinxupquote{bool}}}, default: \sphinxcode{\sphinxupquote{False}}) \textendash{} whether to calculate in\sphinxhyphen{}place

\end{itemize}

\sphinxlineitem{Return type}
\sphinxAtStartPar
\sphinxhref{https://numpy.org/doc/1.24/reference/generated/numpy.ndarray.html\#numpy.ndarray}{\sphinxcode{\sphinxupquote{ndarray}}}

\sphinxlineitem{Returns}
\sphinxAtStartPar
images: numpy array (frames, y pixels, x pixels)

\end{description}\end{quote}

\end{fulllineitems}


\sphinxstepscope


\chapter{CalSciPy.interactive\_visuals module}
\label{\detokenize{CalSciPy.interactive_visuals:module-CalSciPy.interactive_visuals}}\label{\detokenize{CalSciPy.interactive_visuals:calscipy-interactive-visuals-module}}\label{\detokenize{CalSciPy.interactive_visuals::doc}}\index{module@\spxentry{module}!CalSciPy.interactive\_visuals@\spxentry{CalSciPy.interactive\_visuals}}\index{CalSciPy.interactive\_visuals@\spxentry{CalSciPy.interactive\_visuals}!module@\spxentry{module}}\index{interactive\_traces() (in module CalSciPy.interactive\_visuals)@\spxentry{interactive\_traces()}\spxextra{in module CalSciPy.interactive\_visuals}}

\begin{fulllineitems}
\phantomsection\label{\detokenize{CalSciPy.interactive_visuals:CalSciPy.interactive_visuals.interactive_traces}}
\pysigstartsignatures
\pysiglinewithargsret{\sphinxcode{\sphinxupquote{CalSciPy.interactive\_visuals.}}\sphinxbfcode{\sphinxupquote{interactive\_traces}}}{\emph{\DUrole{n}{traces}\DUrole{p}{:}\DUrole{w}{  }\DUrole{n}{\sphinxhref{https://numpy.org/doc/1.24/reference/generated/numpy.ndarray.html\#numpy.ndarray}{numpy.ndarray}}}, \emph{\DUrole{n}{frame\_rate}\DUrole{p}{:}\DUrole{w}{  }\DUrole{n}{\sphinxhref{https://docs.python.org/3/library/functions.html\#float}{float}}}, \emph{\DUrole{o}{**}\DUrole{n}{kwargs}}}{{ $\rightarrow$ \sphinxhref{https://docs.python.org/3/library/constants.html\#None}{None}}}
\pysigstopsignatures
\sphinxAtStartPar
Function to interactive compare traces. Press Up/Down to switch neurons
\begin{quote}\begin{description}
\sphinxlineitem{Parameters}\begin{itemize}
\item {} 
\sphinxAtStartPar
\sphinxstyleliteralstrong{\sphinxupquote{traces}} (\sphinxhref{https://numpy.org/doc/1.24/reference/generated/numpy.ndarray.html\#numpy.ndarray}{\sphinxcode{\sphinxupquote{ndarray}}}) \textendash{} primary traces

\item {} 
\sphinxAtStartPar
\sphinxstyleliteralstrong{\sphinxupquote{frame\_rate}} (\sphinxhref{https://docs.python.org/3/library/functions.html\#float}{\sphinxcode{\sphinxupquote{float}}}) \textendash{} frame rate

\end{itemize}

\sphinxlineitem{Return type}
\sphinxAtStartPar
\sphinxhref{https://docs.python.org/3/library/constants.html\#None}{\sphinxcode{\sphinxupquote{None}}}

\sphinxlineitem{Returns}
\sphinxAtStartPar
interactive figure

\end{description}\end{quote}

\end{fulllineitems}

\index{interactive\_traces\_compare() (in module CalSciPy.interactive\_visuals)@\spxentry{interactive\_traces\_compare()}\spxextra{in module CalSciPy.interactive\_visuals}}

\begin{fulllineitems}
\phantomsection\label{\detokenize{CalSciPy.interactive_visuals:CalSciPy.interactive_visuals.interactive_traces_compare}}
\pysigstartsignatures
\pysiglinewithargsret{\sphinxcode{\sphinxupquote{CalSciPy.interactive\_visuals.}}\sphinxbfcode{\sphinxupquote{interactive\_traces\_compare}}}{\emph{\DUrole{n}{traces}\DUrole{p}{:}\DUrole{w}{  }\DUrole{n}{\sphinxhref{https://numpy.org/doc/1.24/reference/generated/numpy.ndarray.html\#numpy.ndarray}{numpy.ndarray}}}, \emph{\DUrole{n}{traces2}\DUrole{p}{:}\DUrole{w}{  }\DUrole{n}{\sphinxhref{https://numpy.org/doc/1.24/reference/generated/numpy.ndarray.html\#numpy.ndarray}{numpy.ndarray}}}, \emph{\DUrole{n}{frame\_rate}\DUrole{p}{:}\DUrole{w}{  }\DUrole{n}{\sphinxhref{https://docs.python.org/3/library/functions.html\#float}{float}}}, \emph{\DUrole{o}{**}\DUrole{n}{kwargs}}}{{ $\rightarrow$ \sphinxhref{https://docs.python.org/3/library/constants.html\#None}{None}}}
\pysigstopsignatures
\sphinxAtStartPar
Function to interactively compare two sets of traces. Press Up/Down to switch neurons
\begin{quote}\begin{description}
\sphinxlineitem{Parameters}\begin{itemize}
\item {} 
\sphinxAtStartPar
\sphinxstyleliteralstrong{\sphinxupquote{traces}} (\sphinxhref{https://numpy.org/doc/1.24/reference/generated/numpy.ndarray.html\#numpy.ndarray}{\sphinxcode{\sphinxupquote{ndarray}}}) \textendash{} primary traces

\item {} 
\sphinxAtStartPar
\sphinxstyleliteralstrong{\sphinxupquote{traces2}} (\sphinxhref{https://numpy.org/doc/1.24/reference/generated/numpy.ndarray.html\#numpy.ndarray}{\sphinxcode{\sphinxupquote{ndarray}}}) \textendash{} secondary trace

\item {} 
\sphinxAtStartPar
\sphinxstyleliteralstrong{\sphinxupquote{frame\_rate}} (\sphinxhref{https://docs.python.org/3/library/functions.html\#float}{\sphinxcode{\sphinxupquote{float}}}) \textendash{} frame\_rate

\end{itemize}

\sphinxlineitem{Return type}
\sphinxAtStartPar
\sphinxhref{https://docs.python.org/3/library/constants.html\#None}{\sphinxcode{\sphinxupquote{None}}}

\sphinxlineitem{Returns}
\sphinxAtStartPar
interactive figure

\end{description}\end{quote}

\end{fulllineitems}

\index{interactive\_traces\_overlay() (in module CalSciPy.interactive\_visuals)@\spxentry{interactive\_traces\_overlay()}\spxextra{in module CalSciPy.interactive\_visuals}}

\begin{fulllineitems}
\phantomsection\label{\detokenize{CalSciPy.interactive_visuals:CalSciPy.interactive_visuals.interactive_traces_overlay}}
\pysigstartsignatures
\pysiglinewithargsret{\sphinxcode{\sphinxupquote{CalSciPy.interactive\_visuals.}}\sphinxbfcode{\sphinxupquote{interactive\_traces\_overlay}}}{\emph{\DUrole{n}{traces}\DUrole{p}{:}\DUrole{w}{  }\DUrole{n}{\sphinxhref{https://numpy.org/doc/1.24/reference/generated/numpy.ndarray.html\#numpy.ndarray}{numpy.ndarray}}}, \emph{\DUrole{n}{traces2}\DUrole{p}{:}\DUrole{w}{  }\DUrole{n}{\sphinxhref{https://numpy.org/doc/1.24/reference/generated/numpy.ndarray.html\#numpy.ndarray}{numpy.ndarray}}}, \emph{\DUrole{n}{frame\_rate}\DUrole{p}{:}\DUrole{w}{  }\DUrole{n}{\sphinxhref{https://docs.python.org/3/library/functions.html\#float}{float}}}, \emph{\DUrole{o}{**}\DUrole{n}{kwargs}}}{{ $\rightarrow$ \sphinxhref{https://docs.python.org/3/library/constants.html\#None}{None}}}
\pysigstopsignatures
\sphinxAtStartPar
Function to interactive compare traces with an overlay trace (e.g., noise). Press Up/Down to switch neurons
\begin{quote}\begin{description}
\sphinxlineitem{Parameters}\begin{itemize}
\item {} 
\sphinxAtStartPar
\sphinxstyleliteralstrong{\sphinxupquote{traces}} (\sphinxhref{https://numpy.org/doc/1.24/reference/generated/numpy.ndarray.html\#numpy.ndarray}{\sphinxcode{\sphinxupquote{ndarray}}}) \textendash{} primary traces

\item {} 
\sphinxAtStartPar
\sphinxstyleliteralstrong{\sphinxupquote{traces2}} (\sphinxhref{https://numpy.org/doc/1.24/reference/generated/numpy.ndarray.html\#numpy.ndarray}{\sphinxcode{\sphinxupquote{ndarray}}}) \textendash{} secondary trace

\item {} 
\sphinxAtStartPar
\sphinxstyleliteralstrong{\sphinxupquote{frame\_rate}} (\sphinxhref{https://docs.python.org/3/library/functions.html\#float}{\sphinxcode{\sphinxupquote{float}}}) \textendash{} frame\_rate

\end{itemize}

\sphinxlineitem{Return type}
\sphinxAtStartPar
\sphinxhref{https://docs.python.org/3/library/constants.html\#None}{\sphinxcode{\sphinxupquote{None}}}

\sphinxlineitem{Returns}
\sphinxAtStartPar
interactive figure

\end{description}\end{quote}

\end{fulllineitems}


\sphinxstepscope


\chapter{CalSciPy.io\_tools module}
\label{\detokenize{CalSciPy.io_tools:module-CalSciPy.io_tools}}\label{\detokenize{CalSciPy.io_tools:calscipy-io-tools-module}}\label{\detokenize{CalSciPy.io_tools::doc}}\index{module@\spxentry{module}!CalSciPy.io\_tools@\spxentry{CalSciPy.io\_tools}}\index{CalSciPy.io\_tools@\spxentry{CalSciPy.io\_tools}!module@\spxentry{module}}\index{load\_binary() (in module CalSciPy.io\_tools)@\spxentry{load\_binary()}\spxextra{in module CalSciPy.io\_tools}}

\begin{fulllineitems}
\phantomsection\label{\detokenize{CalSciPy.io_tools:CalSciPy.io_tools.load_binary}}
\pysigstartsignatures
\pysiglinewithargsret{\sphinxcode{\sphinxupquote{CalSciPy.io\_tools.}}\sphinxbfcode{\sphinxupquote{load\_binary}}}{\emph{\DUrole{n}{path}\DUrole{p}{:}\DUrole{w}{  }\DUrole{n}{\sphinxhref{https://docs.python.org/3/library/stdtypes.html\#str}{str}\DUrole{w}{  }\DUrole{p}{|}\DUrole{w}{  }\sphinxhref{https://docs.python.org/3/library/pathlib.html\#pathlib.Path}{pathlib.Path}}}, \emph{\DUrole{n}{mapped}\DUrole{p}{:}\DUrole{w}{  }\DUrole{n}{\sphinxhref{https://docs.python.org/3/library/functions.html\#bool}{bool}}\DUrole{w}{  }\DUrole{o}{=}\DUrole{w}{  }\DUrole{default_value}{False}}}{{ $\rightarrow$ \sphinxhref{https://numpy.org/doc/1.24/reference/generated/numpy.ndarray.html\#numpy.ndarray}{numpy.ndarray}\DUrole{w}{  }\DUrole{p}{|}\DUrole{w}{  }\sphinxhref{https://numpy.org/doc/1.24/reference/generated/numpy.memmap.html\#numpy.memmap}{numpy.memmap}}}
\pysigstopsignatures
\sphinxAtStartPar
This function loads images saved in language\sphinxhyphen{}agnostic binary format. Ideal for optimal read/write speeds and
highly\sphinxhyphen{}robust to corruption. However, the downside is that the images and their metadata are split into two
separate files. Images are saved with the \sphinxstyleemphasis{.bin} extension, while metadata is saved with extension \sphinxstyleemphasis{.json}.
If for some reason you lose the metadata, you can still load the binary if you know three of the following:
number of frames, y\sphinxhyphen{}pixels, x\sphinxhyphen{}pixels, and the datatype (\sphinxhref{https://numpy.org/doc/1.24/reference/generated/numpy.dtype.html\#numpy.dtype}{\sphinxcode{\sphinxupquote{numpy.dtype}}})
\begin{quote}\begin{description}
\sphinxlineitem{Parameters}\begin{itemize}
\item {} 
\sphinxAtStartPar
\sphinxstyleliteralstrong{\sphinxupquote{path}} (\sphinxhref{https://docs.python.org/3/library/typing.html\#typing.Union}{\sphinxcode{\sphinxupquote{Union}}}{[}\sphinxhref{https://docs.python.org/3/library/stdtypes.html\#str}{\sphinxcode{\sphinxupquote{str}}}, \sphinxhref{https://docs.python.org/3/library/pathlib.html\#pathlib.Path}{\sphinxcode{\sphinxupquote{Path}}}{]}) \textendash{} folder containing binary file

\item {} 
\sphinxAtStartPar
\sphinxstyleliteralstrong{\sphinxupquote{mapped}} (\sphinxhref{https://docs.python.org/3/library/functions.html\#bool}{\sphinxcode{\sphinxupquote{bool}}}, default: \sphinxcode{\sphinxupquote{False}}) \textendash{} boolean indicating whether to load image using memory\sphinxhyphen{}mapping

\end{itemize}

\sphinxlineitem{Return type}
\sphinxAtStartPar
\sphinxhref{https://docs.python.org/3/library/typing.html\#typing.Union}{\sphinxcode{\sphinxupquote{Union}}}{[}\sphinxhref{https://numpy.org/doc/1.24/reference/generated/numpy.ndarray.html\#numpy.ndarray}{\sphinxcode{\sphinxupquote{ndarray}}}, \sphinxhref{https://numpy.org/doc/1.24/reference/generated/numpy.memmap.html\#numpy.memmap}{\sphinxcode{\sphinxupquote{memmap}}}{]}

\sphinxlineitem{Returns}
\sphinxAtStartPar
image (frames, y\sphinxhyphen{}pixels, x\sphinxhyphen{}pixels)

\end{description}\end{quote}

\end{fulllineitems}

\index{load\_images() (in module CalSciPy.io\_tools)@\spxentry{load\_images()}\spxextra{in module CalSciPy.io\_tools}}

\begin{fulllineitems}
\phantomsection\label{\detokenize{CalSciPy.io_tools:CalSciPy.io_tools.load_images}}
\pysigstartsignatures
\pysiglinewithargsret{\sphinxcode{\sphinxupquote{CalSciPy.io\_tools.}}\sphinxbfcode{\sphinxupquote{load\_images}}}{\emph{\DUrole{n}{path}\DUrole{p}{:}\DUrole{w}{  }\DUrole{n}{\sphinxhref{https://docs.python.org/3/library/stdtypes.html\#str}{str}\DUrole{w}{  }\DUrole{p}{|}\DUrole{w}{  }\sphinxhref{https://docs.python.org/3/library/pathlib.html\#pathlib.Path}{pathlib.Path}}}}{{ $\rightarrow$ \sphinxhref{https://numpy.org/doc/1.24/reference/generated/numpy.ndarray.html\#numpy.ndarray}{numpy.ndarray}}}
\pysigstopsignatures
\sphinxAtStartPar
Load images into a numpy array. If path is a folder, all .tif files found non\sphinxhyphen{}recursively in the directory will be
compiled to a single array.
\begin{quote}\begin{description}
\sphinxlineitem{Parameters}
\sphinxAtStartPar
\sphinxstyleliteralstrong{\sphinxupquote{path}} (\sphinxhref{https://docs.python.org/3/library/typing.html\#typing.Union}{\sphinxcode{\sphinxupquote{Union}}}{[}\sphinxhref{https://docs.python.org/3/library/stdtypes.html\#str}{\sphinxcode{\sphinxupquote{str}}}, \sphinxhref{https://docs.python.org/3/library/pathlib.html\#pathlib.Path}{\sphinxcode{\sphinxupquote{Path}}}{]}) \textendash{} a file containing images or a folder containing several imaging stacks

\sphinxlineitem{Return type}
\sphinxAtStartPar
\sphinxhref{https://numpy.org/doc/1.24/reference/generated/numpy.ndarray.html\#numpy.ndarray}{\sphinxcode{\sphinxupquote{ndarray}}}

\sphinxlineitem{Returns}
\sphinxAtStartPar
numpy array (frames, y\sphinxhyphen{}pixels, x\sphinxhyphen{}pixels)

\end{description}\end{quote}

\end{fulllineitems}

\index{save\_binary() (in module CalSciPy.io\_tools)@\spxentry{save\_binary()}\spxextra{in module CalSciPy.io\_tools}}

\begin{fulllineitems}
\phantomsection\label{\detokenize{CalSciPy.io_tools:CalSciPy.io_tools.save_binary}}
\pysigstartsignatures
\pysiglinewithargsret{\sphinxcode{\sphinxupquote{CalSciPy.io\_tools.}}\sphinxbfcode{\sphinxupquote{save\_binary}}}{\emph{\DUrole{n}{path}\DUrole{p}{:}\DUrole{w}{  }\DUrole{n}{\sphinxhref{https://docs.python.org/3/library/stdtypes.html\#str}{str}\DUrole{w}{  }\DUrole{p}{|}\DUrole{w}{  }\sphinxhref{https://docs.python.org/3/library/pathlib.html\#pathlib.Path}{pathlib.Path}}}, \emph{\DUrole{n}{images}\DUrole{p}{:}\DUrole{w}{  }\DUrole{n}{\sphinxhref{https://numpy.org/doc/1.24/reference/generated/numpy.ndarray.html\#numpy.ndarray}{numpy.ndarray}}}}{{ $\rightarrow$ \sphinxhref{https://docs.python.org/3/library/functions.html\#int}{int}}}
\pysigstopsignatures
\sphinxAtStartPar
Save images to language\sphinxhyphen{}agnostic binary format. Ideal for optimal read/write speeds and highly\sphinxhyphen{}robust to corruption.
However, the downside is that the images and their metadata are split into two separate files. Images are saved with
the \sphinxstyleemphasis{.bin} extension, while metadata is saved with extension \sphinxstyleemphasis{.json}. If for some reason you lose the metadata, you
can still load the binary if you know three of the following: number of frames, y\sphinxhyphen{}pixels, x\sphinxhyphen{}pixels, and the
datatype. The datatype is almost always unsigned 16\sphinxhyphen{}bit (\sphinxcode{\sphinxupquote{numpy.uint16}}) for all modern imaging
systems\textendash{}even if they are collected at 12 or 13\sphinxhyphen{}bit.
\begin{quote}\begin{description}
\sphinxlineitem{Parameters}
\sphinxAtStartPar
\sphinxstyleliteralstrong{\sphinxupquote{path}} (\sphinxhref{https://docs.python.org/3/library/typing.html\#typing.Union}{\sphinxcode{\sphinxupquote{Union}}}{[}\sphinxhref{https://docs.python.org/3/library/stdtypes.html\#str}{\sphinxcode{\sphinxupquote{str}}}, \sphinxhref{https://docs.python.org/3/library/pathlib.html\#pathlib.Path}{\sphinxcode{\sphinxupquote{Path}}}{]}) \textendash{} path to save images to. The path stem is considered the filename if it doesn’t have any extension. If

\end{description}\end{quote}

\begin{DUlineblock}{0em}
\item[] no filename is provided then the default filename is \sphinxstyleemphasis{binary\_video}.
\end{DUlineblock}
\begin{quote}\begin{description}
\sphinxlineitem{Parameters}
\sphinxAtStartPar
\sphinxstyleliteralstrong{\sphinxupquote{images}} (\sphinxhref{https://numpy.org/doc/1.24/reference/generated/numpy.ndarray.html\#numpy.ndarray}{\sphinxcode{\sphinxupquote{ndarray}}}) \textendash{} images to save (frames, y\sphinxhyphen{}pixels, x\sphinxhyphen{}pixels)

\sphinxlineitem{Return type}
\sphinxAtStartPar
\sphinxhref{https://docs.python.org/3/library/functions.html\#int}{\sphinxcode{\sphinxupquote{int}}}

\sphinxlineitem{Returns}
\sphinxAtStartPar
0 if successful

\end{description}\end{quote}

\end{fulllineitems}

\index{save\_images() (in module CalSciPy.io\_tools)@\spxentry{save\_images()}\spxextra{in module CalSciPy.io\_tools}}

\begin{fulllineitems}
\phantomsection\label{\detokenize{CalSciPy.io_tools:CalSciPy.io_tools.save_images}}
\pysigstartsignatures
\pysiglinewithargsret{\sphinxcode{\sphinxupquote{CalSciPy.io\_tools.}}\sphinxbfcode{\sphinxupquote{save\_images}}}{\emph{\DUrole{n}{path}\DUrole{p}{:}\DUrole{w}{  }\DUrole{n}{\sphinxhref{https://docs.python.org/3/library/stdtypes.html\#str}{str}\DUrole{w}{  }\DUrole{p}{|}\DUrole{w}{  }\sphinxhref{https://docs.python.org/3/library/pathlib.html\#pathlib.Path}{pathlib.Path}}}, \emph{\DUrole{n}{images}\DUrole{p}{:}\DUrole{w}{  }\DUrole{n}{\sphinxhref{https://numpy.org/doc/1.24/reference/generated/numpy.ndarray.html\#numpy.ndarray}{numpy.ndarray}}}, \emph{\DUrole{n}{size\_cap}\DUrole{p}{:}\DUrole{w}{  }\DUrole{n}{\sphinxhref{https://docs.python.org/3/library/functions.html\#float}{float}}\DUrole{w}{  }\DUrole{o}{=}\DUrole{w}{  }\DUrole{default_value}{3.9}}}{{ $\rightarrow$ \sphinxhref{https://docs.python.org/3/library/functions.html\#int}{int}}}
\pysigstopsignatures
\sphinxAtStartPar
Save a numpy array to a single .tif file. If size \textgreater{} 4GB then saved as a series of files. If path is not a file and
already exists the default filename will be \sphinxstyleemphasis{images}.
\begin{quote}\begin{description}
\sphinxlineitem{Parameters}\begin{itemize}
\item {} 
\sphinxAtStartPar
\sphinxstyleliteralstrong{\sphinxupquote{path}} (\sphinxhref{https://docs.python.org/3/library/typing.html\#typing.Union}{\sphinxcode{\sphinxupquote{Union}}}{[}\sphinxhref{https://docs.python.org/3/library/stdtypes.html\#str}{\sphinxcode{\sphinxupquote{str}}}, \sphinxhref{https://docs.python.org/3/library/pathlib.html\#pathlib.Path}{\sphinxcode{\sphinxupquote{Path}}}{]}) \textendash{} filename or absolute path

\item {} 
\sphinxAtStartPar
\sphinxstyleliteralstrong{\sphinxupquote{images}} (\sphinxhref{https://numpy.org/doc/1.24/reference/generated/numpy.ndarray.html\#numpy.ndarray}{\sphinxcode{\sphinxupquote{ndarray}}}) \textendash{} numpy array (frames, y pixels, x pixels)

\item {} 
\sphinxAtStartPar
\sphinxstyleliteralstrong{\sphinxupquote{size\_cap}} (\sphinxhref{https://docs.python.org/3/library/functions.html\#float}{\sphinxcode{\sphinxupquote{float}}}, default: \sphinxcode{\sphinxupquote{3.9}}) \textendash{} maximum size per file

\end{itemize}

\sphinxlineitem{Return type}
\sphinxAtStartPar
\sphinxhref{https://docs.python.org/3/library/functions.html\#int}{\sphinxcode{\sphinxupquote{int}}}

\sphinxlineitem{Returns}
\sphinxAtStartPar
returns 0 if successful

\end{description}\end{quote}

\end{fulllineitems}


\sphinxstepscope


\chapter{CalSciPy.misc module}
\label{\detokenize{CalSciPy.misc:module-CalSciPy.misc}}\label{\detokenize{CalSciPy.misc:calscipy-misc-module}}\label{\detokenize{CalSciPy.misc::doc}}\index{module@\spxentry{module}!CalSciPy.misc@\spxentry{CalSciPy.misc}}\index{CalSciPy.misc@\spxentry{CalSciPy.misc}!module@\spxentry{module}}\index{PatternMatching (class in CalSciPy.misc)@\spxentry{PatternMatching}\spxextra{class in CalSciPy.misc}}

\begin{fulllineitems}
\phantomsection\label{\detokenize{CalSciPy.misc:CalSciPy.misc.PatternMatching}}
\pysigstartsignatures
\pysiglinewithargsret{\sphinxbfcode{\sphinxupquote{class\DUrole{w}{  }}}\sphinxcode{\sphinxupquote{CalSciPy.misc.}}\sphinxbfcode{\sphinxupquote{PatternMatching}}}{\emph{\DUrole{n}{value}\DUrole{p}{:}\DUrole{w}{  }\DUrole{n}{\sphinxhref{https://docs.python.org/3/library/typing.html\#typing.Any}{Any}}}, \emph{\DUrole{n}{comparison\_expressions}\DUrole{p}{:}\DUrole{w}{  }\DUrole{n}{\sphinxhref{https://docs.python.org/3/library/typing.html\#typing.Iterable}{Iterable}\DUrole{p}{{[}}\sphinxhref{https://docs.python.org/3/library/typing.html\#typing.Any}{Any}\DUrole{p}{{]}}}}}{}
\pysigstopsignatures
\sphinxAtStartPar
Bases: \sphinxhref{https://docs.python.org/3/library/functions.html\#object}{\sphinxcode{\sphinxupquote{object}}}

\end{fulllineitems}

\index{calculate\_frames\_per\_file() (in module CalSciPy.misc)@\spxentry{calculate\_frames\_per\_file()}\spxextra{in module CalSciPy.misc}}

\begin{fulllineitems}
\phantomsection\label{\detokenize{CalSciPy.misc:CalSciPy.misc.calculate_frames_per_file}}
\pysigstartsignatures
\pysiglinewithargsret{\sphinxcode{\sphinxupquote{CalSciPy.misc.}}\sphinxbfcode{\sphinxupquote{calculate\_frames\_per\_file}}}{\emph{\DUrole{n}{y\_pixels: int}}, \emph{\DUrole{n}{x\_pixels: int}}, \emph{\DUrole{n}{bit\_depth: numpy.dtype = \textless{}class \textquotesingle{}numpy.uint16\textquotesingle{}\textgreater{}}}, \emph{\DUrole{n}{size\_cap: numbers.Number = 3.9}}}{{ $\rightarrow$ \sphinxhref{https://docs.python.org/3/library/functions.html\#int}{int}}}
\pysigstopsignatures
\sphinxAtStartPar
Estimates the number of image frames to allocate to each file given some maximum size.
\begin{quote}\begin{description}
\sphinxlineitem{Parameters}\begin{itemize}
\item {} 
\sphinxAtStartPar
\sphinxstyleliteralstrong{\sphinxupquote{y\_pixels}} (\sphinxhref{https://docs.python.org/3/library/functions.html\#int}{\sphinxcode{\sphinxupquote{int}}}) \textendash{} number of y\_pixels in image

\item {} 
\sphinxAtStartPar
\sphinxstyleliteralstrong{\sphinxupquote{x\_pixels}} (\sphinxhref{https://docs.python.org/3/library/functions.html\#int}{\sphinxcode{\sphinxupquote{int}}}) \textendash{} number of x\_pixels in image

\item {} 
\sphinxAtStartPar
\sphinxstyleliteralstrong{\sphinxupquote{bit\_depth}} (\sphinxhref{https://numpy.org/doc/1.24/reference/generated/numpy.dtype.html\#numpy.dtype}{\sphinxcode{\sphinxupquote{dtype}}}, default: \sphinxcode{\sphinxupquote{\textless{}class \textquotesingle{}numpy.uint16\textquotesingle{}\textgreater{}}}) \textendash{} bit\sphinxhyphen{}depth / type of image elements

\item {} 
\sphinxAtStartPar
\sphinxstyleliteralstrong{\sphinxupquote{size\_cap}} (\sphinxhref{https://docs.python.org/3/library/numbers.html\#numbers.Number}{\sphinxcode{\sphinxupquote{Number}}}, default: \sphinxcode{\sphinxupquote{3.9}}) \textendash{} maximum file size

\end{itemize}

\sphinxlineitem{Return type}
\sphinxAtStartPar
\sphinxhref{https://docs.python.org/3/library/functions.html\#int}{\sphinxcode{\sphinxupquote{int}}}

\sphinxlineitem{Returns}
\sphinxAtStartPar
the maximum number of frames to allocate for each file

\end{description}\end{quote}

\end{fulllineitems}

\index{generate\_blocks() (in module CalSciPy.misc)@\spxentry{generate\_blocks()}\spxextra{in module CalSciPy.misc}}

\begin{fulllineitems}
\phantomsection\label{\detokenize{CalSciPy.misc:CalSciPy.misc.generate_blocks}}
\pysigstartsignatures
\pysiglinewithargsret{\sphinxcode{\sphinxupquote{CalSciPy.misc.}}\sphinxbfcode{\sphinxupquote{generate\_blocks}}}{\emph{\DUrole{n}{sequence}\DUrole{p}{:}\DUrole{w}{  }\DUrole{n}{\sphinxhref{https://docs.python.org/3/library/typing.html\#typing.Iterable}{Iterable}}}, \emph{\DUrole{n}{block\_size}\DUrole{p}{:}\DUrole{w}{  }\DUrole{n}{\sphinxhref{https://docs.python.org/3/library/functions.html\#int}{int}}}, \emph{\DUrole{n}{block\_buffer}\DUrole{p}{:}\DUrole{w}{  }\DUrole{n}{\sphinxhref{https://docs.python.org/3/library/functions.html\#int}{int}}\DUrole{w}{  }\DUrole{o}{=}\DUrole{w}{  }\DUrole{default_value}{0}}}{{ $\rightarrow$ \sphinxhref{https://docs.python.org/3/library/typing.html\#typing.Iterator}{Iterator}}}
\pysigstopsignatures
\sphinxAtStartPar
Returns a generator of some arbitrary iterable sequence that yields m blocks with overlapping regions of size n
\begin{quote}\begin{description}
\sphinxlineitem{Parameters}\begin{itemize}
\item {} 
\sphinxAtStartPar
\sphinxstyleliteralstrong{\sphinxupquote{sequence}} (\sphinxhref{https://docs.python.org/3/library/typing.html\#typing.Iterable}{\sphinxcode{\sphinxupquote{Iterable}}}) \textendash{} Sequence to be split into overlapping blocks

\item {} 
\sphinxAtStartPar
\sphinxstyleliteralstrong{\sphinxupquote{block\_size}} (\sphinxhref{https://docs.python.org/3/library/functions.html\#int}{\sphinxcode{\sphinxupquote{int}}}) \textendash{} size of blocks

\item {} 
\sphinxAtStartPar
\sphinxstyleliteralstrong{\sphinxupquote{block\_buffer}} (\sphinxhref{https://docs.python.org/3/library/functions.html\#int}{\sphinxcode{\sphinxupquote{int}}}, default: \sphinxcode{\sphinxupquote{0}}) \textendash{} size of overlap between blocks

\end{itemize}

\sphinxlineitem{Return type}
\sphinxAtStartPar
\sphinxhref{https://docs.python.org/3/library/typing.html\#typing.Iterator}{\sphinxcode{\sphinxupquote{Iterator}}}

\sphinxlineitem{Returns}
\sphinxAtStartPar
generator yielding m blocks with overlapping regions of size n

\end{description}\end{quote}

\end{fulllineitems}

\index{generate\_overlapping\_blocks() (in module CalSciPy.misc)@\spxentry{generate\_overlapping\_blocks()}\spxextra{in module CalSciPy.misc}}

\begin{fulllineitems}
\phantomsection\label{\detokenize{CalSciPy.misc:CalSciPy.misc.generate_overlapping_blocks}}
\pysigstartsignatures
\pysiglinewithargsret{\sphinxcode{\sphinxupquote{CalSciPy.misc.}}\sphinxbfcode{\sphinxupquote{generate\_overlapping\_blocks}}}{\emph{\DUrole{n}{sequence}\DUrole{p}{:}\DUrole{w}{  }\DUrole{n}{\sphinxhref{https://docs.python.org/3/library/typing.html\#typing.Iterable}{Iterable}}}, \emph{\DUrole{n}{block\_size}\DUrole{p}{:}\DUrole{w}{  }\DUrole{n}{\sphinxhref{https://docs.python.org/3/library/functions.html\#int}{int}}}, \emph{\DUrole{n}{block\_buffer}\DUrole{p}{:}\DUrole{w}{  }\DUrole{n}{\sphinxhref{https://docs.python.org/3/library/functions.html\#int}{int}}}}{{ $\rightarrow$ \sphinxhref{https://docs.python.org/3/library/typing.html\#typing.Iterator}{Iterator}}}
\pysigstopsignatures
\sphinxAtStartPar
Returns a generator of some arbitrary iterable sequence that yields m blocks with overlapping regions of size n
\begin{quote}\begin{description}
\sphinxlineitem{Parameters}\begin{itemize}
\item {} 
\sphinxAtStartPar
\sphinxstyleliteralstrong{\sphinxupquote{sequence}} (\sphinxhref{https://docs.python.org/3/library/typing.html\#typing.Iterable}{\sphinxcode{\sphinxupquote{Iterable}}}) \textendash{} Sequence to be split into overlapping blocks

\item {} 
\sphinxAtStartPar
\sphinxstyleliteralstrong{\sphinxupquote{block\_size}} (\sphinxhref{https://docs.python.org/3/library/functions.html\#int}{\sphinxcode{\sphinxupquote{int}}}) \textendash{} size of blocks

\item {} 
\sphinxAtStartPar
\sphinxstyleliteralstrong{\sphinxupquote{block\_buffer}} (\sphinxhref{https://docs.python.org/3/library/functions.html\#int}{\sphinxcode{\sphinxupquote{int}}}) \textendash{} size of overlap between blocks

\end{itemize}

\sphinxlineitem{Return type}
\sphinxAtStartPar
\sphinxhref{https://docs.python.org/3/library/typing.html\#typing.Iterator}{\sphinxcode{\sphinxupquote{Iterator}}}

\sphinxlineitem{Returns}
\sphinxAtStartPar
generator yielding m blocks with overlapping regions of size n

\end{description}\end{quote}

\end{fulllineitems}

\index{generate\_padded\_filename() (in module CalSciPy.misc)@\spxentry{generate\_padded\_filename()}\spxextra{in module CalSciPy.misc}}

\begin{fulllineitems}
\phantomsection\label{\detokenize{CalSciPy.misc:CalSciPy.misc.generate_padded_filename}}
\pysigstartsignatures
\pysiglinewithargsret{\sphinxcode{\sphinxupquote{CalSciPy.misc.}}\sphinxbfcode{\sphinxupquote{generate\_padded\_filename}}}{\emph{\DUrole{n}{output\_folder}\DUrole{p}{:}\DUrole{w}{  }\DUrole{n}{\sphinxhref{https://docs.python.org/3/library/pathlib.html\#pathlib.Path}{pathlib.Path}}}, \emph{\DUrole{n}{index}\DUrole{p}{:}\DUrole{w}{  }\DUrole{n}{\sphinxhref{https://docs.python.org/3/library/functions.html\#int}{int}}}, \emph{\DUrole{n}{base}\DUrole{p}{:}\DUrole{w}{  }\DUrole{n}{\sphinxhref{https://docs.python.org/3/library/stdtypes.html\#str}{str}}\DUrole{w}{  }\DUrole{o}{=}\DUrole{w}{  }\DUrole{default_value}{\textquotesingle{}images\textquotesingle{}}}, \emph{\DUrole{n}{digits}\DUrole{p}{:}\DUrole{w}{  }\DUrole{n}{\sphinxhref{https://docs.python.org/3/library/functions.html\#int}{int}}\DUrole{w}{  }\DUrole{o}{=}\DUrole{w}{  }\DUrole{default_value}{2}}, \emph{\DUrole{n}{ext}\DUrole{p}{:}\DUrole{w}{  }\DUrole{n}{\sphinxhref{https://docs.python.org/3/library/stdtypes.html\#str}{str}}\DUrole{w}{  }\DUrole{o}{=}\DUrole{w}{  }\DUrole{default_value}{\textquotesingle{}.tif\textquotesingle{}}}}{{ $\rightarrow$ \sphinxhref{https://docs.python.org/3/library/pathlib.html\#pathlib.Path}{pathlib.Path}}}
\pysigstopsignatures
\sphinxAtStartPar
Generates a pathlib Path whose name is defined as ‘\{base\}\_\{index\}\{ext\}’ where index is zero\sphinxhyphen{}padded if it
is not equal to the number of digits
\begin{quote}\begin{description}
\sphinxlineitem{Parameters}\begin{itemize}
\item {} 
\sphinxAtStartPar
\sphinxstyleliteralstrong{\sphinxupquote{output\_folder}} (\sphinxhref{https://docs.python.org/3/library/pathlib.html\#pathlib.Path}{\sphinxcode{\sphinxupquote{Path}}}) \textendash{} folder that will contain file

\item {} 
\sphinxAtStartPar
\sphinxstyleliteralstrong{\sphinxupquote{index}} (\sphinxhref{https://docs.python.org/3/library/functions.html\#int}{\sphinxcode{\sphinxupquote{int}}}) \textendash{} index of file

\item {} 
\sphinxAtStartPar
\sphinxstyleliteralstrong{\sphinxupquote{base}} (\sphinxhref{https://docs.python.org/3/library/stdtypes.html\#str}{\sphinxcode{\sphinxupquote{str}}}, default: \sphinxcode{\sphinxupquote{\textquotesingle{}images\textquotesingle{}}}) \textendash{} base tag of file

\item {} 
\sphinxAtStartPar
\sphinxstyleliteralstrong{\sphinxupquote{digits}} (\sphinxhref{https://docs.python.org/3/library/functions.html\#int}{\sphinxcode{\sphinxupquote{int}}}, default: \sphinxcode{\sphinxupquote{2}}) \textendash{} number of digits for representing index

\item {} 
\sphinxAtStartPar
\sphinxstyleliteralstrong{\sphinxupquote{ext}} (\sphinxhref{https://docs.python.org/3/library/stdtypes.html\#str}{\sphinxcode{\sphinxupquote{str}}}, default: \sphinxcode{\sphinxupquote{\textquotesingle{}.tif\textquotesingle{}}}) \textendash{} file extension

\end{itemize}

\sphinxlineitem{Return type}
\sphinxAtStartPar
\sphinxhref{https://docs.python.org/3/library/pathlib.html\#pathlib.Path}{\sphinxcode{\sphinxupquote{Path}}}

\sphinxlineitem{Returns}
\sphinxAtStartPar
generated filename

\end{description}\end{quote}

\end{fulllineitems}

\index{generate\_sliding\_window() (in module CalSciPy.misc)@\spxentry{generate\_sliding\_window()}\spxextra{in module CalSciPy.misc}}

\begin{fulllineitems}
\phantomsection\label{\detokenize{CalSciPy.misc:CalSciPy.misc.generate_sliding_window}}
\pysigstartsignatures
\pysiglinewithargsret{\sphinxcode{\sphinxupquote{CalSciPy.misc.}}\sphinxbfcode{\sphinxupquote{generate\_sliding\_window}}}{\emph{\DUrole{n}{sequence}\DUrole{p}{:}\DUrole{w}{  }\DUrole{n}{\sphinxhref{https://docs.python.org/3/library/typing.html\#typing.Iterable}{Iterable}}}, \emph{\DUrole{n}{window\_length}\DUrole{p}{:}\DUrole{w}{  }\DUrole{n}{\sphinxhref{https://docs.python.org/3/library/functions.html\#int}{int}}}, \emph{\DUrole{n}{step\_size}\DUrole{p}{:}\DUrole{w}{  }\DUrole{n}{\sphinxhref{https://docs.python.org/3/library/functions.html\#int}{int}}\DUrole{w}{  }\DUrole{o}{=}\DUrole{w}{  }\DUrole{default_value}{1}}}{{ $\rightarrow$ \sphinxhref{https://numpy.org/doc/1.24/reference/generated/numpy.ndarray.html\#numpy.ndarray}{numpy.ndarray}}}
\pysigstopsignatures\begin{quote}\begin{description}
\sphinxlineitem{Return type}
\sphinxAtStartPar
\sphinxhref{https://numpy.org/doc/1.24/reference/generated/numpy.ndarray.html\#numpy.ndarray}{\sphinxcode{\sphinxupquote{ndarray}}}

\end{description}\end{quote}

\end{fulllineitems}

\index{sliding\_window() (in module CalSciPy.misc)@\spxentry{sliding\_window()}\spxextra{in module CalSciPy.misc}}

\begin{fulllineitems}
\phantomsection\label{\detokenize{CalSciPy.misc:CalSciPy.misc.sliding_window}}
\pysigstartsignatures
\pysiglinewithargsret{\sphinxcode{\sphinxupquote{CalSciPy.misc.}}\sphinxbfcode{\sphinxupquote{sliding\_window}}}{\emph{\DUrole{n}{sequence}\DUrole{p}{:}\DUrole{w}{  }\DUrole{n}{\sphinxhref{https://numpy.org/doc/1.24/reference/generated/numpy.ndarray.html\#numpy.ndarray}{numpy.ndarray}}}, \emph{\DUrole{n}{window\_length}\DUrole{p}{:}\DUrole{w}{  }\DUrole{n}{\sphinxhref{https://docs.python.org/3/library/functions.html\#int}{int}}}, \emph{\DUrole{n}{function}\DUrole{p}{:}\DUrole{w}{  }\DUrole{n}{\sphinxhref{https://docs.python.org/3/library/typing.html\#typing.Callable}{Callable}}}, \emph{\DUrole{o}{*}\DUrole{n}{args}}, \emph{\DUrole{o}{**}\DUrole{n}{kwargs}}}{{ $\rightarrow$ \sphinxhref{https://numpy.org/doc/1.24/reference/generated/numpy.ndarray.html\#numpy.ndarray}{numpy.ndarray}}}
\pysigstopsignatures\begin{quote}\begin{description}
\sphinxlineitem{Return type}
\sphinxAtStartPar
\sphinxhref{https://numpy.org/doc/1.24/reference/generated/numpy.ndarray.html\#numpy.ndarray}{\sphinxcode{\sphinxupquote{ndarray}}}

\end{description}\end{quote}

\end{fulllineitems}

\index{wrap\_cupy\_block() (in module CalSciPy.misc)@\spxentry{wrap\_cupy\_block()}\spxextra{in module CalSciPy.misc}}

\begin{fulllineitems}
\phantomsection\label{\detokenize{CalSciPy.misc:CalSciPy.misc.wrap_cupy_block}}
\pysigstartsignatures
\pysiglinewithargsret{\sphinxcode{\sphinxupquote{CalSciPy.misc.}}\sphinxbfcode{\sphinxupquote{wrap\_cupy\_block}}}{\emph{\DUrole{n}{cupy\_function}\DUrole{p}{:}\DUrole{w}{  }\DUrole{n}{\sphinxhref{https://docs.python.org/3/library/typing.html\#typing.Callable}{Callable}}}}{{ $\rightarrow$ \sphinxhref{https://docs.python.org/3/library/typing.html\#typing.Callable}{Callable}}}
\pysigstopsignatures
\sphinxAtStartPar
Wraps a cupy function such that incoming numpy arrays are converting to cupy arrays and swapped back on return
\begin{quote}\begin{description}
\sphinxlineitem{Parameters}
\sphinxAtStartPar
\sphinxstyleliteralstrong{\sphinxupquote{cupy\_function}} (\sphinxhref{https://docs.python.org/3/library/typing.html\#typing.Callable}{\sphinxcode{\sphinxupquote{Callable}}}) \textendash{} any cupy function that accepts numpy arrays

\sphinxlineitem{Return type}
\sphinxAtStartPar
\sphinxhref{https://docs.python.org/3/library/typing.html\#typing.Callable}{\sphinxcode{\sphinxupquote{Callable}}}

\sphinxlineitem{Returns}
\sphinxAtStartPar
wrapped function

\end{description}\end{quote}

\end{fulllineitems}


\sphinxstepscope


\chapter{CalSciPy.reorganization module}
\label{\detokenize{CalSciPy.reorganization:module-CalSciPy.reorganization}}\label{\detokenize{CalSciPy.reorganization:calscipy-reorganization-module}}\label{\detokenize{CalSciPy.reorganization::doc}}\index{module@\spxentry{module}!CalSciPy.reorganization@\spxentry{CalSciPy.reorganization}}\index{CalSciPy.reorganization@\spxentry{CalSciPy.reorganization}!module@\spxentry{module}}\index{generate\_raster() (in module CalSciPy.reorganization)@\spxentry{generate\_raster()}\spxextra{in module CalSciPy.reorganization}}

\begin{fulllineitems}
\phantomsection\label{\detokenize{CalSciPy.reorganization:CalSciPy.reorganization.generate_raster}}
\pysigstartsignatures
\pysiglinewithargsret{\sphinxcode{\sphinxupquote{CalSciPy.reorganization.}}\sphinxbfcode{\sphinxupquote{generate\_raster}}}{\emph{\DUrole{n}{event\_frames}\DUrole{p}{:}\DUrole{w}{  }\DUrole{n}{\sphinxhref{https://docs.python.org/3/library/typing.html\#typing.Iterable}{Iterable}\DUrole{p}{{[}}\sphinxhref{https://docs.python.org/3/library/typing.html\#typing.Iterable}{Iterable}\DUrole{p}{{[}}\sphinxhref{https://docs.python.org/3/library/functions.html\#int}{int}\DUrole{p}{{]}}\DUrole{p}{{]}}}}, \emph{\DUrole{n}{total\_frames}\DUrole{p}{:}\DUrole{w}{  }\DUrole{n}{\sphinxhref{https://docs.python.org/3/library/functions.html\#int}{int}\DUrole{w}{  }\DUrole{p}{|}\DUrole{w}{  }\sphinxhref{https://docs.python.org/3/library/constants.html\#None}{None}}\DUrole{w}{  }\DUrole{o}{=}\DUrole{w}{  }\DUrole{default_value}{None}}}{{ $\rightarrow$ \sphinxhref{https://numpy.org/doc/1.24/reference/generated/numpy.ndarray.html\#numpy.ndarray}{numpy.ndarray}}}
\pysigstopsignatures
\sphinxAtStartPar
Generate raster from an iterable of iterables containing the spike or event times for each neuron
\begin{quote}\begin{description}
\sphinxlineitem{Parameters}\begin{itemize}
\item {} 
\sphinxAtStartPar
\sphinxstyleliteralstrong{\sphinxupquote{event\_frames}} (\sphinxhref{https://docs.python.org/3/library/typing.html\#typing.Iterable}{\sphinxcode{\sphinxupquote{Iterable}}}{[}\sphinxhref{https://docs.python.org/3/library/typing.html\#typing.Iterable}{\sphinxcode{\sphinxupquote{Iterable}}}{[}\sphinxhref{https://docs.python.org/3/library/functions.html\#int}{\sphinxcode{\sphinxupquote{int}}}{]}{]}) \textendash{} iterable containing an iterable identifying the event frames for each neuron

\item {} 
\sphinxAtStartPar
\sphinxstyleliteralstrong{\sphinxupquote{total\_frames}} (\sphinxhref{https://docs.python.org/3/library/typing.html\#typing.Optional}{\sphinxcode{\sphinxupquote{Optional}}}{[}\sphinxhref{https://docs.python.org/3/library/functions.html\#int}{\sphinxcode{\sphinxupquote{int}}}{]}, default: \sphinxcode{\sphinxupquote{None}}) \textendash{} total number of frames

\end{itemize}

\sphinxlineitem{Return type}
\sphinxAtStartPar
\sphinxhref{https://numpy.org/doc/1.24/reference/generated/numpy.ndarray.html\#numpy.ndarray}{\sphinxcode{\sphinxupquote{ndarray}}}

\sphinxlineitem{Returns}
\sphinxAtStartPar
event matrix of neurons x total frames

\end{description}\end{quote}

\end{fulllineitems}

\index{generate\_tensor() (in module CalSciPy.reorganization)@\spxentry{generate\_tensor()}\spxextra{in module CalSciPy.reorganization}}

\begin{fulllineitems}
\phantomsection\label{\detokenize{CalSciPy.reorganization:CalSciPy.reorganization.generate_tensor}}
\pysigstartsignatures
\pysiglinewithargsret{\sphinxcode{\sphinxupquote{CalSciPy.reorganization.}}\sphinxbfcode{\sphinxupquote{generate\_tensor}}}{\emph{\DUrole{n}{traces\_as\_matrix}\DUrole{p}{:}\DUrole{w}{  }\DUrole{n}{\sphinxhref{https://numpy.org/doc/1.24/reference/generated/numpy.ndarray.html\#numpy.ndarray}{numpy.ndarray}}}, \emph{\DUrole{n}{chunk\_size}\DUrole{p}{:}\DUrole{w}{  }\DUrole{n}{\sphinxhref{https://docs.python.org/3/library/functions.html\#int}{int}}}}{{ $\rightarrow$ \sphinxhref{https://numpy.org/doc/1.24/reference/generated/numpy.ndarray.html\#numpy.ndarray}{numpy.ndarray}}}
\pysigstopsignatures
\sphinxAtStartPar
Generates a tensor given chunk / trial indices
\begin{quote}\begin{description}
\sphinxlineitem{Parameters}\begin{itemize}
\item {} 
\sphinxAtStartPar
\sphinxstyleliteralstrong{\sphinxupquote{traces\_as\_matrix}} (\sphinxhref{https://numpy.org/doc/1.24/reference/generated/numpy.ndarray.html\#numpy.ndarray}{\sphinxcode{\sphinxupquote{ndarray}}}) \textendash{} traces in matrix form (neurons x frames)

\item {} 
\sphinxAtStartPar
\sphinxstyleliteralstrong{\sphinxupquote{chunk\_size}} (\sphinxhref{https://docs.python.org/3/library/functions.html\#int}{\sphinxcode{\sphinxupquote{int}}}) \textendash{} size of each chunk

\end{itemize}

\sphinxlineitem{Return type}
\sphinxAtStartPar
\sphinxhref{https://numpy.org/doc/1.24/reference/generated/numpy.ndarray.html\#numpy.ndarray}{\sphinxcode{\sphinxupquote{ndarray}}}

\sphinxlineitem{Returns}
\sphinxAtStartPar
traces as a tensor of trial x neurons x frames

\end{description}\end{quote}

\end{fulllineitems}

\index{merge\_factorized\_matrices() (in module CalSciPy.reorganization)@\spxentry{merge\_factorized\_matrices()}\spxextra{in module CalSciPy.reorganization}}

\begin{fulllineitems}
\phantomsection\label{\detokenize{CalSciPy.reorganization:CalSciPy.reorganization.merge_factorized_matrices}}
\pysigstartsignatures
\pysiglinewithargsret{\sphinxcode{\sphinxupquote{CalSciPy.reorganization.}}\sphinxbfcode{\sphinxupquote{merge\_factorized\_matrices}}}{\emph{\DUrole{n}{factorized\_traces}\DUrole{p}{:}\DUrole{w}{  }\DUrole{n}{\sphinxhref{https://numpy.org/doc/1.24/reference/generated/numpy.ndarray.html\#numpy.ndarray}{numpy.ndarray}}}, \emph{\DUrole{n}{component}\DUrole{p}{:}\DUrole{w}{  }\DUrole{n}{\sphinxhref{https://docs.python.org/3/library/functions.html\#int}{int}}\DUrole{w}{  }\DUrole{o}{=}\DUrole{w}{  }\DUrole{default_value}{0}}}{{ $\rightarrow$ \sphinxhref{https://numpy.org/doc/1.24/reference/generated/numpy.ndarray.html\#numpy.ndarray}{numpy.ndarray}}}
\pysigstopsignatures
\sphinxAtStartPar
Concatenate a neuron x chunk or trial array in which each element is a component x frame factorization of the
original trace:
\begin{quote}\begin{description}
\sphinxlineitem{Parameters}\begin{itemize}
\item {} 
\sphinxAtStartPar
\sphinxstyleliteralstrong{\sphinxupquote{factorized\_traces}} (\sphinxhref{https://numpy.org/doc/1.24/reference/generated/numpy.ndarray.html\#numpy.ndarray}{\sphinxcode{\sphinxupquote{ndarray}}}) \textendash{} neurons x chunks (trial, tif, etc) containing the neuron’s trace factorized
into several components

\item {} 
\sphinxAtStartPar
\sphinxstyleliteralstrong{\sphinxupquote{component}} (\sphinxhref{https://docs.python.org/3/library/functions.html\#int}{\sphinxcode{\sphinxupquote{int}}}, default: \sphinxcode{\sphinxupquote{0}}) \textendash{} specific component to extract

\end{itemize}

\sphinxlineitem{Return type}
\sphinxAtStartPar
\sphinxhref{https://numpy.org/doc/1.24/reference/generated/numpy.ndarray.html\#numpy.ndarray}{\sphinxcode{\sphinxupquote{ndarray}}}

\sphinxlineitem{Returns}
\sphinxAtStartPar
traces of specific component in matrix form

\end{description}\end{quote}

\end{fulllineitems}

\index{merge\_tensor() (in module CalSciPy.reorganization)@\spxentry{merge\_tensor()}\spxextra{in module CalSciPy.reorganization}}

\begin{fulllineitems}
\phantomsection\label{\detokenize{CalSciPy.reorganization:CalSciPy.reorganization.merge_tensor}}
\pysigstartsignatures
\pysiglinewithargsret{\sphinxcode{\sphinxupquote{CalSciPy.reorganization.}}\sphinxbfcode{\sphinxupquote{merge\_tensor}}}{\emph{\DUrole{n}{traces\_as\_tensor}\DUrole{p}{:}\DUrole{w}{  }\DUrole{n}{\sphinxhref{https://numpy.org/doc/1.24/reference/generated/numpy.ndarray.html\#numpy.ndarray}{numpy.ndarray}}}}{{ $\rightarrow$ \sphinxhref{https://numpy.org/doc/1.24/reference/generated/numpy.ndarray.html\#numpy.ndarray}{numpy.ndarray}}}
\pysigstopsignatures
\sphinxAtStartPar
Concatenate multiple trials or tiffs into single matrix:
\begin{quote}\begin{description}
\sphinxlineitem{Parameters}
\sphinxAtStartPar
\sphinxstyleliteralstrong{\sphinxupquote{traces\_as\_tensor}} (\sphinxhref{https://numpy.org/doc/1.24/reference/generated/numpy.ndarray.html\#numpy.ndarray}{\sphinxcode{\sphinxupquote{ndarray}}}) \textendash{} chunk (trial, tif, etc) x neurons x frames

\sphinxlineitem{Return type}
\sphinxAtStartPar
\sphinxhref{https://numpy.org/doc/1.24/reference/generated/numpy.ndarray.html\#numpy.ndarray}{\sphinxcode{\sphinxupquote{ndarray}}}

\sphinxlineitem{Returns}
\sphinxAtStartPar
traces in matrix form (neurons x frames)

\end{description}\end{quote}

\end{fulllineitems}


\sphinxstepscope


\chapter{CalSciPy.trace\_processing module}
\label{\detokenize{CalSciPy.trace_processing:module-CalSciPy.trace_processing}}\label{\detokenize{CalSciPy.trace_processing:calscipy-trace-processing-module}}\label{\detokenize{CalSciPy.trace_processing::doc}}\index{module@\spxentry{module}!CalSciPy.trace\_processing@\spxentry{CalSciPy.trace\_processing}}\index{CalSciPy.trace\_processing@\spxentry{CalSciPy.trace\_processing}!module@\spxentry{module}}\index{calculate\_dfof() (in module CalSciPy.trace\_processing)@\spxentry{calculate\_dfof()}\spxextra{in module CalSciPy.trace\_processing}}

\begin{fulllineitems}
\phantomsection\label{\detokenize{CalSciPy.trace_processing:CalSciPy.trace_processing.calculate_dfof}}
\pysigstartsignatures
\pysiglinewithargsret{\sphinxcode{\sphinxupquote{CalSciPy.trace\_processing.}}\sphinxbfcode{\sphinxupquote{calculate\_dfof}}}{\emph{\DUrole{n}{traces}\DUrole{p}{:}\DUrole{w}{  }\DUrole{n}{\sphinxhref{https://numpy.org/doc/1.24/reference/generated/numpy.ndarray.html\#numpy.ndarray}{numpy.ndarray}}}, \emph{\DUrole{n}{frame\_rate}\DUrole{p}{:}\DUrole{w}{  }\DUrole{n}{\sphinxhref{https://docs.python.org/3/library/functions.html\#float}{float}}\DUrole{w}{  }\DUrole{o}{=}\DUrole{w}{  }\DUrole{default_value}{30.0}}, \emph{\DUrole{n}{in\_place}\DUrole{p}{:}\DUrole{w}{  }\DUrole{n}{\sphinxhref{https://docs.python.org/3/library/functions.html\#bool}{bool}}\DUrole{w}{  }\DUrole{o}{=}\DUrole{w}{  }\DUrole{default_value}{False}}, \emph{\DUrole{n}{offset}\DUrole{p}{:}\DUrole{w}{  }\DUrole{n}{\sphinxhref{https://docs.python.org/3/library/functions.html\#float}{float}}\DUrole{w}{  }\DUrole{o}{=}\DUrole{w}{  }\DUrole{default_value}{0.0}}, \emph{\DUrole{n}{external\_reference}\DUrole{p}{:}\DUrole{w}{  }\DUrole{n}{\sphinxhref{https://numpy.org/doc/1.24/reference/generated/numpy.ndarray.html\#numpy.ndarray}{numpy.ndarray}\DUrole{w}{  }\DUrole{p}{|}\DUrole{w}{  }\sphinxhref{https://docs.python.org/3/library/constants.html\#None}{None}}\DUrole{w}{  }\DUrole{o}{=}\DUrole{w}{  }\DUrole{default_value}{None}}}{{ $\rightarrow$ \sphinxhref{https://numpy.org/doc/1.24/reference/generated/numpy.ndarray.html\#numpy.ndarray}{numpy.ndarray}}}
\pysigstopsignatures
\sphinxAtStartPar
Calculates Δf/f0 (fold fluorescence over baseline). Baseline is defined as the 5th percentile of the signal
after a 1Hz low\sphinxhyphen{}pass filter using a Hamming window. Baseline can be calculated using an external reference
| using the raw argument or adjusted by using the offset argument. Supports in\sphinxhyphen{}place calculation
| (off by default).
\begin{quote}\begin{description}
\sphinxlineitem{Parameters}\begin{itemize}
\item {} 
\sphinxAtStartPar
\sphinxstyleliteralstrong{\sphinxupquote{traces}} (\sphinxhref{https://numpy.org/doc/1.24/reference/generated/numpy.ndarray.html\#numpy.ndarray}{\sphinxcode{\sphinxupquote{ndarray}}}) \textendash{} matrix of traces in the form of neurons x frames

\item {} 
\sphinxAtStartPar
\sphinxstyleliteralstrong{\sphinxupquote{frame\_rate}} (\sphinxhref{https://docs.python.org/3/library/functions.html\#float}{\sphinxcode{\sphinxupquote{float}}}, default: \sphinxcode{\sphinxupquote{30.0}}) \textendash{} frame rate of dataset

\item {} 
\sphinxAtStartPar
\sphinxstyleliteralstrong{\sphinxupquote{in\_place}} (\sphinxhref{https://docs.python.org/3/library/functions.html\#bool}{\sphinxcode{\sphinxupquote{bool}}}, default: \sphinxcode{\sphinxupquote{False}}) \textendash{} boolean indicating whether to perform calculation in\sphinxhyphen{}place

\item {} 
\sphinxAtStartPar
\sphinxstyleliteralstrong{\sphinxupquote{offset}} (\sphinxhref{https://docs.python.org/3/library/functions.html\#float}{\sphinxcode{\sphinxupquote{float}}}, default: \sphinxcode{\sphinxupquote{0.0}}) \textendash{} offset added to baseline; useful if traces are non\sphinxhyphen{}negative

\item {} 
\sphinxAtStartPar
\sphinxstyleliteralstrong{\sphinxupquote{external\_reference}} (\sphinxhref{https://docs.python.org/3/library/typing.html\#typing.Optional}{\sphinxcode{\sphinxupquote{Optional}}}{[}\sphinxhref{https://numpy.org/doc/1.24/reference/generated/numpy.ndarray.html\#numpy.ndarray}{\sphinxcode{\sphinxupquote{ndarray}}}{]}, default: \sphinxcode{\sphinxupquote{None}}) \textendash{} secondary dataset used to calculate baseline; useful if traces have been factorized

\end{itemize}

\sphinxlineitem{Return type}
\sphinxAtStartPar
\sphinxhref{https://numpy.org/doc/1.24/reference/generated/numpy.ndarray.html\#numpy.ndarray}{\sphinxcode{\sphinxupquote{ndarray}}}

\sphinxlineitem{Returns}
\sphinxAtStartPar
Δf/f0 matrix of n neurons x m samples

\end{description}\end{quote}

\end{fulllineitems}

\index{calculate\_standardized\_noise() (in module CalSciPy.trace\_processing)@\spxentry{calculate\_standardized\_noise()}\spxextra{in module CalSciPy.trace\_processing}}

\begin{fulllineitems}
\phantomsection\label{\detokenize{CalSciPy.trace_processing:CalSciPy.trace_processing.calculate_standardized_noise}}
\pysigstartsignatures
\pysiglinewithargsret{\sphinxcode{\sphinxupquote{CalSciPy.trace\_processing.}}\sphinxbfcode{\sphinxupquote{calculate\_standardized\_noise}}}{\emph{\DUrole{n}{fold\_fluorescence\_over\_baseline}\DUrole{p}{:}\DUrole{w}{  }\DUrole{n}{\sphinxhref{https://numpy.org/doc/1.24/reference/generated/numpy.ndarray.html\#numpy.ndarray}{numpy.ndarray}}}, \emph{\DUrole{n}{frame\_rate}\DUrole{p}{:}\DUrole{w}{  }\DUrole{n}{\sphinxhref{https://docs.python.org/3/library/functions.html\#float}{float}}\DUrole{w}{  }\DUrole{o}{=}\DUrole{w}{  }\DUrole{default_value}{30.0}}}{{ $\rightarrow$ \sphinxhref{https://numpy.org/doc/1.24/reference/generated/numpy.ndarray.html\#numpy.ndarray}{numpy.ndarray}}}
\pysigstopsignatures\begin{description}
\sphinxlineitem{Calculates a frame\sphinxhyphen{}rate independent standardized noise as defined as:}
\begin{DUlineblock}{0em}
\item[] \(v = \frac{\sigma \frac{\Delta F}F}\sqrt{f}\)
\end{DUlineblock}

\end{description}

\sphinxAtStartPar
It is robust against outliers and approximates the standard deviation of Δf/f0 baseline fluctuations.
For comparison, the more exquisite of the Allen Brain Institute’s public datasets are approximately 1*\%Hz\textasciicircum{}(\sphinxhyphen{}1/2)
\begin{quote}\begin{description}
\sphinxlineitem{Parameters}\begin{itemize}
\item {} 
\sphinxAtStartPar
\sphinxstyleliteralstrong{\sphinxupquote{fold\_fluorescence\_over\_baseline}} (\sphinxhref{https://numpy.org/doc/1.24/reference/generated/numpy.ndarray.html\#numpy.ndarray}{\sphinxcode{\sphinxupquote{ndarray}}}) \textendash{} fold fluorescence over baseline (i.e., Δf/f0)

\item {} 
\sphinxAtStartPar
\sphinxstyleliteralstrong{\sphinxupquote{frame\_rate}} (\sphinxhref{https://docs.python.org/3/library/functions.html\#float}{\sphinxcode{\sphinxupquote{float}}}, default: \sphinxcode{\sphinxupquote{30.0}}) \textendash{} frame rate of dataset

\end{itemize}

\sphinxlineitem{Return type}
\sphinxAtStartPar
\sphinxhref{https://numpy.org/doc/1.24/reference/generated/numpy.ndarray.html\#numpy.ndarray}{\sphinxcode{\sphinxupquote{ndarray}}}

\sphinxlineitem{Returns}
\sphinxAtStartPar
standardized noise (units are  1*\%Hz\textasciicircum{}(\sphinxhyphen{}1/2) ) for each neuron

\end{description}\end{quote}

\end{fulllineitems}

\index{detrend\_polynomial() (in module CalSciPy.trace\_processing)@\spxentry{detrend\_polynomial()}\spxextra{in module CalSciPy.trace\_processing}}

\begin{fulllineitems}
\phantomsection\label{\detokenize{CalSciPy.trace_processing:CalSciPy.trace_processing.detrend_polynomial}}
\pysigstartsignatures
\pysiglinewithargsret{\sphinxcode{\sphinxupquote{CalSciPy.trace\_processing.}}\sphinxbfcode{\sphinxupquote{detrend\_polynomial}}}{\emph{\DUrole{n}{traces}\DUrole{p}{:}\DUrole{w}{  }\DUrole{n}{\sphinxhref{https://numpy.org/doc/1.24/reference/generated/numpy.ndarray.html\#numpy.ndarray}{numpy.ndarray}}}, \emph{\DUrole{n}{in\_place}\DUrole{p}{:}\DUrole{w}{  }\DUrole{n}{\sphinxhref{https://docs.python.org/3/library/functions.html\#bool}{bool}}\DUrole{w}{  }\DUrole{o}{=}\DUrole{w}{  }\DUrole{default_value}{False}}}{{ $\rightarrow$ \sphinxhref{https://numpy.org/doc/1.24/reference/generated/numpy.ndarray.html\#numpy.ndarray}{numpy.ndarray}}}
\pysigstopsignatures
\sphinxAtStartPar
Detrend traces using a fourth\sphinxhyphen{}order polynomial
\begin{quote}\begin{description}
\sphinxlineitem{Parameters}\begin{itemize}
\item {} 
\sphinxAtStartPar
\sphinxstyleliteralstrong{\sphinxupquote{traces}} (\sphinxhref{https://numpy.org/doc/1.24/reference/generated/numpy.ndarray.html\#numpy.ndarray}{\sphinxcode{\sphinxupquote{ndarray}}}) \textendash{} matrix of traces in the form of neurons x frames

\item {} 
\sphinxAtStartPar
\sphinxstyleliteralstrong{\sphinxupquote{in\_place}} (\sphinxhref{https://docs.python.org/3/library/functions.html\#bool}{\sphinxcode{\sphinxupquote{bool}}}, default: \sphinxcode{\sphinxupquote{False}}) \textendash{} boolean indicating whether to perform calculation in\sphinxhyphen{}place

\end{itemize}

\sphinxlineitem{Return type}
\sphinxAtStartPar
\sphinxhref{https://numpy.org/doc/1.24/reference/generated/numpy.ndarray.html\#numpy.ndarray}{\sphinxcode{\sphinxupquote{ndarray}}}

\sphinxlineitem{Returns}
\sphinxAtStartPar
detrended traces

\end{description}\end{quote}

\end{fulllineitems}

\index{perona\_malik\_diffusion() (in module CalSciPy.trace\_processing)@\spxentry{perona\_malik\_diffusion()}\spxextra{in module CalSciPy.trace\_processing}}

\begin{fulllineitems}
\phantomsection\label{\detokenize{CalSciPy.trace_processing:CalSciPy.trace_processing.perona_malik_diffusion}}
\pysigstartsignatures
\pysiglinewithargsret{\sphinxcode{\sphinxupquote{CalSciPy.trace\_processing.}}\sphinxbfcode{\sphinxupquote{perona\_malik\_diffusion}}}{\emph{\DUrole{n}{traces}\DUrole{p}{:}\DUrole{w}{  }\DUrole{n}{\sphinxhref{https://numpy.org/doc/1.24/reference/generated/numpy.ndarray.html\#numpy.ndarray}{numpy.ndarray}}}, \emph{\DUrole{n}{iters}\DUrole{p}{:}\DUrole{w}{  }\DUrole{n}{\sphinxhref{https://docs.python.org/3/library/functions.html\#int}{int}}\DUrole{w}{  }\DUrole{o}{=}\DUrole{w}{  }\DUrole{default_value}{25}}, \emph{\DUrole{n}{kappa}\DUrole{p}{:}\DUrole{w}{  }\DUrole{n}{\sphinxhref{https://docs.python.org/3/library/functions.html\#float}{float}}\DUrole{w}{  }\DUrole{o}{=}\DUrole{w}{  }\DUrole{default_value}{0.15}}, \emph{\DUrole{n}{gamma}\DUrole{p}{:}\DUrole{w}{  }\DUrole{n}{\sphinxhref{https://docs.python.org/3/library/functions.html\#float}{float}}\DUrole{w}{  }\DUrole{o}{=}\DUrole{w}{  }\DUrole{default_value}{0.25}}, \emph{\DUrole{n}{in\_place}\DUrole{p}{:}\DUrole{w}{  }\DUrole{n}{\sphinxhref{https://docs.python.org/3/library/functions.html\#bool}{bool}}\DUrole{w}{  }\DUrole{o}{=}\DUrole{w}{  }\DUrole{default_value}{False}}}{{ $\rightarrow$ \sphinxhref{https://numpy.org/doc/1.24/reference/generated/numpy.ndarray.html\#numpy.ndarray}{numpy.ndarray}}}
\pysigstopsignatures
\sphinxAtStartPar
Edge\sphinxhyphen{}preserving smoothing using perona malik diffusion. This is a non\sphinxhyphen{}linear smoothing technique that avoids the
temporal distortion introduced onto traces by standard gaussian smoothing.

\sphinxAtStartPar
The parameter \sphinxtitleref{kappa} controls the level of smoothing (“diffusion”) as a function of the derivative of the trace
(or “gradient” in the case of 2D images where this algorithm is often used). This function is known as the
diffusion coefficient. When the derivative for some portion of the trace is low, the algorithm will encourage
smoothing to reduce noise. If the derivative is large like during a burst of activity, the algorithm will discourage
smoothing to maintain its structure. Here, the argument \sphinxtitleref{kappa} is multiplied by the dynamic range to generate the
true kappa.

\sphinxAtStartPar
The diffusion coefficient implemented here is e\textasciicircum{}(\sphinxhyphen{}(derivative/kappa)\textasciicircum{}2).

\sphinxAtStartPar
Perona\sphinxhyphen{}Malik diffusion is an iterative process. The parameter \sphinxtitleref{gamma} controls the rate of diffusion, while
parameter \sphinxtitleref{iters} sets the number of iterations to perform.

\sphinxAtStartPar
This implementation is currently situated to handle 1\sphinxhyphen{}D vectors because it gives us some performance benefits.
\begin{quote}\begin{description}
\sphinxlineitem{Parameters}\begin{itemize}
\item {} 
\sphinxAtStartPar
\sphinxstyleliteralstrong{\sphinxupquote{traces}} (\sphinxhref{https://numpy.org/doc/1.24/reference/generated/numpy.ndarray.html\#numpy.ndarray}{\sphinxcode{\sphinxupquote{ndarray}}}) \textendash{} matrix of M neurons by N samples

\item {} 
\sphinxAtStartPar
\sphinxstyleliteralstrong{\sphinxupquote{iters}} (\sphinxhref{https://docs.python.org/3/library/functions.html\#int}{\sphinxcode{\sphinxupquote{int}}}, default: \sphinxcode{\sphinxupquote{25}}) \textendash{} number of iterations

\item {} 
\sphinxAtStartPar
\sphinxstyleliteralstrong{\sphinxupquote{kappa}} (\sphinxhref{https://docs.python.org/3/library/functions.html\#float}{\sphinxcode{\sphinxupquote{float}}}, default: \sphinxcode{\sphinxupquote{0.15}}) \textendash{} used to calculate the true kappa, where true kappa = kappa * dynamic range. range 0\sphinxhyphen{}1

\item {} 
\sphinxAtStartPar
\sphinxstyleliteralstrong{\sphinxupquote{gamma}} (\sphinxhref{https://docs.python.org/3/library/functions.html\#float}{\sphinxcode{\sphinxupquote{float}}}, default: \sphinxcode{\sphinxupquote{0.25}}) \textendash{} rate of diffusion for each iter. range 0\sphinxhyphen{}1

\item {} 
\sphinxAtStartPar
\sphinxstyleliteralstrong{\sphinxupquote{in\_place}} (\sphinxhref{https://docs.python.org/3/library/functions.html\#bool}{\sphinxcode{\sphinxupquote{bool}}}, default: \sphinxcode{\sphinxupquote{False}}) \textendash{} whether to calculate in\sphinxhyphen{}place

\end{itemize}

\sphinxlineitem{Return type}
\sphinxAtStartPar
\sphinxhref{https://numpy.org/doc/1.24/reference/generated/numpy.ndarray.html\#numpy.ndarray}{\sphinxcode{\sphinxupquote{ndarray}}}

\sphinxlineitem{Returns}
\sphinxAtStartPar
smoothed traces

\end{description}\end{quote}

\end{fulllineitems}



\chapter{Indices and tables}
\label{\detokenize{index:indices-and-tables}}\begin{itemize}
\item {} 
\sphinxAtStartPar
\DUrole{xref,std,std-ref}{genindex}

\item {} 
\sphinxAtStartPar
\DUrole{xref,std,std-ref}{modindex}

\item {} 
\sphinxAtStartPar
\DUrole{xref,std,std-ref}{search}

\end{itemize}


\renewcommand{\indexname}{Python Module Index}
\begin{sphinxtheindex}
\let\bigletter\sphinxstyleindexlettergroup
\bigletter{c}
\item\relax\sphinxstyleindexentry{CalSciPy.bruker}\sphinxstyleindexpageref{CalSciPy.bruker:\detokenize{module-CalSciPy.bruker}}
\item\relax\sphinxstyleindexentry{CalSciPy.coloring}\sphinxstyleindexpageref{CalSciPy.coloring:\detokenize{module-CalSciPy.coloring}}
\item\relax\sphinxstyleindexentry{CalSciPy.event\_processing}\sphinxstyleindexpageref{CalSciPy.event_processing:\detokenize{module-CalSciPy.event_processing}}
\item\relax\sphinxstyleindexentry{CalSciPy.image\_processing}\sphinxstyleindexpageref{CalSciPy.image_processing:\detokenize{module-CalSciPy.image_processing}}
\item\relax\sphinxstyleindexentry{CalSciPy.interactive\_visuals}\sphinxstyleindexpageref{CalSciPy.interactive_visuals:\detokenize{module-CalSciPy.interactive_visuals}}
\item\relax\sphinxstyleindexentry{CalSciPy.io\_tools}\sphinxstyleindexpageref{CalSciPy.io_tools:\detokenize{module-CalSciPy.io_tools}}
\item\relax\sphinxstyleindexentry{CalSciPy.misc}\sphinxstyleindexpageref{CalSciPy.misc:\detokenize{module-CalSciPy.misc}}
\item\relax\sphinxstyleindexentry{CalSciPy.reorganization}\sphinxstyleindexpageref{CalSciPy.reorganization:\detokenize{module-CalSciPy.reorganization}}
\item\relax\sphinxstyleindexentry{CalSciPy.trace\_processing}\sphinxstyleindexpageref{CalSciPy.trace_processing:\detokenize{module-CalSciPy.trace_processing}}
\end{sphinxtheindex}

\renewcommand{\indexname}{Index}
\printindex
\end{document}